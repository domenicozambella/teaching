\documentclass[10pt]{article}
\usepackage[utf8]{inputenc}
\usepackage[a4paper,height=24cm,width=13cm]{geometry}
\usepackage[italian]{babel}
\usepackage{amssymb}
\usepackage{dsfont}
\usepackage{calc}
\usepackage{graphicx}
\usepackage{pstricks}
\usepackage{pst-node}
\usepackage{fourier}
\usepackage{euscript}
\usepackage{amsmath,amssymb, amsthm}

\def\lh{\textrm{lh}}
\def\phi{\varphi}
\def\P{\EuScript P}
\def\M{\EuScript M}
\def\D{\EuScript D}
\def\U{\EuScript U}
\def\V{\EuScript V}
\def\sm{\smallsetminus}
\def\niff{\nleftrightarrow}
\def\ZZ{\mathds Z}
\def\NN{\mathds N}
\def\PP{\mathds P}
\def\QQ{\mathds Q}
\def\RR{\mathds R}
\def\<{\langle}
\def\>{\rangle}
\def\E{\exists}
\def\A{\forall}
\def\0{\varnothing}
\def\imp{\rightarrow}
\def\iff{\leftrightarrow}
\def\IMP{\Rightarrow}
\def\IFF{\Leftrightarrow}
\def\range{\textrm{im}}
\def\Mod{\textrm{Mod}}
\def\Aut{\textrm{Aut}}
\def\Th{\textrm{Th}}
\def\acl{\textrm{acl}}
\def\eq{{\rm eq}}
\def\tp{\textrm{tp}}
\def\equivL{\stackrel{\smash{\scalebox{.5}{\rm L}}}{\equiv}}
\def\swedge{\mathbin{\raisebox{.2ex}{\tiny$\mathbin\wedge$}}}
\def\svee{\mathbin{\raisebox{.2ex}{\tiny$\mathbin\vee$}}}

\newcommand{\labella}[1]{{\sf\footnotesize #1}\hfill}
\renewenvironment{itemize}
  {\begin{list}{$\triangleright$}{%
   \setlength{\parskip}{0mm}
   \setlength{\topsep}{0mm}
   \setlength{\rightmargin}{0mm}
   \setlength{\listparindent}{0mm}
   \setlength{\itemindent}{0mm}
   \setlength{\labelwidth}{3ex}
   \setlength{\itemsep}{0mm}
   \setlength{\parsep}{0mm}
   \setlength{\partopsep}{0mm}
   \setlength{\labelsep}{1ex}
   \setlength{\leftmargin}{\labelwidth+\labelsep}
   \let\makelabel\labella}}{%
   \end{list}}
%\def\ssf#1{\textsf{\small #1}}
\newcounter{ex}
\newenvironment{exercise}{\bigskip\addtocounter{ex}{1}\textbf{Exercise \theex.\quad}}{}
%\pagestyle{empty}
\parindent0ex
\parskip2ex
\raggedbottom
\renewcommand{\baselinestretch}{1.5}


\usepackage{fancyhdr}
\pagestyle{fancy}
\lhead{Example sheet 3}
\rhead{Part III}
\chead{Model Theory 2018/19}
\cfoot{}
\begin{document}


\begin{exercise}
Let $p(x)\subseteq L(B)$ and $p_n(x)\subseteq L(A)$, for $n<\omega$, be consistent types such that\smallskip

\noindent\kern15ex$\displaystyle p(x)\ \ \imp\ \ \bigvee_{n<\omega}p_n(x)$

Prove that there is an $n<\omega$ and a formula $\phi(x)\in L(A)$ consistent with $p(x)$ such that \smallskip

\noindent\kern15ex$p(x)\wedge\phi(x)\ \ \imp\ \ p_n(x)$.
\end{exercise}

\begin{exercise}
Let $M$ be a second countable topological space (i.e.\@ the topology has a countable base).
We say that $A\subseteq M$ is meager if it is the countable union of nowhere dense sets.
Use Lemma~12.1 to prove the Kuratowski-Ulam Theorem, i.e.\@ that for $A\subseteq M^2$ the following are equivalent  (w.r.t.\@ the product topology)
\begin{itemize}
\item[1.] $A$ is meager in $M^2$;
\item[2.] $\Big\{x\in M : A\cap \{x\}{\times}M \ \textrm{is not meager}\Big\}$ is meager in $M$.
\end{itemize}
Hint: Use the base of the topology as predicates of a first-order language.
\end{exercise}

\begin{exercise}
Let $p(x)\subseteq L(A)$ and let $\phi(x\,;y)\in L(A)$ be a formula that defines, when restricted to $p(\U)$, an equivalence relation with finitely many classes. Prove that there is a finite equivalence relation definable over $A$ that coincides with $\phi(x\,;y)$ on $p(\U)$.
\end{exercise}


\begin{exercise}
Let $p(x)\subseteq L(A)$ and let $\phi(x\,;y)\in L(A)$ be a formula that defines, when restricted to $p(\U)$, an equivalence relation with finitely many classes. Prove that there is a finite equivalence relation definable over $A$ that coincides with $\phi(x\,;y)$ on $p(\U)$.
\end{exercise}



\begin{exercise}
Let $T$ be a consistent theory. Suppose that all completions of $T$ are of the form 
$T\cup S$ for some set $S$ of quantifier-free sentences (as, for example, $T_{\rm acf}$). Prove that, if all completion of $T$ have elimination of quantifiers, so does $T$.

Hint: prove that for every formula $\phi(x)$ there are some quantifier-free sentences $\sigma_i$ and quantifier-free formulas $\psi_i(x)$ such that

$\displaystyle\qquad\sigma_i\vdash\phi(x)\leftrightarrow\psi_i(x)$, 
$\qquad\displaystyle T\vdash\bigvee^n_{i=1}\sigma_i$,\qquad and  $\qquad\sigma_i\vdash\neg\sigma_j$ for $i\neq j$.

For a counter example consider the empty theory in the language with a single unary predicate.
\end{exercise}



\begin{exercise}
Show that the claim in the exercise above fails when the theories $S$ have arbitrary complexity. 

Hint: let $T$ be the empty theory in the language with a single unary predicate.
\end{exercise}

\end{document} 
