\documentclass[10pt]{article}
\usepackage[utf8]{inputenc}
\usepackage[a4paper,height=24cm,width=13cm]{geometry}
\usepackage[italian]{babel}
\usepackage{amssymb}
\usepackage{dsfont}
\usepackage{calc}
\usepackage{graphicx}
\usepackage{pstricks}
\usepackage{pst-node}
\usepackage{fourier}
\usepackage{euscript}
\usepackage{amsmath,amssymb, amsthm}

\def\lh{\textrm{lh}}
\def\phi{\varphi}
\def\K{\EuScript K}
\def\M{\EuScript M}
\def\D{\EuScript D}
\def\U{\EuScript U}
\def\V{\EuScript V}
\def\sm{\smallsetminus}
\def\niff{\nleftrightarrow}
\def\ZZ{\mathds Z}
\def\NN{\mathds N}
\def\PP{\mathds P}
\def\QQ{\mathds Q}
\def\RR{\mathds R}
\def\<{\langle}
\def\>{\rangle}
\def\E{\exists}
\def\A{\forall}
\def\0{\varnothing}
\def\imp{\rightarrow}
\def\iff{\leftrightarrow}
\def\IMP{\Rightarrow}
\def\IFF{\Leftrightarrow}
\def\range{\textrm{im}}
\def\Mod{\textrm{Mod}}
\def\Aut{\textrm{Aut}}
\def\Th{\textrm{Th}}
\def\acl{\textrm{acl}}
\def\eq{{\rm eq}}
\def\tp{\textrm{tp}}
\def\equivL{\stackrel{\smash{\scalebox{.5}{\rm L}}}{\equiv}}
\def\swedge{\mathbin{\raisebox{.2ex}{\tiny$\mathbin\wedge$}}}
\def\svee{\mathbin{\raisebox{.2ex}{\tiny$\mathbin\vee$}}}
\def\ssf#1{{\sf\footnotesize #1}}

\newcommand{\labella}[1]{\ssf{#1}\hfill}
\renewenvironment{itemize}
  {\begin{list}{$\triangleright$}{%
   \setlength{\parskip}{0mm}
   \setlength{\topsep}{0mm}
   \setlength{\rightmargin}{0mm}
   \setlength{\listparindent}{0mm}
   \setlength{\itemindent}{0mm}
   \setlength{\labelwidth}{3ex}
   \setlength{\itemsep}{0mm}
   \setlength{\parsep}{0mm}
   \setlength{\partopsep}{0mm}
   \setlength{\labelsep}{1ex}
   \setlength{\leftmargin}{\labelwidth+\labelsep}
   \let\makelabel\labella}}{%
   \end{list}\vskip-0.5\baselineskip}
\newcounter{ex}
\newenvironment{exercise}{\bigskip\refstepcounter{ex}\textbf{Exercise \theex.\quad}}{}
\parindent0ex
\parskip2ex
\raggedbottom
\renewcommand{\baselinestretch}{1.5}


\usepackage{fancyhdr}
\pagestyle{fancy}
\lhead{Example sheet 2}
\chead{Model Theory 2018/19}
\rhead{Part III}
\cfoot{}
\begin{document}

\begin{exercise}\label{ex1}
Assume $L$ is countable and let $M\preceq N$ have arbitrary (large) cardinality.
Let $A\subseteq N$ be countable.
Adapt the construction used for the downward L\"owenheim-Skolem Theorem to prove that there is a countable model $K$ such that $A\subseteq K\preceq N$ and $K\cap M\preceq N$ (in particular, $K\cap M$ is a model).
\end{exercise}

\begin{exercise}\label{ex2}
Give an alternative proof of Exercise~\ref{ex1} using the elementary chain lemma and the downward L\"owenheim-Skolem Theorem (instead of its proof).
Hint: construct two chains of countable models such that $K_i\cap M\subseteq M_i\preceq N$ and $A\cup M_i\subseteq K_{i+1}\preceq N$.
\end{exercise} 

\begin{exercise}
Let $N$ be a countable random graph. Prove that if we erase from $N$ finitely many vertices and all their direct neighbours we obtain a graph isomorphic to $N$.
\end{exercise}

\begin{exercise}
Let $N$ be free union of two countable random graphs $N_1$ and $N_2$. That is, $N=N_1\sqcup N_2$ and $r^N= r^{N_1}\sqcup r^{N_2}$. By $\sqcup$ we denote the disjoint union. Prove that $N$ is not a random graph. 
Write a first-order sentence $\psi(x,y)\in L$ true if $x$ and $y$ belong both to $N_1$ or both to $N_2$.
\end{exercise}


\begin{exercise}\label{ex5}
Let $T_{\rm grph}$ be the theory  of graphs that is, the theory that says that $r(x,y)$ is a irreflexive, symmetric relation. Assume $N\models T_{\rm grph}$ is such that for every $b\in M\models T_{\rm grph}$, every finite partial isomorphism $k:M\to N$ has an extension to a  partial isomorphism defined in $b$. Prove that $N$ is a random graph.
\end{exercise}

\begin{exercise}\label{ex7}
The language contains only two binary relations: $<$ and $e$. Let $T_0$ be the theory that says that $<$ is a strict linear order and that $e$ is an equivalence relation. Axiomatize a theory $T_1\supseteq T_0$ such that every $N\models T_1$ has the following property: for every $b\in M\models T_0$, every finite partial isomorphism $k:M\to N$ has an extension to a  partial isomorphism defined in $b$.

Proof the claim for yourself, hand in only the axiomatization.
\end{exercise}


\begin{exercise}\label{ex8}
Prove that the theory $T_1$ in Exercise~\ref{ex7} is $\omega$-categorical.
\end{exercise}


\begin{exercise}  Let $N$ be a saturated model and let $\phi(x\,;y)\in L(N)$. Prove that the following are equivalent
\begin{itemize}
\item[1.] there is a sequence $\<a_i\,:\,i\in\omega\>$ such that $\phi(N\,;a_i)\subset\phi(N\,;a_{i+1})$ for every $i<\omega$;
\item[2.] there is a sequence $\<a_i\,:\,i\in\omega\>$ such that $\phi(N\,;a_{i+1})\subset\phi(N\,;a_i)$ for every $i<\omega$.
\end{itemize}
\end{exercise}


\begin{exercise}  Let $N$ be a saturated model and let $\phi(x)\in L(N)$. Prove that the following are equivalent
\begin{itemize}
\item[1.] $\phi(N)$ is infinite;
\item[2.] $\phi(N)$ has the cardinality of $N$.
\end{itemize}
\end{exercise}


\begin{exercise}  Let $N$ be a saturated model and let $p(x)\subseteq L(A)$, for some $A\subseteq N$ of cardinality $<|N|$, be a type closed under conjunction (for simplicity). Prove that the following are equivalent
\begin{itemize}
\item[1.] $p(x)$ has finitely many solution;
\item[2.] $p(x)$ contains a formula with finitely many solutions.
\end{itemize}
\end{exercise}


\begin{exercise}  Let $N$ be a saturated model of $T_{\rm lo}$. Prove that the following are equivalent
\begin{itemize}
\item[1.] there is a sequence $\<a_i\,:\,i\in\omega\>$ such that $a_i<a_{i+1}$ for every $i<\omega$;
\item[2.] there is a sequence $\<a_i\,:\,i\in\omega\>$ such that $a_i>a_{i+1}$ for every $i<\omega$.
\end{itemize}
\end{exercise}


\end{document} 
