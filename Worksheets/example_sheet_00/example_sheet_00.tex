\documentclass[10pt]{article}
\usepackage[utf8]{inputenc}
\usepackage[a4paper,height=24cm,width=13cm]{geometry}
\usepackage[italian]{babel}
\usepackage{amssymb}
\usepackage{dsfont}
\usepackage{calc}
\usepackage{graphicx}
\usepackage{pstricks}
\usepackage{pst-node}
\usepackage{fourier}
\usepackage{euscript}
\usepackage{amsmath,amssymb, amsthm}

\def\lh{\textrm{lh}}
\def\phi{\varphi}
\def\P{\EuScript P}
\def\M{\EuScript M}
\def\D{\EuScript D}
\def\U{\EuScript U}
\def\V{\EuScript V}
\def\sm{\smallsetminus}
\def\niff{\nleftrightarrow}
\def\ZZ{\mathds Z}
\def\NN{\mathds N}
\def\PP{\mathds P}
\def\QQ{\mathds Q}
\def\RR{\mathds R}
\def\<{\langle}
\def\>{\rangle}
\def\E{\exists}
\def\A{\forall}
\def\0{\varnothing}
\def\imp{\rightarrow}
\def\iff{\leftrightarrow}
\def\IMP{\Rightarrow}
\def\IFF{\Leftrightarrow}
\def\range{\textrm{im}}
\def\Mod{\textrm{Mod}}
\def\Aut{\textrm{Aut}}
\def\Th{\textrm{Th}}
\def\acl{\textrm{acl}}
\def\dcl{\textrm{dcl}}
\def\eq{{\rm eq}}
\def\tp{\textrm{tp}}
\def\equivL{\stackrel{\smash{\scalebox{.5}{\rm L}}}{\equiv}}
\def\swedge{\mathbin{\raisebox{.2ex}{\tiny$\mathbin\wedge$}}}
\def\svee{\mathbin{\raisebox{.2ex}{\tiny$\mathbin\vee$}}}

\newcommand{\labella}[1]{{\sf\footnotesize #1}\hfill}
\renewenvironment{itemize}
  {\begin{list}{$\triangleright$}{%
   \setlength{\parskip}{0mm}
   \setlength{\topsep}{0mm}
   \setlength{\rightmargin}{0mm}
   \setlength{\listparindent}{0mm}
   \setlength{\itemindent}{0mm}
   \setlength{\labelwidth}{3ex}
   \setlength{\itemsep}{0mm}
   \setlength{\parsep}{0mm}
   \setlength{\partopsep}{0mm}
   \setlength{\labelsep}{1ex}
   \setlength{\leftmargin}{\labelwidth+\labelsep}
   \let\makelabel\labella}}{%
   \end{list}}
%\def\ssf#1{\textsf{\small #1}}
\newcounter{ex}
\newenvironment{exercise}{\bigskip\addtocounter{ex}{1}\textbf{Exercise \theex.\quad}}{}
%\pagestyle{empty}
\parindent0ex
\parskip2ex
\raggedbottom
\renewcommand{\baselinestretch}{1.5}


\usepackage{fancyhdr}
\pagestyle{fancy}
\lhead{Proposta d'esame}
\rhead{Part III}
\chead{Model Theory 2018/19}
\cfoot{}
\begin{document}



\begin{exercise}
The language contains only a symbol for a binary relation $e$. The theory $T$ says that $e$ is an equivalence relation and that for each $n\in\omega\sm\{0\}$ there are $2$ classes containing $n$ elements. 

Most of the following questions have a short answer.

\medskip
\begin{itemize}
\item[1.] Is every countable model of $T$ with infinitely many infinite classes saturated?
\item[2.] Is every uncountable model of $T$ with infinitely many infinite classes saturated?
\item[3.] Describe a non homogeneous model of $T$.
\item[4.] Sketch a proof that the theory $T$ has quantifier elimination.  
\item[5.] Is $T$ countably categorical? 
\item[6.] Is $T$ categorical is some uncountable cardinal?
\item[7.] Is $T$ is complete?
\item[8.] What is $\acl(\varnothing)$?
\item[9.] Let $\U\models T$ be saturated. If $e(a,\U)$ contains 2 elements, what is $\dcl\{a\}$?
\item[10.] Let $e(b,\U)$ be infinite, what is $\acl\{b\}$?
\item[11.] If $b$ is like above, what is the orbit of $b$ under $\Aut(\U)$?
\item[12.] Is $T$ strongly minimal?
\end{itemize}
\end{exercise}


\end{document} 
