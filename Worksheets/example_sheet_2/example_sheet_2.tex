\documentclass[10pt]{article}
\usepackage[utf8]{inputenc}
\usepackage[a4paper,height=24cm,width=13cm]{geometry}
\usepackage[italian]{babel}
\usepackage{amssymb}
\usepackage{dsfont}
\usepackage{calc}
\usepackage{graphicx}
\usepackage{pstricks}
\usepackage{pst-node}
\usepackage{fourier}
\usepackage{euscript}
\usepackage{amsmath,amssymb, amsthm}

\def\lh{\textrm{lh}}
\def\phi{\varphi}
\def\P{\EuScript P}
\def\M{\EuScript M}
\def\D{\EuScript D}
\def\U{\EuScript U}
\def\V{\EuScript V}
\def\sm{\smallsetminus}
\def\niff{\nleftrightarrow}
\def\ZZ{\mathds Z}
\def\NN{\mathds N}
\def\PP{\mathds P}
\def\QQ{\mathds Q}
\def\RR{\mathds R}
\def\<{\langle}
\def\>{\rangle}
\def\E{\exists}
\def\A{\forall}
\def\0{\varnothing}
\def\imp{\rightarrow}
\def\iff{\leftrightarrow}
\def\IMP{\Rightarrow}
\def\IFF{\Leftrightarrow}
\def\range{\textrm{im}}
\def\Mod{\textrm{Mod}}
\def\Aut{\textrm{Aut}}
\def\Th{\textrm{Th}}
\def\acl{\textrm{acl}}
\def\eq{{\rm eq}}
\def\tp{\textrm{tp}}
\def\equivL{\stackrel{\smash{\scalebox{.5}{\rm L}}}{\equiv}}
\def\swedge{\mathbin{\raisebox{.2ex}{\tiny$\mathbin\wedge$}}}
\def\svee{\mathbin{\raisebox{.2ex}{\tiny$\mathbin\vee$}}}

\newcommand{\labella}[1]{{\sf\footnotesize #1}\hfill}
\renewenvironment{itemize}
  {\begin{list}{$\triangleright$}{%
   \setlength{\parskip}{0mm}
   \setlength{\topsep}{0mm}
   \setlength{\rightmargin}{0mm}
   \setlength{\listparindent}{0mm}
   \setlength{\itemindent}{0mm}
   \setlength{\labelwidth}{3ex}
   \setlength{\itemsep}{0mm}
   \setlength{\parsep}{0mm}
   \setlength{\partopsep}{0mm}
   \setlength{\labelsep}{1ex}
   \setlength{\leftmargin}{\labelwidth+\labelsep}
   \let\makelabel\labella}}{%
   \end{list}}
%\def\ssf#1{\textsf{\small #1}}
\newcounter{ex}
\newenvironment{exercise}{\bigskip\addtocounter{ex}{1}\textbf{Exercise \theex.\quad}}{}
%\pagestyle{empty}
\parindent0ex
\parskip2ex
\raggedbottom
\renewcommand{\baselinestretch}{1.5}


\usepackage{fancyhdr}
\pagestyle{fancy}
\lhead{Example sheet 2}
\rhead{Part III}
\chead{Model Theory 2018/19}
\cfoot{}
\begin{document}


\begin{exercise}
Let $L$ be a language that extends that of strict linear orders with the constants $\{c_i: i\in\omega\}$. Let $T$ be the theory that extends $T_{\rm dlo}$ with the axioms $c_i<c_{i+1}$ for every  $i\in\omega$.  From what is known of $T_{\rm dlo}$ we deduce that $T$ has elimination of quantifiers and is complete. Exhibit a countable models that are
\begin{itemize}
\item[1.] saturated;
\item[2.] homogeneous but not saturated;
\item[3.] not homogeneous.
\end{itemize}
\end{exercise}

\begin{exercise}
Let $L$ be the language of strict orders Prove that the structure $\QQ\times\ZZ$ ordered with the lexicographic order 

\hfil$(a_1,a_2)<(b_1,b_2)\quad\IFF\quad a_1<b_1$\ \ or\ \ ($a_1=b_1$\ \ e\ \ $a_2<b_2$)

is a saturated model.
\end{exercise}


\begin{exercise}
Let $M$ and $N$ be elementarily homogeneous structures of the same cardinality $\lambda$. Suppose that $M\models\E x\, p(x)\,\IFF\,N\models\E x\, p(x)$ for every $p(x)\subseteq L$ such that $|x|<\lambda$. Prove that the two structures are isomorphic.
\end{exercise}


\begin{exercise}
Let $\phi( z)\in L(A)$ be a consistent formula. Prove that, if $a\in\acl(A, b)$ for every $b\models\phi(z)$, then $a\in\acl(A)$. Prove the same claim with a type $p(z)\subseteq L(A)$ for $\phi(z)$.
\end{exercise}


\begin{exercise}
Let $\phi(x\,;z)\in L$. Prove that if the set $\big\{\phi(a\,;\U)\ :\ a\in\U^{|x|}\big\}$ is infinite then it has cardinality $\kappa$. Does the claim remains true with a type $p(x\,;z)\subseteq L$ for $\phi(x\,;z)$? Hint: consider $\Th(\NN,<)$.
\end{exercise}

\begin{exercise}
Let $C$ be a finite set. Prove that if $C\cap M\neq\0$ for every model $M$ containing $A$, then $C\cap\acl(A)\neq\0$.
\end{exercise}


\begin{exercise}
Prove that for every $A\subseteq N$ there is an $M$ such that $\acl A\,=\,M\cap N$.
\end{exercise}


\begin{exercise}
Let $\U$ be some large saturated model and let $a\in\U\sm\acl\0$. Prove that $\U$ is isomorphic to some $\V\preceq \U$ such that $a\notin\V$. Hint: let $c$ be an enumeration of $\U$ and let $p(u)=\tp(c)$ prove that $p(u)\cup\big\{u_i\neq a :i<\kappa\big\}$ is realized in $\U$ and that any realization yields the required substructure of $\U$.
\end{exercise}



\begin{exercise}
Let $T$ be a consistent theory. Suppose that all completions of $T$ are of the form 
$T\cup S$ for some set $S$ of quantifier-free sentences. Prove that, if all completion of $T$ have elimination of quantifiers, so does $T$. Show that this fails when the completions of $T$ have arbitrary complexity.
\end{exercise}



\end{document} 
