\documentclass[10pt]{article}
\usepackage[utf8]{inputenc}
\usepackage[a4paper,height=24cm,width=13cm]{geometry}
\usepackage[italian]{babel}
\usepackage{amssymb}
\usepackage{dsfont}
\usepackage{calc}
\usepackage{graphicx}
\usepackage{pstricks}
\usepackage{pst-node}
\usepackage{fourier}
\usepackage{euscript}
\usepackage{amsmath,amssymb, amsthm}

\def\lh{\textrm{lh}}
\def\phi{\varphi}
\def\K{\EuScript K}
\def\M{\EuScript M}
\def\D{\EuScript D}
\def\U{\EuScript U}
\def\V{\EuScript V}
\def\sm{\smallsetminus}
\def\niff{\nleftrightarrow}
\def\ZZ{\mathds Z}
\def\NN{\mathds N}
\def\PP{\mathds P}
\def\QQ{\mathds Q}
\def\RR{\mathds R}
\def\<{\langle}
\def\>{\rangle}
\def\E{\exists}
\def\A{\forall}
\def\0{\varnothing}
\def\imp{\rightarrow}
\def\iff{\leftrightarrow}
\def\IMP{\Rightarrow}
\def\IFF{\Leftrightarrow}
\def\range{\textrm{im}}
\def\Mod{\textrm{Mod}}
\def\Aut{\textrm{Aut}}
\def\Th{\textrm{Th}}
\def\acl{\textrm{acl}}
\def\eq{{\rm eq}}
\def\tp{\textrm{tp}}
\def\equivL{\stackrel{\smash{\scalebox{.5}{\rm L}}}{\equiv}}
\def\swedge{\mathbin{\raisebox{.2ex}{\tiny$\mathbin\wedge$}}}
\def\svee{\mathbin{\raisebox{.2ex}{\tiny$\mathbin\vee$}}}

\newcommand{\labella}[1]{{\sf\footnotesize #1}\hfill}
\renewenvironment{itemize}
  {\begin{list}{$\triangleright$}{%
   \setlength{\parskip}{0mm}
   \setlength{\topsep}{0mm}
   \setlength{\rightmargin}{0mm}
   \setlength{\listparindent}{0mm}
   \setlength{\itemindent}{0mm}
   \setlength{\labelwidth}{3ex}
   \setlength{\itemsep}{0mm}
   \setlength{\parsep}{0mm}
   \setlength{\partopsep}{0mm}
   \setlength{\labelsep}{1ex}
   \setlength{\leftmargin}{\labelwidth+\labelsep}
   \let\makelabel\labella}}{%
   \end{list}\vskip-0.5\baselineskip}
\newcounter{ex}
\newenvironment{exercise}{\bigskip\refstepcounter{ex}\textbf{Exercise \theex.\quad}}{}
\parindent0ex
\parskip2ex
\raggedbottom
\renewcommand{\baselinestretch}{1.5}


\usepackage{fancyhdr}
\pagestyle{fancy}
\lhead{Example sheet 1}
\chead{Model Theory 2018/19}
\rhead{Part III}
\cfoot{}
\begin{document}

\begin{exercise}\label{ex1}
Assume $L$ is countable and let $M\preceq N$ have arbitrary (large) cardinality.
Let $A\subseteq N$ be countable.
Adapt the construction used for the downward L\"owenheim-Skolem Theorem to prove that there is a countable model $K$ such that $A\subseteq K\preceq N$ and $K\cap M\preceq N$ (in particular, $K\cap M$ is a model).
\end{exercise}


\begin{exercise}
Let $\<M_i:i\in\lambda\>$ be an elementary chain of substructures of $N$.
Let $M$ be the union of the chain.
Prove that $M\preceq N$.
\end{exercise}

\begin{exercise}
Give an alternative proof of Exercise~\ref{ex1} using the elementary chain lemma and the downward L\"owenheim-Skolem Theorem (instead of its proof).
Hint: construct two chains of countable models such that $K_i\cap M\subseteq M_i\preceq N$ and $A\cup M_i\subseteq K_{i+1}\preceq N$.
\end{exercise}


\begin{exercise}
Prove that $T_{\rm dlo}$ is not $\lambda$-categorical for any uncountable $\lambda$.
\end{exercise}


\begin{exercise}
Show that there is an $\omega$-categorical theory that is not complete (the language need to be  uncountable). Hint. Let $\nu$ be an uncountable cardinal. The language contains only the ordinals $i<\nu$ as constants. The theory $T$ says that there are infinitely many elements and either $i=0$ for every $i<\nu$, or $i\neq j$ for every $i<j<\nu$. Prove that $T$ is $\omega$-categorical but incomplete.
\end{exercise}


\begin{exercise}
Let $N$ be free union of two random graphs $N_1$ and $N_2$. That is, $N=N_1\sqcup N_2$ and $r^N= r^{N_1}\sqcup r^{N_2}$. By $\sqcup$ we denote the disjoint union. Prove that $N$ is not a random graph. Show that $N_1$ is not definable without parameters (assume $|N_1|=|N_2|=\omega$, otherwise the proof is involved). Write a first order sentence $\psi(x,y)$ true if $x$ and $y$ belong to the same connected component of $N$. Axiomatize the class $\K$ of graphs that are free union of two random graphs.
\end{exercise}


\begin{exercise}
The language contains only the binary relations $<$ and $e$. The theory $T_0$ says that $<$ is a strict linear order and that $e$ is an equivalence relation. Let $\M$ consists of models of $T_0$ and partial isomorphisms. Do rich models exist? Can we axiomatize their theory? If so, does it have elimination of quantifiers? Is it $\lambda$-categorical for some $\lambda$?
\end{exercise}


\begin{exercise}
Prove that for every infinite graph $M$ the following are equivalent
\begin{itemize}
\item[1.] $M$ is either random, empty, or complete;
\item[2.] if $M_1,M_2\subseteq M$ are such that $M_1\sqcup M_2=M$, then $M_1\simeq M$ or $M_2\simeq M$.
\end{itemize}
With $\sqcup$ we denote the disjoint union.
\end{exercise}
\end{document} 
