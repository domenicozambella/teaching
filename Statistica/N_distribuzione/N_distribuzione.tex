\documentclass[11pt,twoside,a4paper]{article}
\usepackage[T1]{fontenc}
\usepackage[utf8]{inputenc}
\usepackage[top=20mm, head=6mm, headsep=6mm, foot=6mm, bottom= 15mm, left=15mm, right=15mm]{geometry}
\usepackage{amsmath}
\usepackage{xcolor}
\usepackage{pythontex}
\parindent0ex
\parskip2ex
\def\Pr{{\rm Pr\,}}
\def\Ex{{\rm E\,}}
\def\Var{{\rm Var\,}}
\newcounter{quesito}
\newenvironment{question}{\addtocounter{quesito}{1}\bigskip\bigskip\par\textbf{Quesito \thequesito.\kern1ex}}{\vspace{\parskip}}
\newenvironment{xquestion}{\addtocounter{quesito}{1}\bigskip\bigskip\par\textbf{Quesito \thequesito.\kern1ex}}{\vspace{\parskip}}
\newenvironment{answer}{\par\textbf{Risposta\quad}}{\vspace{\parskip}}
\pagestyle{empty} 

\begin{document}
\colorbox{blue!10}{\begin{minipage}{\textwidth}
Domande per verificare la comprensione del significato di distribuzione continua (solo caso distribuzione normale). Richiede anche le nozioni di standardizzazione e di media campionaria.\medskip

N.B. Alcune domande potrebbero contenere informazioni irrilevanti.
\end{minipage}}
\bigskip\bigskip


\begin{pycode}
import random
random.seed('daxtxsdsxssme')
ESAME = False
\end{pycode}

\begin{question} %1
\begin{pycode}
from scipy.stats import norm
mu = random.choice([6,7,8,9])
sigma = random.choice([3,5])
n = random.choice([4,8])
a = mu - random.choice([4,6])
b = mu + random.choice([1,2])

def pr(x):
    return "{0:.3f}".format(round(x,3))
\end{pycode}
La variabile aleatoria $X$ ha distribuzione normale con media $\mu=\py{mu}$ e deviazione standard $\sigma=\py{sigma}$ 

1. Calcolare la probabilità dell'evento $X\in [\py{a},\py{b}]$ 

2. Calcolare la probabilità che da campione di rango $n=\py{n**2}$ si ottenga una media in $[\py{a},\py{b}]$. 

Esprimere il risutato numerico tramite (solo) le funzioni elencate in calce.
\begin{answer}

$P(\py{a}\le X\le\py{b})\ =\ P\bigg(\dfrac{\py{a}-\mu}{\sigma}\le Z\le\dfrac{\py{b}-\mu}{\sigma}\bigg)\ =\  P\big(Z\le\py{b-mu}/\py{sigma}\big) -  P\big(Z\le\py{a-mu}/\py{sigma}\big)$ 

$\phantom{P(\py{a}\le X\le\py{b})}\ =\ ${\color{blue}{\tt\ norm.cdf(\py{b-mu}/\py{sigma}) -  norm.cdf(\py{a-mu}/\py{sigma})} \hfill Risposta 1}

$\phantom{P(\py{a}\le X\le\py{b})}\ =\ ${\tt\ norm.cdf(\py{round((b-mu)/sigma,2)}) -  norm.cdf(\py{round((a-mu)/sigma, 2)}) = \py{pr(norm.cdf((b-mu)/sigma) -  norm.cdf((a-mu)/sigma))}}

$\bar X$ v.a.\@ media campionaria

$P(\py{a}\le\bar X\le\py{b})\ =\ P\bigg(\dfrac{\py{a}-\mu}{\sigma/\sqrt{n}}\le Z\le\dfrac{\py{b}-\mu}{\sigma/\sqrt{n}}\bigg)\ =\  P\big(Z\le\py{n*(b-mu)}/\py{sigma}\big) -  P\big(Z\le\py{n*(a-mu)}/\py{sigma}\big)$

$\phantom{P(\py{a}\le\bar X\le\py{b})}\ =\ ${\color{blue}{\tt\ norm.cdf(\py{n*(b-mu)}/\py{sigma}) -  norm.cdf(\py{n*(a-mu)}/\py{sigma}) }\hfill Risposta 2}

$\phantom{P(\py{a}\le\bar X\le\py{b})}\ =\ ${\tt\ norm.cdf(\py{round(n*(b-mu)/sigma,2)}) -  norm.cdf(\py{round(n*(a-mu)/sigma, 2)})  = \py{pr(norm.cdf(n*(b-mu)/sigma) -  norm.cdf(n*(a-mu)/sigma))}}

\end{answer}
\end{question}



\begin{question} %2
\begin{pycode}
mu = random.choice( [6, 7, 8, 9] )
sigma = random.choice( [2, 3, 4] )
n = random.choice( [10, 5, 4, 3] )
d = random.choice( [3, 4] )
a = mu - d 
alpha = random.choice( [0.1, 0.2, 0.3, 0.4, 0.5] )
z = 0

def pr(x):
    return "{0:.3f}".format(round(x,3))
\end{pycode}
La variabile aleatoria $X$ ha distribuzione normale con media $\mu=\py{mu}$ e deviazione standard $\sigma=\py{sigma}$. Qual'è il minimo $\varepsilon$ tale che $\Pr\big(\py{a}\le X\le\py{a}+\varepsilon\big)\ge\py{alpha}$.

Esprimere il risutato numerico tramite (solo) le funzioni elencate in calce. 
\begin{answer}

$\Pr\big(\py{a}\le X\le\py{a}+\varepsilon\big)$ cresce al variare di $\varepsilon$, quindi il minimo è quando $\Pr\big(\py{a}\le X\le\py{a}+\varepsilon\big)=\py{alpha}$\medskip

$\Pr\big(\py{a}\le X\le\py{a}+\varepsilon\big)
\ =\ 
\Pr\bigg(\dfrac{\py{a}-\mu}{\sigma}\le Z\le\dfrac{\py{a}-\mu+\varepsilon}{\sigma}\bigg)
\ =\ 
\Pr\bigg(-\dfrac{\py{d}}{\py{sigma}}\le Z\le\dfrac{\py{-d}+\varepsilon}{\py{sigma}}\bigg)\ =\ \py{alpha}$

$\Pr\bigg(Z\le\dfrac{\py{-d}+\varepsilon}{\py{sigma}}\bigg) = \py{alpha}+\Pr\bigg(Z\le-\dfrac{\py{d}}{\py{sigma}}\bigg)$\medskip

{\tt(\py{-d} + $\varepsilon$) / \py{sigma} = norm.ppf( \py{alpha} + norm.cdf(-\py{d}/\py{sigma}) ) }\medskip

$\varepsilon\ =\ ${\color{blue}\tt  \py{sigma} * norm.ppf( \py{alpha} + norm.cdf(-\py{d}/\py{sigma}) ) + \py{d} }{\tt\ =\ \py{pr(sigma * norm.ppf( alpha + norm.cdf(-d/sigma) ) + d)}}{\color{blue}\hfill Risposta}

\end{answer}
\end{question}


\clearpage
\begin{question} %3
\begin{pycode}
cosa = 'una data cultura'
if ESAME:
   cosa = random.choice( ['suolo', 'acqua marina', 'gas di scarico', 'sangue'] )
   
sigma = random.choice( [5, 7] )
n_campioni = random.choice( [15, 16, 17, 18, 19] )
n = random.choice( [3, 4] )
a = mu - random.choice( [3, 4, 5, 6] ) 
b = a - random.choice( [1, 2] )

def pr(x):
    return "{0:.3f}".format(round(x,3))
\end{pycode}
Abbiamo prelevato \py{n_campioni} campioni di \py{cosa}. Ci interessa selezionare dei campioni che hanno una concentrazione $\le\py{a}$ di una data sostanza. La misura produce risultati che differiscono dal valore corretto per un errore distribuito normalmente con media $0$ e deviazione standard $\py{sigma}$. Consideriamo la seguente procedura: se la media di $\py{n**2}$ misure è $\le\py{b}$ concludiamo che il campione è come desiderato altrimenti lo scartiamo.

Calcolare (nel caso più sfavorevole) la probabilità di scartare erroneamente un campione.

Esprimere il risutato numerico tramite (solo) le funzioni elencate in calce.
\begin{answer}

Il caso più sfavorevole occorre quando la concentrazione vera nel compione è $\mu=\py{a}$ 


$\bar X\sim N(\mu,\sigma^2)$\hfill media di $\py{n**2}$ misure\kern22ex

$\Pr\big(\bar X\ge\py{b}\big)\ =\ \Pr\bigg(\dfrac{\bar X-\mu}{\sigma/\sqrt{\py{n**2}}}\ge\dfrac{\py{b}-\mu}{\sigma/\sqrt{\py{n**2}}}\bigg)\ =\  \Pr\big(Z\ge\py{n * (b-a) } /\py{sigma} \big)$

$\phantom{\Pr\big(\bar X\ge\py{b}\big)}\ =\ ${\color{blue}{\tt\ 1 -  norm.cdf(\py{n * (b-a)}/\py{sigma})}\hfill Risposta}

$\phantom{\Pr\big(\bar X\ge\py{b}\big)}\ =\ ${\tt\ 1 - norm.cdf(\py{round( n * (b-a)/sigma, 2)})\ =\ \py{ round( 1 -  norm.cdf(n * (b-a)/sigma), 3) } }
\end{answer}
\end{question}







\begin{question} %4
\begin{pycode}
if ESAME:
    cosa = random.choice( ['suolo', 'acqua marina', 'gas di scarico', 'sangue'] )
    sigma = random.choice( [5,7] )
    n_campioni = random.choice( [15,16,17,18,19] )
    n = random.choice( [3,4] )
    a = mu - random.choice( [3,4,5,6] )
    b = a - random.choice( [1,2] )

def pr(x):
    return "{0:.3f}".format(round(x,3))
\end{pycode}
Abbiamo prelevato \py{n_campioni} campioni di \py{cosa}. Ci interessa selezionare quei campioni che hanno una concentrazione $<\py{a}$ di una data sostanza. La misura produce risultati che differiscono dal valore corretto per un errore distribuito normalmente con media $0$ e deviazione standard $\py{sigma}$. Consideriamo la seguente procedura: se la media di $\py{n**2}$ misure è $\le\py{b}$ concludiamo che il campione è come desiderato altrimenti lo scartiamo.

Calcolare (nel caso più sfavorevole) la probabilità di accettare erroneamente un campione.

Esprimere il risutato numerico tramite (solo) le funzioni elencate in calce.
\begin{answer}

Il caso più sfavorevole occorre quando la concentrazione vera nel compione è $\mu=\py{a}$ 


$\bar X\sim N(\mu,\sigma^2)$\hfill media di $\py{n**2}$ misure\kern22ex

$\displaystyle\Pr\big(\bar X\le\py{b}\big)\ =\ \Pr\bigg(\frac{\bar X-\mu}{\sigma/\sqrt{\py{n**2}}}\le\frac{\py{b}-\mu}{\sigma/\sqrt{\py{n**2}}}\bigg)\ =\  \Pr\big(Z\le\py{n * (b-a) } /\py{sigma} \big)$

$\phantom{\Pr\big(\bar X\le\py{b}\big)}\ =\ ${\color{blue}{\tt\ norm.cdf(\py{n * (b-a)}/\py{sigma})}\hfill Risposta}

$\phantom{\Pr\big(\bar X\le\py{b}\big)}\ =\ ${\tt\ norm.cdf(\py{round( n * (b-a)/sigma, 2)})\ =\ \py{ round( norm.cdf(n * (b-a)/sigma), 3) } }
\end{answer}
\end{question}



\clearpage
\begin{question} %5
\begin{pycode}
if ESAME:
    cosa = random.choice( ['suolo', 'acqua marina', 'gas di scarico', 'sangue'] )
    sigma = random.choice( [5, 7] )
    n_campioni = random.choice( [15, 16, 17, 18, 19] )
    n = random.choice( [3, 4] )
    a = mu - random.choice( [3, 4, 5, 6] )
    b = a - random.choice( [1, 2] )

def pr(x):
    return "{0:.3f}".format(round(x,3))
\end{pycode}
Abbiamo prelevato \py{n_campioni} campioni di \py{cosa}. Ci interessa selezionare quei campioni che hanno una concentrazione $<\py{a}$ di una data sostanza. La misura produce risultati che differiscono dal valore corretto per un errore distribuito normalmente con media $0$ e deviazione standard $\py{sigma}$. Consideriamo la seguente procedura: se la media di $\py{n**2}$ misure è $\le\py{b}$ concludiamo che il campione è come desiderato altrimenti lo scartiamo.

Calcolare (nel caso più sfavorevole) la probabilità di accettare almeno un campione errato.

Esprimere il risutato numerico tramite (solo) le funzioni elencate in calce.
\begin{answer}

Il caso più sfavorevole occorre quando tutti \py{n_campioni} campioni hanno concentrazione $\mu=\py{a}$ 


$\bar X\sim N(\mu,\sigma^2)$\hfill media di $\py{n**2}$ misure\kern22ex

$\Pr\big(\bar X\le\py{b}\big)\ =\ \Pr\bigg(\dfrac{\bar X-\mu}{\sigma/\sqrt{\py{n**2}}}\le\dfrac{\py{b}-\mu}{\sigma/\sqrt{\py{n**2}}}\bigg)\ =\  \Pr\big(Z\le\py{n * (b-a) } /\py{sigma} \big)$\medskip

$\phantom{\Pr\big(\bar X\le\py{b}\big)}\ =\ ${\tt\ norm.cdf(\py{n * (b-a)}/\py{sigma})}\hfill\parbox[t]{43ex}{ probabilità (per un singolo campione) di essere accettato erroneamente}\kern22ex\medskip

{\tt\noindent\phantom{1 - }( 1 - norm.cdf(\py{n * (b-a)}/\py{sigma}) )**\py{n_campioni}}\hfill\parbox[t]{43ex}{probabilità nessuno dei \py{n_campioni} campioni venga accettato erroneamente}\kern22ex\medskip


{\color{blue}\tt 1 - ( 1 - norm.cdf(\py{n * (b-a)}/\py{sigma}) )**\py{n_campioni} }\hfill {\color{blue}Risposta}\medskip

{\tt 1 - ( 1 - norm.cdf(\py{n * (b-a)/sigma}) )**\py{n_campioni}\ =\ \py{pr( 1 - ( 1 - norm.cdf(n * (b-a)/sigma) )**n_campioni) } } 
\end{answer}
\end{question}


\begin{question} %6
\begin{pycode}
import random
import math
from scipy.stats import norm
n = random.choice( [3, 4, 5] )
sigma = random.choice( [53, 73, 97] )
err = random.choice( [19, 22, 17] )
xbar = random.choice( [83, 98, 102] )
def pr(x):
    return "{0:.3f}".format(round(x,3))

\end{pycode}
Da una popolazione con distribuzione normale con media $\mu$ ignota e deviazione standard $\py{sigma}$ estraiamo un campione di $\py{n**2}$ individui. Se $\bar x= \py{xbar}$ è la media ottenuta, qual è la probabilità che $\bar x>\mu+ \py{err}$~? 
    
Esprimere il risutato numerico tramite (solo) le funzioni elencate in calce.

\begin{answer}
  
  $n = \py{n**2}$\hfill dimensione del campione\kern22ex
  
  $\varepsilon = \py{err}$ 
  
  $\sigma = \py{sigma}$\hfill deviazione standard della popolazione\kern22ex
  
  $\sigma/\sqrt{n}$\hfill errore standard (deviazione standard della media)\kern22ex
  
  $P\bigg(Z>\dfrac {\varepsilon}{\sigma/\sqrt{n}}\bigg)\ =\ P\big(Z>\py{n * err}/\py{sigma}\big)\ =\ ${\color{blue}\tt\ 1 - norm.cdf(\py{n * err}/\py{sigma})}{\tt\ = \py{pr(1 - norm.cdf( n * err/sigma ) ) } }{\color{blue}\hfill Risposta}

\end{answer}
\end{question}


\vfill
\hrulefill

Si assuma noto il valore delle seguenti funzioni della libreria {\tt scipy.stats\/}

{\tt norm.cdf(z)} = $\Pr\big(Z<{\tt z}\big)$ per $Z\sim N(0,1)$ 

{\tt norm.ppf($\alpha$)} = $z_\alpha$ dove $z_\alpha$ è tale che $\Pr\big(Z<z_\alpha\big)=\alpha$ per $Z\sim N(0,1)$ 


\end{document}

