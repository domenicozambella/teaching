\documentclass[11pt,twoside,a4paper]{article}
\usepackage[T1]{fontenc}
\usepackage[utf8]{inputenc}
\usepackage[top=20mm, head=6mm, headsep=6mm, foot=6mm, bottom= 15mm, left=25mm, right=25mm]{geometry}
\usepackage{amsmath}
\usepackage{calc} 
\usepackage{pythontex}
\newcommand{\mylabel}[1]{#1\hfill}
\renewenvironment{itemize}
  {\begin{list}{$\triangleright$}{%
   \setlength{\parskip}{0mm}
   \setlength{\topsep}{.4\baselineskip}
   \setlength{\rightmargin}{0mm}
   \setlength{\listparindent}{0mm}
   \setlength{\itemindent}{0mm}
   \setlength{\labelwidth}{2ex}
   \setlength{\itemsep}{.4\baselineskip}
   \setlength{\parsep}{0mm}
   \setlength{\partopsep}{0mm}
   \setlength{\labelsep}{1ex}
   \setlength{\leftmargin}{\labelwidth+\labelsep}
   \let\makelabel\mylabel}}{%
   \end{list}\vspace*{-1.3mm}}
\parindent0ex
\parskip2ex
\def\Pr{{\rm Pr\,}}
\newcounter{quesito}
\newenvironment{question}{\bigskip\addtocounter{quesito}{1}\par\textbf{Quesito \thequesito.\kern1ex}}{\vspace{\parskip}}
\newenvironment{xquestion}{\bigskip\addtocounter{quesito}{1}\par\textbf{Quesito \thequesito.\kern1ex}}{\vspace{\parskip}}
\newenvironment{answer}{\par\textbf{Risposta\quad}}{\vspace{\parskip}}

\pagestyle{empty} 

\begin{document}
Domande per verificare la comprensione del significato di probabilità condizionata; dei termini che descrivono l'attendibilità dei test diagnostici; della regola di Bayes.

\bigskip\bigskip

\begin{pycode}
import random
random.seed(25874445)
\end{pycode}


\begin{question}
\def\Pr{{\rm Pr\,}}
\begin{pycode}
f_a = random.choice([40,50,60])
f = random.choice([10,15,20,25])
a = random.choice([2,3,4,5])
a_f = round ( f_a * a / f, 1)
\end{pycode}
Tra le persone di cui A è causa del decesso il $\py{f_a}\%$ è fumatore. La percentuale dei fumatori in tutta la popolazione è del $\py{f}\%$ e quella dei decessi dovuti ad A è del $\py{a}\%$. Calcolare la probabilità che un fumatore ha di morire per A.
\begin{answer}

$F$\hfill insieme dei fumatori\kern22ex

$A$\hfill insieme persone decedute per $A$\kern22ex

$\Pr(A)=\py{a}\%$\hfill prevalenza di $A$ nella popolazione\kern22ex

$\Pr(F)=\py{f}\%$\hfill frazione di fumatori nella popolazione\kern22ex

$\Pr(F|A)=\py{f_a}\%$\hfill prevalenza di $A$ tra i fumatori\kern22ex

$\Pr(A|F)=\dfrac{\Pr(F|A)\cdot \Pr(A)}{\Pr(F)}$ {\color{blue}$=\py{a_f}\%$\hfill Risposta}
\end{answer}
\end{question}


\begin{question}
\def\Pr{{\rm Pr\,}}
\begin{pycode}
f_a = random.choice([40,50,60])
f = random.choice([10,15,20,25])
a = random.choice([5, 10])
a_n = round ( ((100-f_a) * a) / ((100-f) * 100), 2)
\end{pycode}
Tra le persone di cui A è causa del decesso il $\py{f_a}\%$ è fumatore.  La percentuale dei fumatori in tutta la popolazione è del $\py{f}\%$ e quella dei decessi dovuti ad A è del $\py{a}\%$. Calcolare la probabilità che un \textit{non\/} fumatore ha di morire per A.
\begin{answer}

$F$\hfill insieme dei fumatori\kern22ex

$A$\hfill insieme persone decedute per $A$\kern22ex

$\Pr(A)=\py{a}\%$\hfill prevalenza di $A$ nella popolazione\kern22ex

$\Pr(\neg F)=100\% - \Pr(F) =\py{100-f}\%$\hfill frazione di fumatori nella popolazione\kern22ex

$\Pr(\neg F|A)=100\%-\Pr(F|A)  =\py{100-f_a}\%$\hfill prevalenza di $A$ tra i fumatori\kern22ex

$\Pr(A|\neg F)=\dfrac{\Pr(\neg F|A)\cdot \Pr(A)}{\Pr(\neg F)}$  {\color{blue}$=\py{a_n}\%$\hfill Risposta}

\end{answer}
\end{question}

\clearpage
\begin{question}
\def\Pr{{\rm Pr\,}}
\begin{pycode}
sens = random.choice([96, 97])
spec = random.choice([98,99])
prev = random.choice([5,6])
pos =  sens * prev + (100-spec) * (100-prev)
pos = pos / 100
ppv =  prev * sens  / pos
prev = round(prev, 1)
ppv = round(ppv, 1)
pos = round(pos, 1)
\end{pycode}
A common blood test indicates the presence of a disease $\py{sens}\%$ of the time when the disease is actually present in an individual and $\py{100-spec}\%$ of the time when the disease is not present. The prevalence of the disease is $\py{prev}\%$.
\begin{itemize}
\item[1.] What is the sensitivity of the test?
\item[2.] What is the specificity of the test?
\item[3.] What is the positive predictive value of the test?
\end{itemize}
\begin{answer}


$A$\hfill insieme delle persone affette\kern22ex

$T_+$\hfill insieme delle persone positive al test\kern22ex

$T_-$\hfill insieme delle persone negative al test\kern22ex

$\Pr(A)=\py{prev}\%$\hfill prevalenza\kern22ex

$\Pr(\neg A)=1-\Pr(A)=\py{100-prev}\%$

$\Pr(T_+|~ A)$ {\color{blue}$=\py{sens}\%$}\hfill sensitività\kern22ex \llap{\color{blue} Risposta 1}

$\Pr(T_+|\neg A)=\py{100-spec}\%$\hfill probabilità falsi positivi\kern22ex

$\Pr(T_-|\neg A)=1-P(T_+|\neg A)$ {\color{blue}$=\py{spec}\%$}\hfill specificità\kern22ex \llap{\color{blue} Risposta 2}

$\Pr(T_+)=\Pr(T_+|~A)\ \Pr(A) + \Pr(T_+|\neg A)\ \Pr(\neg A)=\py{pos}\%$



$\displaystyle\Pr(A ~|~T_+)\ =\ \frac{\Pr(T_+|~A)\Pr(A)}{\Pr(T_+)}$ {\color{blue}$=\py{ppv}\%$}\hfill valore predittivo positivo\kern22ex \llap{\color{blue} Risposta 3} 



\end{answer}
\end{question}


\begin{question}
\def\Pr{{\rm Pr\,}}
\begin{pycode}
sens = random.choice([96, 97])
spec = random.choice([98,99])
prev = random.choice([5,6])
pos =  sens * prev + (100-spec) * (100-prev)
pos = pos / 100
ppv =  prev * sens  / pos
prev = round(prev, 1)
ppv = round(ppv, 1)
pos = round(pos, 1)
\end{pycode}
A common blood test indicates the presence of a disease $\py{sens}\%$ of the time when the disease is actually present in an individual and $\py{100-spec}\%$ of the time when the disease is not present. The prevalence of the disease is $\py{prev}\%$.
\begin{itemize}
\item[1.] What is the probability that a person that is chosen at random from the general population is positive to the test?
\item[2.] What is the positive predictive value of the test? 
\end{itemize}
\begin{answer}

$A$\hfill insieme delle persone affette\kern22ex

$T_+$\hfill insieme delle persone positive al test\kern22ex

$\Pr(A)=\py{prev}\%$\hfill prevalenza\kern22ex

$\Pr(\neg A)=1-\Pr(A)=\py{100-prev}\%$

$\Pr(T_+|~ A)=\py{sens}\%$\hfill sensitività\kern22ex

$\Pr(T_+|\neg A)=\py{100-spec}\%$

$\Pr(T_+)=\Pr(T_+|~A)\ \Pr(A) + \Pr(T_+|\neg A)\ \Pr(\neg A)$ {\color{blue} $=\py{pos}\%$\hfill Risposta 1}

$\displaystyle\Pr(A ~|~T_+)\ =\ \frac{\Pr(T_+|~A)\Pr(A)}{\Pr(T_+)}$ {\color{blue} $=\py{ppv}\%$\hfill Risposta 2}

\end{answer}
\end{question}




\begin{question}
\def\Pr{{\rm Pr\,}}
\begin{pycode}
rain_days = random.choice([5,6,7,8])
prev = rain_days * 100 / 365
sens =  random.choice([85, 90,95])
spec = random.choice([85, 90,95])
pos =  sens * prev + (100-spec) * (100-prev)
pos = pos / 100
ppv =  prev * sens  / pos
prev = round(prev, 1)
ppv = round(ppv, 1)
pos = round(pos, 1)
\end{pycode}
Marie is getting married tomorrow at an outdoor ceremony in the desert. In recent years it has rained only $\py{rain_days}$ days each year. But the weatherman has predicted rain for tomorrow. When it actually rains, the weatherman correctly forecasts rain $\py{sens}\%$ of the time. When it doesn’t rain, he incorrectly forecasts rain $\py{100-spec}\%$ of the times. What is the probability that it will rain on the day of Marie’s wedding?

\begin{answer}

$R$\hfill event: it rains on Marie’s wedding\kern22ex

$T_+$\hfill event: the weatherman predicts rain\kern22ex

$\Pr(R) =\py{rain_days}/365 = \py{prev}\%$\hfill it rains $\py{rain_days}$ days out of $365$\kern22ex

$\Pr(T_+|R) = \py{sens}\%$\hfill when it rains, rain is predicted\kern22ex

$\Pr(T_+|\neg R) = \py{100-spec}\%$\hfill when it does not rain,  rain is predicted\kern22ex

$\Pr(T_+) = \Pr(T_+|R)\cdot \Pr(R)+ \Pr(T_+|\neg R)\cdot \Pr(\neg R) = \py{pos}\%$ 

$\displaystyle \Pr(R\,| T_+)\ =\ \frac{\Pr(R)\cdot \Pr(T_+|R)}{\Pr(T_+)\vphantom{\Big[}}$  {\color{blue} $=\py{ppv}\%$\hfill Risposta}

\end{answer}
\end{question}

\begin{question}
\def\Pr{{\rm Pr\,}}
\begin{pycode}
from numpy import arange
from scipy.stats import binom

p = random.choice([0.6, 0.7])

n_monete = random.choice([30, 35, 40])
n_sample = n_monete - 5

pH0 = random.choice([0.6, 0.8])
n_eq = n_monete * pH0
n_eq = int( n_eq )
n_di = n_monete - n_eq
k = n_sample * random.choice([0.6, 0.7])
k = int( k )
prev = n_di / n_monete


sens = sum( binom.pmf(arange(k,n_sample+1), n_sample, p) )
spec = sum( binom.pmf(arange(0,k), n_sample, 0.5) )

pos =  sens * prev + (1-spec) * (1-prev)
ppv =  prev * sens  / pos
pfn =  (1-spec) * (1-prev) 

sens = round(sens, 3)
spec = round(spec, 3)
prev = round(prev, 3)
ppv = round(ppv, 3)
pos = round(pos, 3)
pfn = round(pfn, 3)
\end{pycode}
Abbiamo $\py{n_eq+n_di}$ monete di cui $\py{n_eq}$ sono equilibrate, le altre sono difettose e hanno probabilità $\py{p}$ di dare come risultato {\tt Testa}. Scegliamo a caso una di queste $\py{n_eq+n_di}$ monete. Per decidere se è equilibrata o no, la lanciamo $\py{n_sample}$ volte. Se otteniamo $\ge\py{k}$ volte {\tt Testa\/} diremo che è èquilibrata. Qual è la probabilità di dichiarare equilibrata una moneta che non lo è? Dei seguenti dati si usino quelli pertinenti


$\Pr(X\ge \py{k})\ =\ \py{1-spec}$\hfill se $X\sim B(\py{n_sample},0.5)$\kern50ex

$\phantom{\Pr(X\ge \py{k})}\ =\ \py{sens}$\hfill se $X\sim B(\py{n_sample},\py{p})$\kern50ex

$\phantom{\Pr(X\ge \py{k})}\ =\ \py{round(1-binom.cdf(k-1, n_monete,0.5), 3)}$\hfill se $X\sim B(\py{n_monete},0.5)$\kern50ex

$\phantom{\Pr(X\ge \py{k})}\ =\ \py{round(1-binom.cdf(k-1, n_monete,  p), 3)}$\hfill se $X\sim B(\py{n_monete},\py{p})$\kern50ex


\begin{answer}

$D$\hfill insieme degli esperimenti fatti con monete sbilanciate\kern22ex

$T_{\ge\py{k}}$\hfill insieme degli esperimenti con risultato $\ge\py{k}$\kern22ex

$\Pr(D)=\py{prev}$\hfill prevalenza\kern22ex

$\Pr(\neg D)=1-\Pr(D)=\py{1-prev}$

$\Pr(T_{\ge\py{k}}| \neg D)=\py{1-spec}$

$\Pr(T_{\ge\py{k}}| D)=\py{sens}$

$\Pr(T_{\ge\py{k}})=\Pr(T_{\ge\py{k}}|D)\ \Pr(D) + \Pr(T_{\ge\py{k}}|\neg D)\ \Pr(\neg D)$ {\color{blue} $=\py{pos}$}

$\displaystyle\Pr(\neg D ~|~T_{\ge\py{k}})\ =\ \frac{\Pr(T_{\ge\py{k}}|\neg D)\ \Pr(\neg D)}{\Pr(T_{\ge\py{k}})}$ {\color{blue} $=\py{ppv}$\hfill Risposta}

\end{answer}
\end{question}

\end{document}


\begin{xquestion}
\begin{pycode}
from numpy import arange
from scipy.stats import binom

p = random.choice([0.6])
n_monete = random.choice([100, 200])
pH0 = random.choice([0.8, 0.9])
n_eq = n_monete * pH0
n_eq = int( n_eq )
n_di = n_monete - n_eq
n_sample = random.choice([50, 100])
k = n_sample * random.choice([0.7, 0.8])
k = int( k )
prev = n_di / n_monete


sens = binom.pmf(k, n_sample, p)
spec = 1 - binom.pmf(k, n_sample, 0.5)
pos =  sens * prev + (1-spec) * (1-prev)
ppv =  prev * sens  / pos

sens = round(sens, 3)
spec = round(spec, 3)
prev = round(prev, 3)
ppv = round(ppv, 3)
pos = round(pos, 3)
\end{pycode}
Abbiamo $\py{n_eq+n_di}$ monete di cui $\py{n_eq}$ sono equilibrate, le altre sono difettose e hanno probabilità $\py{p}$ di dare come risultato {\tt Testa}. Scegliamo a caso una di queste $\py{n_eq+n_di}$ monete. Lanciandola $n=\py{n_sample}$ volte otteniamo $\py{k}$ volte {\tt Testa\/}. Qual è la probabilità che si tratti di una moneta non equilibrata? Si usi che

$\Pr(X=\py{k})\ =\ \py{round(1-spec, 4)}$\hfill se $X\sim B(n,0.5)$\kern50ex

$\phantom{\Pr(X=\py{k})}\ =\ \py{sens}$\hfill se $X\sim B(n,\py{p})$\kern50ex

$\Pr(X\ge \py{k})\ =\ \py{round(1-binom.cdf(k-1,n_sample,0.5), 4)}$\hfill se $X\sim B(n,0.5)$\kern50ex

$\phantom{\Pr(X\ge \py{k})}\ =\ \py{round(1-binom.cdf(k-1,n_sample,p), 4)}$\hfill se $X\sim B(n,\py{p})$\kern50ex

\begin{answer} 1

$D$\hfill insieme degli esperimenti fatti con monete sbilanciate\kern22ex

$T_{=\py{k}}$\hfill insieme degli esperimenti con risultato $\py{k}$\kern22ex

$\Pr(D)=\py{prev}$\hfill prevalenza\kern22ex

$\Pr(\neg D)=1-\Pr(D)=\py{1-prev}$

$\Pr(T_{=\py{k}}| \neg D)=\py{1-spec}$

$\Pr(T_{=\py{k}}| D)=\py{sens}$

$\Pr(T_{=\py{k}})=\Pr(T_{=\py{k}}|D)\ \Pr(D) + \Pr(T_{=\py{k}}|\neg D)\ \Pr(\neg D)$ {\color{blue} $=\py{pos}$}

$\displaystyle\Pr(D ~|~T_{=\py{k}})\ =\ \frac{\Pr(T_{=\py{k}}|D)\ \Pr(D)}{\Pr(T_{=\py{k}})}$ {\color{blue} $=\py{ppv}$\hfill Risposta}
\end{answer}

\begin{answer} 2
\begin{pycode}
sens = sum( binom.pmf(arange(k,n_sample+1), n_sample, p) )
spec = sum( binom.pmf(arange(0,k), n_sample, 0.5) )

pos =  sens * prev + (1-spec) * (1-prev)
ppv =  prev * sens  / pos

sens = round(sens, 3)
spec = round(spec, 3)
prev = round(prev, 3)
ppv = round(ppv, 3)
pos = round(pos, 3)
\end{pycode}

$D$\hfill insieme degli esperimenti fatti con monete sbilanciate\kern22ex

$T_{\ge\py{k}}$\hfill insieme degli esperimenti con risultato $\ge\py{k}$\kern22ex

$\Pr(D)=\py{prev}$\hfill prevalenza\kern22ex

$\Pr(\neg D)=1-\Pr(D)=\py{1-prev}$

$\Pr(T_{\ge\py{k}}| \neg D)=\py{1-spec}$

$\Pr(T_{\ge\py{k}}| D)=\py{sens}$

$\Pr(T_{\ge\py{k}})=\Pr(T_{\ge\py{k}}|D)\ \Pr(D) + \Pr(T_{\ge\py{k}}|\neg D)\ \Pr(\neg D)$ {\color{blue} $=\py{pos}$}

$\displaystyle\Pr(D ~|~T_{\ge\py{k}})\ =\ \frac{\Pr(T_{\ge\py{k}}|D)\ \Pr(D)}{\Pr(T_{\ge\py{k}})}$ {\color{blue} $=\py{ppv}$\hfill Risposta}

\end{answer}


\begin{answer} 3
\begin{pycode}
i = 3
sens =     sum( binom.pmf(arange(k-i, k+i+1), n_sample, p) )
spec = 1 - sum( binom.pmf(arange(k-i, k+i+1), n_sample, 0.5) )

pos =  sens * prev + (1-spec) * (1-prev)
ppv =  prev * sens  / pos

sens = round(sens, 3)
spec = round(spec, 3)
prev = round(prev, 3)
ppv = round(ppv, 3)
pos = round(pos, 3)
\end{pycode}

$D$\hfill insieme degli esperimenti fatti con monete sbilanciate\kern22ex

$T_{[\py{k-i},\py{k+i}]}$\hfill insieme degli esperimenti con risultato $\ge\py{k}$\kern22ex

$\Pr(D)=\py{prev}$\hfill prevalenza\kern22ex

$\Pr(\neg D)=1-\Pr(D)=\py{1-prev}$

$\Pr(T_{[\py{k-i},\py{k+i}]}| \neg D)=\py{1-spec}$

$\Pr(T_{[\py{k-i},\py{k+i}]}| D)=\py{sens}$

$\Pr(T_{[\py{k-i},\py{k+i}]})=\Pr(T_{[\py{k-i},\py{k+i}]}|D)\ \Pr(D) + \Pr(T_{[\py{k-i},\py{k+i}]}|\neg D)\ \Pr(\neg D)$ {\color{blue} $=\py{pos}$}

$\displaystyle\Pr(D ~|~T_{[\py{k-i},\py{k+i}]})\ =\ \frac{\Pr(T_{[\py{k-i},\py{k+i}]}|D)\ \Pr(D)}{\Pr(T_{[\py{k-i},\py{k+i}]})}$ {\color{blue} $=\py{ppv}$\hfill Risposta}

\end{answer}




\begin{answer} 4

\begin{pycode}
i = 2
sens = sum( binom.pmf(arange(k-i,n_sample+1), n_sample, p) )
spec = sum( binom.pmf(arange(0,k-i), n_sample, 0.5) )

pos =  sens * prev + (1-spec) * (1-prev)
ppv =  prev * sens  / pos

sens = round(sens, 3)
spec = round(spec, 3)
prev = round(prev, 3)
ppv = round(ppv, 3)
pos = round(pos, 3)
\end{pycode}

$D$\hfill insieme degli esperimenti fatti con monete sbilanciate\kern22ex

$T_{\ge\py{k-i}}$\hfill insieme degli esperimenti con risultato $\ge\py{k}$\kern22ex

$\Pr(D)=\py{prev}$\hfill prevalenza\kern22ex

$\Pr(\neg D)=1-\Pr(D)=\py{1-prev}$

$\Pr(T_{\ge\py{k-i}}| \neg D)=\py{1-spec}$

$\Pr(T_{\ge\py{k-i}}| D)=\py{sens}$

$\Pr(T_{\ge\py{k-i}})=\Pr(T_{\ge\py{k-i}}|D)\ \Pr(D) + \Pr(T_{\ge\py{k-i}}|\neg D)\ \Pr(\neg D)$ {\color{blue} $=\py{pos}$}

$\displaystyle\Pr(D ~|~T_{\ge\py{k-i}})\ =\ \frac{\Pr(T_{\ge\py{k-i}}|D)\ \Pr(D)}{\Pr(T_{\ge\py{k-i}})}$ {\color{blue} $=\py{ppv}$\hfill Risposta}



\end{answer}

\end{xquestion}



