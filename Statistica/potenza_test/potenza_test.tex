\documentclass[11pt,twoside,a4paper]{article}
\usepackage[T1]{fontenc}
\usepackage[utf8]{inputenc}
\usepackage[top=20mm, head=6mm, headsep=6mm, foot=6mm, bottom= 15mm, left=15mm, right=15mm]{geometry}
\usepackage{amsmath}
\usepackage{pythontex}
\parindent0ex
\parskip2ex
\def\Pr{{\rm Pr\,}}
\def\Ex{{\rm E\,}}
\def\Var{{\rm Var\,}}
\newcounter{quesito}
\newenvironment{question}{\addtocounter{quesito}{1}\par\textbf{Quesito \thequesito.\kern1ex}}{\vspace{\parskip}}
\newenvironment{xquestion}{\addtocounter{quesito}{1}\par\textbf{Quesito \thequesito.\kern1ex}}{\vspace{\parskip}}
\newenvironment{answer}{\par\textbf{Risposta\quad}}{\vspace{\parskip}}
\pagestyle{empty} 

\begin{document}
Domande (qualcuna capziosa e artificiale) per verificare la comprensione del significato di potenza di un test. 

N.B. Le domande contengono anche informazioni irrilevanti.

\begin{pycode}
import random
\end{pycode}

\begin{question}
\begin{pycode}

\end{pycode}

\begin{answer}

\end{answer}
\end{question}


\begin{question}
\begin{pycode}
test = random.choice(['T-test a una coda','Z-test ($\sigma$ nota, una coda)'])
test = random.choice(['identica'])
% x1 = -x[0]
% x2 = x[1]
% x3 = x[0]
% p1 = Rational(1,random.choice([2,3,4]))
% p2 = Rational(1,random.choice([3,4]))
% p3 = 1- p1 - p2
% \end{pycode}
Vogliamo fare un \py{test} con un campione di dimensione $\py{n}$.

Fissata una significatività $\alpha=\py{alpha}$ otteniamo un valore di soglia $x_\alpha$. 

Per un dato effect size $\delta$, il test risulta avere potenza $\py{1-beta}$.

Assumendo $H_A$ (almeno di $\delta$), qual'è la probabilità $q$ che in almeno uno di questi test il p-valore risulti $\ge\py{pval}$. (Nel caso non sia possibile determinare il valore esatto di $q$, ma solo un limite superiore/inferiore, si risponda di consequenza.)

\begin{itemize};
\item[1.] $q=$
\item[1.] $q\le$
\item[1.] $q\ge$
\item[1.] Non ci sono sufficienti informazioni per stimare $q$.

\begin{answer}
$q\le\py{beta}$

\end{answer}
\end{question}




\end{document}

