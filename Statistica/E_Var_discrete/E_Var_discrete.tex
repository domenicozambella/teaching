\documentclass[11pt,twoside,a4paper]{article}
\usepackage[T1]{fontenc}
\usepackage[utf8]{inputenc}
\usepackage[top=20mm, head=6mm, headsep=6mm, foot=6mm, bottom= 15mm, left=15mm, right=15mm]{geometry}
\usepackage{amsmath}
\usepackage{calc} 
\usepackage{pythontex}
\newcommand{\mylabel}[1]{#1\hfill}
\renewenvironment{itemize}
  {\begin{list}{$\triangleright$}{%
   \setlength{\parskip}{0mm}
   \setlength{\topsep}{.4\baselineskip}
   \setlength{\rightmargin}{0mm}
   \setlength{\listparindent}{0mm}
   \setlength{\itemindent}{0mm}
   \setlength{\labelwidth}{2ex}
   \setlength{\itemsep}{.4\baselineskip}
   \setlength{\parsep}{0mm}
   \setlength{\partopsep}{0mm}
   \setlength{\labelsep}{1ex}
   \setlength{\leftmargin}{\labelwidth+\labelsep}
   \let\makelabel\mylabel}}{%
   \end{list}\vspace*{-1.3mm}}
\parindent0ex
\parskip2ex
\newcounter{quesito}
\newenvironment{question}{\bigskip\addtocounter{quesito}{1}\bigskip\bigskip\par\textbf{Quesito \thequesito.\kern1ex}}{\vspace{\parskip}}
\newenvironment{xquestion}{\bigskip\addtocounter{quesito}{1}\bigskip\bigskip\par\textbf{Quesito \thequesito.\kern1ex}}{\vspace{\parskip}}
\newenvironment{answer}{\par\textbf{Risposta\quad}}{\vspace{\parskip}}

\pagestyle{empty} 

\begin{document}
Domande  (qualcuna artificiale) per verificare la comprensione della definizione di valore atteso e varianza delle variabili aleatorie discrete.

\bigskip\bigskip


\begin{pycode}
import random
random.seed(258466445)
\end{pycode}


\begin{question}
\def\Pr{{\rm Pr\,}}
\def\Ex{{\rm E\,}}
\def\Var{{\rm Var\,}}
\begin{pycode}
from sympy import *
x = random.sample([1,2,3],2)
x1 = -x[0]
x2 = x[1]
x3 = x[0]
p1 = Rational(1,random.choice([2,3,4]))
p2 = Rational(1,random.choice([3,4]))
p3 = 1- p1 - p2
\end{pycode}
La v.a.\@ discreta $X$ ha distribuzione di probabilità 

\hfil$\displaystyle\Pr(X=\py{x1}) =\py{latex(p1)}$,\hfil  $\displaystyle\Pr(X=\py{x2}) =\py{latex(p2)}$,\hfil $\displaystyle\Pr(X=\py{x3}) =\py{latex(p3)}$. 

\begin{itemize}
\item[1.] Calcolare la distribuzione di probabilità di $X^2$
\item[2.] Calcolare $\Var(X)$. 
\end{itemize}

Esprimere i numeri razionali come frazioni.


\begin{answer}

$\displaystyle\Pr(X^2=\py{x1**2})$ {\color{blue}$\displaystyle\ =\ \py{latex(p1+p3)}$} 
\ \ e \ \ 
$\displaystyle\Pr(X^2=\py{x2**2})$ {\color{blue}$\displaystyle\ =\ \py{latex(p2)}$\hfill Risposta 1} 

$\displaystyle\Ex(X) = \py{x1}\cdot\Pr(X=\py{x1})+\py{x2}\cdot\Pr(X=\py{x2})+\py{x3}\cdot\Pr(X=\py{x3})=\py{latex(x1*p1)} + \py{latex(x2*p2)} + \py{latex(x3*p3)}=\py{latex(x1*p1 + x2*p2 + x3*p3)}$

$\displaystyle\Ex(X^2) = \py{x1**2}\cdot\Pr(X^2=\py{x1**2})\ +\ \py{x2**2}\cdot\Pr(X^2=\py{x2**2})=\py{latex((x1**2)*(p1+p3))} + \py{latex((x2**2)*p2)} = \py{latex((x1**2)*(p1+p3)+ (x2**2)*p2)}$

$\displaystyle\Var(X)=\Ex(X^2)-\Ex(X)^2=\py{latex((x1**2)*(p1+p3)+ (x2**2)*p2)} - \py{latex((x1*p1 + x2*p2 + x3*p3)**2)}$ {\color{blue}$\displaystyle\ =\ \py{latex((x1**2)*(p1+p3)+ (x2**2)*p2 - (x1*p1 + x2*p2 + x3*p3)**2)} $\hfill Risposta 2} 
\end{answer}
\end{question}



\begin{question}
\def\Pr{{\rm Pr\,}}
\def\Ex{{\rm E\,}}
\def\Var{{\rm Var\,}}
\begin{pycode}
from sympy import *
x = random.sample([1,2,3,4,5,6],2)
y = random.choice([2,3])
mu = x[0]
var = x[1]
\end{pycode}
La v.a.\@ discreta $X$ ha valore atteso $\Ex(X)=\py{mu}$ e varianza $\Var(X)=\py{var}$. Qual è il valore atteso di $X(X-\py{y})$?  
\begin{answer}

$\displaystyle\Ex\big(X(X-\py{y})\big)=\Ex(X^2)-\py{y}\cdot\Ex(X)=\Var(X)+\Ex(X)^2-\py{y}\cdot\Ex(X)$ {\color{blue}$\displaystyle\ =\ \py{var+mu**2-y*mu}$\hfill Risposta} 
\end{answer}
\end{question}


\clearpage
\begin{question}
\def\Pr{{\rm Pr\,}}
\def\Ex{{\rm E\,}}
\def\Var{{\rm Var\,}}
\begin{pycode}
from sympy import *
x = random.sample([2,3,4,5],2)
x1 = x[0]
x2 = x[1]

p = random.sample([2,3,4,5],2)
d = random.choice(list(range(1,p[0])))
px = Rational(d,p[0])
d = random.choice(list(range(1,p[1])))
py = Rational(d,p[1])

\end{pycode}
Le v.a.\@ discrete $X$ e $Y$ sono indipendenti. La loro distribuzione di probabilità è data da 


\noindent\rlap{\kern15ex$\displaystyle\Pr(X=\py{x1}) =\py{latex(px)}$}\kern44ex   $\displaystyle\Pr(Y=1) =\py{latex(py)}$ 


\noindent\rlap{\kern15ex$\displaystyle\Pr(X=\py{x2}) =\py{latex(1-px)}$}\kern44ex $\displaystyle\Pr(Y=0) =\py{latex(1-py)}$

\begin{itemize}
\item[1.] Calcolare la distribuzione di probabilità di $X\cdot Y$
\item[2.] Calcolare $\Ex(X\cdot Y)$. 
\end{itemize}

Esprimere i numeri razionali come frazioni.
 
\begin{answer}

$\displaystyle\Pr(X\cdot Y=\py{x1})$ {\color{blue}$\displaystyle\ =\ \py{latex(px * py)}$}
\hfill 
$\displaystyle\Pr(X\cdot Y=\py{x2})$ {\color{blue}$\displaystyle\ =\ \py{latex((1-px) * py)}$}
\hfill 
$\displaystyle\Pr(X\cdot Y=0)$ {\color{blue}$\displaystyle\ =\ \py{latex(1-py)}$\hfill Risposta 1} 

$\displaystyle\Ex(X\cdot Y)\ =\ \py{x1}\cdot\Pr(X\cdot Y=\py{x1}) + \py{x2}\cdot\Pr(X\cdot Y=\py{x2})$ {\color{blue}$\displaystyle\ =\ \py{latex(x1 * px * py + x2 * (1-px) * py)}$\hfill Risposta 2} 

\end{answer}

\end{question}



\end{document}

