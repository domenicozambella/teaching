\documentclass[11pt,twoside,a4paper]{article}
\usepackage[italian]{babel}
\usepackage[T1]{fontenc}
\usepackage[utf8]{inputenc}
\usepackage[top=20mm, head=6mm, headsep=6mm, foot=6mm, bottom= 15mm, left=15mm, right=15mm]{geometry}
\usepackage{amsmath}
\usepackage{calc} 
\usepackage{pythontex}
\usepackage{tikz}
\usetikzlibrary{automata,positioning}
\newcommand{\mylabel}[1]{#1\hfill}
\renewenvironment{itemize}
  {\begin{list}{$\triangleright$}{%
   \setlength{\parskip}{0mm}
   \setlength{\topsep}{.4\baselineskip}
   \setlength{\rightmargin}{0mm}
   \setlength{\listparindent}{0mm}
   \setlength{\itemindent}{0mm}
   \setlength{\labelwidth}{2ex}
   \setlength{\itemsep}{.4\baselineskip}
   \setlength{\parsep}{0mm}
   \setlength{\partopsep}{0mm}
   \setlength{\labelsep}{1ex}
   \setlength{\leftmargin}{\labelwidth+\labelsep}
   \let\makelabel\mylabel}}{%
   \end{list}\vspace*{-1.3mm}}
\parindent0ex
\parskip2ex
\newcounter{quesito}
\newenvironment{question}{\bigskip\addtocounter{quesito}{1}\bigskip\bigskip\par\textbf{Quesito \thequesito.\kern0ex}}{\par\vspace{\parskip}}
\newenvironment{xquestion}{\bigskip\addtocounter{quesito}{1}\bigskip\bigskip\par\textbf{Quesito \thequesito.\kern1ex}}{\par\vspace{\parskip}}
\newenvironment{answer}{\par\textbf{Risposta\quad}}{\par\vspace{\parskip}}

\pagestyle{empty} 

\begin{document}
\colorbox{blue!10}{\begin{minipage}{\textwidth}
Esperimento con HMM (senza definirli)

\end{minipage}}

\bigskip\bigskip


\begin{pycode}
import random
random.seed(258346435)
\end{pycode}


%%%%%%%%%%%%%%%%%%%%%%%%%
%%%%%%%%%%%%%%%%%%%%%%%%%
%%%%%%%%%%%%%%%%%%%%%%%%%
%%%%%%%%%%%%%%%%%%%%%%%%%
% \begin{question}
% \def\Pr{{\rm Pr\,}}
% \def\pyl#1{\py{latex(#1)}}
% \everymath{\displaystyle}
% \renewcommand{\arraystretch}{2}
% \begin{pycode}
% from sympy import *
% p = random.sample([2, 3, 4, 5, 6, 7, 8], 4)
% pab = Rational(p[0],10)
% pba = Rational(p[1],10)
% paa = 1 - pab
% pbb = 1 - pba
% P = Matrix(( (paa, pba), 
%              (pba, pbb) ))
% M = Matrix(( (paa-1, pba, 0),
%              (1,     1,   1) ))
% x = Symbol('x')
% y = Symbol('y')
% equilibrio = solve_linear_system(M, x, y)
% 
% 
% 
% A1= Rational(p[2],10)
% B1= Rational(p[3],10)
% \end{pycode}
% Consideriamo un cammino aleatorio tra due stati {\sf a} e {\sf b}. Quando siamo nello stato {\sf a} lanciamo una moneta con probabilità di successo $q_{\sf a}=\py{A1}$ quando siamo in {\sf b} lanciamo una moneta con probablilità di successo $q_{\sf b}=\py{B1}$. Noi osserviamo solo il risultato del lancio e non abbiamo informazioni sullo stato. Possiamo però assumere che la distribuzione sugli stati siano in equilibrio (steady state). 
% 
% %Modelliamo il lancio di due monete che modelliamo con le v.a.\@ $A,B\in\{0,1\}$. Quando siamo nello stato {\sf a} laciamo la moneta $A$ che ha probablilità di successo $q_{\sf a}=\py{A1}$ quando siamo in {\sf b} lanciamo $B$ che ha probablilità di successo $q_{\sf b}=\py{B1}$.  
% 
% \hfil
% %\begin{minipage}[t]{0.45\textwidth}
% \begin{tikzpicture}[baseline=2ex]
%     % Draw the states
%     \node[state]             (a) {{\sf a}};
%     \node[state, right=of a] (b) {{\sf b}};
%  
%     % Connect the states with arrows
%     \draw[every loop]
%         (a) edge[bend right, auto=right] node {\py{pab}} (b)
%         (b) edge[bend right, auto=right] node {\py{pba}} (a);
% \end{tikzpicture}
% %\end{minipage}
% \hfil
% \begin{minipage}[t]{40ex}
% $q_{\sf a}\ =\ \Pr(A=1)\ =\ \py{A1}$
% \\[2ex] 
% $q_{\sf b}\ =\ \Pr(B=1)\ =\ \py{B1}$
% \end{minipage}
% 
% 
% 
% \begin{itemize}
% \item[1.] Supponiamo di osservare $1$, qual è la probabilità il processo si trovi nello stato {\sf a}.
% \item[2.] Supponiamo di osservare $01$ in due istanti successivi qual è la probabilità che il processo si trovi in entrambi gli istanti nello stato {\sf a}.
% \end{itemize}
% 
% 
% 
% \begin{answer}
% 
% La matrice di transizione è $P=\displaystyle\pyl{P}$ la distribuzione di equilibrio è $\pi_{\sf a}=\py{equilibrio[x]}$,  $\pi_{\sf b}=\py{equilibrio[y]}$ 
% 
% $\Pr(X_0=1 | S_0={\sf a})$
% 
% {\color{blue}\hfill Risposta 1} 
% 
% {\color{blue}\hfill Risposta 2} 
% \end{answer}
% \end{question}






%%%%%%%%%%%%%%%%%%%%%%%%%
%%%%%%%%%%%%%%%%%%%%%%%%%
%%%%%%%%%%%%%%%%%%%%%%%%%
\begin{question}%11111111111111111111
\def\Pr{{\rm Pr\,}}
\def\pyl#1{\py{latex(#1)}}
\everymath{\displaystyle}
\renewcommand{\arraystretch}{2}
\begin{pycode}
from sympy import *
d = random.sample([2,3,4,5,6], 4)
pab = Rational( random.choice( range( 1,d[0] ) ), d[0] ) 
pba = Rational( random.choice( range( 1,d[1] ) ), d[1] ) 
paa = 1 - pab
pbb = 1 - pba
P  = Matrix([ [paa, pba], 
              [pab, pbb] ])
M  = Matrix(( (paa-1, pba, 0),
             (1,     1,   1) ))
S0 = Matrix([1,0])
x = Symbol('x')
y = Symbol('y')
equilibrio = solve_linear_system(M, x, y)



qa = Rational( random.choice( range( 1,d[2] ) ), d[2] ) 
qb = Rational( random.choice( range( 1,d[3] ) ), d[3] ) 
px2 = qa*(P*P*S0)[0] + qb*(P*P*S0)[1] 
\end{pycode}
Consideriamo il percorso aleatorio descritto in figura. Nello stato {\sf a} viene lanciata una moneta a valori {\sf T} o {\sf C}. La probabilità che esca {\sf T} è $\py{qa}$. Nello stato {\sf b} viene lanciata una moneta con probabilità $\py{qb}$ che esca {\sf T}.

\hfil
\begin{tikzpicture}[baseline=0ex]
    % Draw the states
    \node[state,fill=gray!30]             (a) {{\sf a}};
    \node[state,fill=gray!30, right=3 of a] (b) {{\sf b}};
 
    % Connect the states with arrows
    \draw[every loop]
        (a) edge[bend right, line width=1pt, auto=right] node {\py{pab}} (b)
        (b) edge[bend right, line width=1pt, auto=right] node {\py{pba}} (a);
\end{tikzpicture}
\hfil Le transizioni {\sf aa} e {\sf bb} sono implicite.

Il percorso comincia (al tempo $t{=}0$) dallo stato {\sf a}. Al tempo $t=2$ il risultato del lancio della moneta è {\sf T}. Non abbiamo informazioni sul pescorso, qual è la probabilità che il processo al tempo $t{=}2$ si trovi nello stato {\sf a}~?

Indichiamo con $S_t\in\{{\sf a}, {\sf b}\}$ le variabili aleatorie che danno lo stato al tempo $t$. Indichiamo con $X_t\in\{{\sf T}, {\sf C}\}$ le variabili aleatorie che danno il risultato del lancio al tempo $t$. Si esprima usando queste v.a.\@ la probabilità condizionata che si intende calcolare. 

Esprimere i risultati numerici come frazioni di interi.


\begin{answer}

Dalla figura inferiamo la matrice di transizione $P=\pyl{P}$. 

Il testo riporta le seguenti probabilità:

$\Pr\big(S_0={\sf a}\big)=1$\hfill e per ogni $t$\hfill  $\Pr\big(X_t={\sf T}~|~S_t={\sf a}\big)=\pyl{qa}$,\hfill $\Pr\big(X_t={\sf T}~|~S_t={\sf b}\big)=\pyl{qb}$
\bigskip


$\begin{bmatrix}\Pr(S_1={\sf a})\\\Pr(S_1={\sf b})\end{bmatrix}=P\pyl{S0}=\pyl{P*S0}$
\hfil
$\begin{bmatrix}\Pr(S_2={\sf a})\\\Pr(S_2={\sf b})\end{bmatrix}=P^2\pyl{S0}= P\pyl{P*S0} =\pyl{P*P*S0}$

\medskip
$\Pr\big(X_2={\sf T}\big)\ =\ \Pr\big(X_2={\sf T}~|~S_2={\sf a}\big)\cdot\Pr\big(S_2={\sf a}\big) \ + \ \Pr\big(X_2={\sf T}~|~S_2={\sf b}\big)\cdot\Pr\big(S_2={\sf b}\big)$
\\[2ex]
$\phantom{\Pr\big(X_2={\sf T}\big)}\ =\ \pyl{qa}\cdot\pyl{(P*P*S0)[0]} \ + \ \pyl{qb}\cdot\pyl{(P*P*S0)[1]}\ =\ \pyl{px2}$

\bigskip
{\color{blue}$\Pr(S_2={\sf a}~|~X_2={\sf T})$}$\ =\ \frac{\Pr(X_2={\sf T}~|~S_2={\sf a})\cdot\Pr(S_2={\sf a})}{\Pr(X_2={\sf T})}${\color{blue}$\ =\ \pyl{qa*(P*P*S0)[0]/px2}$\hfill Risposta} 
\end{answer}
\end{question}



% %%%%%%%%%%%%%%%%%%%%%%%%%
% %%%%%%%%%%%%%%%%%%%%%%%%%
% %%%%%%%%%%%%%%%%%%%%%%%%%
% \clearpage%2222222222222222222
% \begin{question}
% \def\Pr{{\rm Pr\,}}
% \def\pyl#1{\py{latex(#1)}}
% \everymath{\displaystyle}
% \renewcommand{\arraystretch}{2}
% \begin{pycode}
% from sympy import *
% d = random.sample([2,3,4,5,6], 4)
% pab = Rational( random.choice( range( 1,d[0] ) ), d[0] ) 
% pba = Rational( random.choice( range( 1,d[1] ) ), d[1] ) 
% paa = 1 - pab
% pbb = 1 - pba
% P  = Matrix([ [paa, pba], 
%               [pab, pbb] ])
% M  = Matrix(( (paa-1, pba, 0),
%              (1,     1,   1) ))
% S0 = Matrix([1,0])
% x = Symbol('x')
% y = Symbol('y')
% equilibrio = solve_linear_system(M, x, y)
% 
% 
% 
% qa = Rational( random.choice( range( 1,d[2] ) ), d[2] ) 
% qb = Rational( random.choice( range( 1,d[3] ) ), d[3] ) 
% px1 = qa*(P*S0)[0] + qb*(P*S0)[1] 
% px2 = qa*(P*P*S0)[0] + qb*(P*P*S0)[1] 
% \end{pycode}
% Consideriamo il percorso aleatorio descritto in figura. Nello stato {\sf a} viene lanciata una moneta a valori {\sf T} o {\sf C}. La probabilità che esca {\sf T} è $\py{qa}$. Nello stato {\sf b} viene lanciata una moneta con probabilità $\py{qb}$ che esca {\sf T}.
% 
% \hfil
% \begin{tikzpicture}[baseline=0ex]
%     % Draw the states
%     \node[state,fill=gray!30]             (a) {{\sf a}};
%     \node[state,fill=gray!30, right=3 of a] (b) {{\sf b}};
%  
%     % Connect the states with arrows
%     \draw[every loop]
%         (a) edge[bend right, line width=1pt, auto=right] node {\py{pab}} (b)
%         (b) edge[bend right, line width=1pt, auto=right] node {\py{pba}} (a);
% \end{tikzpicture}
% \hfil Le transizioni {\sf aa} e {\sf bb} sono impicite.
% 
% Il percorso comincia (al tempo $t{=}0$) dallo stato {\sf b}. Al tempo $t=1$ il risultato del lancio della moneta è {\sf T} ma non abbiamo informazioni sul pescorso. Qual è la probabilità di osservare {\sf C}  al tempo $t{=}2$~?
% 
% Indichiamo con $S_t\in\{{\sf a}, {\sf b}\}$ le variabili aleatorie che danno lo stato al tempo $t$. Indichiamo con $X_t\in\{{\sf T}, {\sf C}\}$ le variabili aleatorie che danno il risultato del lancio al tempo $t$. Si esprima usando queste v.a.\@ la probabilità condizionata che si intende calcolare. 
% 
% Esprimere i risultati numerici come frazioni di interi.
% 
% 
% \begin{answer}
% 
% Dalla figura inferiamo la matrice di transizione $P=\pyl{P}$. 
% 
% Il testo riporta le seguenti probabilità:
% 
% $\Pr\big(S_0={\sf b}\big)=1$\hfill e per ogni $t$\hfill  $\Pr\big(X_t={\sf T}~|~S_t={\sf a}\big)=\pyl{qa}$,\hfill $\Pr\big(X_t={\sf T}~|~S_t={\sf b}\big)=\pyl{qb}$
% \bigskip
% 
% 
% $\begin{bmatrix}\Pr(S_1={\sf a})\\\Pr(S_1={\sf b})\end{bmatrix}=P\pyl{S0}=\pyl{P*S0}$
% \hfil
% $\begin{bmatrix}\Pr(S_2={\sf a})\\\Pr(S_2={\sf b})\end{bmatrix}=P^2\pyl{S0}= P\pyl{P*S0} =\pyl{P*P*S0}$
% 
% \medskip
% $\Pr\big(X_1={\sf T}\big)\ =\ \Pr\big(X_1={\sf T}~|~S_1={\sf a}\big)\cdot\Pr\big(S_1={\sf a}\big) \ + \ \Pr\big(X_1={\sf T}~|~S_1={\sf b}\big)\cdot\Pr\big(S_1={\sf b}\big)$
% \\[2ex]
% $\phantom{\Pr\big(X_1={\sf T}\big)}\ =\ \pyl{qa}\cdot\pyl{(P*S0)[0]} \ + \ \pyl{qb}\cdot\pyl{(P*S0)[1]}\ =\ \pyl{px1}$
% 
% \bigskip
% $\Pr(S_1={\sf a}~|~X_1={\sf T})\ =\ \frac{\Pr(X_1={\sf T}~|~S_1={\sf a})\cdot\Pr(S_1={\sf a})}{\Pr(X_1={\sf T})}\ =\ \pyl{qa*(P*S0)[0]/px1}$
% 
% \bigskip
% $\Pr\big(X_2={\sf C}~|~X_1={\sf T}\big)\ =\ \Pr\big(X_2={\sf C}~|~X_1={\sf T},\ S_2={\sf a}\big)\cdot\Pr\big(S_2={\sf a}\big) \ + \ \Pr\big(X_1={\sf T}~|~X_1={\sf T},\ S_2={\sf b}\big)\cdot\Pr\big(S_1={\sf b}\big)$
% 
% \end{answer}
% \end{question}






%%%%%%%%%%%%%%%%%%%%%%%%%
%%%%%%%%%%%%%%%%%%%%%%%%%
%%%%%%%%%%%%%%%%%%%%%%%%%
%%%%%%%%%%%%%%%%%%%%%%%%%
% \begin{question}
% \def\Pr{{\rm Pr\,}}
% \def\pyl#1{\py{latex(#1)}}
% \everymath{\displaystyle}
% \renewcommand{\arraystretch}{2}
% \begin{pycode}
% from sympy import *
% p = random.sample(range(2,9), 3)
% pab = Rational(p[0],10)
% pba = Rational(p[0]+1,10)
% paa = 1 - pab
% pbb = 1 - pba
% P  = Matrix([ [paa, pba], 
%               [pab, pbb] ])
% M  = Matrix(( (paa-1, pba, 0),
%              (1,     1,   1) ))
% S0 = Matrix([1,0])
% x = Symbol('x')
% y = Symbol('y')
% equilibrio = solve_linear_system(M, x, y)
% 
% 
% 
% qa = Rational(p[1],10)
% qb = Rational(p[2],10)
% px1 = qa*(P*S0)[0] + qb*(P*S0)[1] 
% \end{pycode}
% Consideriamo il percorso aleatorio descritto in figura. Nello stato {\sf a} viene lanciata una moneta a valori {\sf T} o {\sf C}. la probabilità che esca {\sf T} è $\py{qa}$. Nello stato {\sf b} viene lanciata una moneta simile, ma la probabilità di {\sf T} è $\py{qb}$.
% 
% \hfil
% \begin{tikzpicture}[baseline=0ex]
%     % Draw the states
%     \node[state,fill=gray!30]             (a) {{\sf a}};
%     \node[state,fill=gray!30, right=3 of a] (b) {{\sf b}};
%  
%     % Connect the states with arrows
%     \draw[every loop]
%         (a) edge[bend right, line width=1pt, auto=right] node {\py{pab}} (b)
%         (b) edge[bend right, line width=1pt, auto=right] node {\py{pba}} (a);
% \end{tikzpicture}
% \hfil(Le transizioni {\sf aa} e {\sf bb} sono state lasciate impicite.) 
% 
% Il percorso comincia (al tempo $t{=}0$) dallo stato {\sf a}. Noi osserviamo solo il risultato del lancio della moneta e non abbiamo informazioni sullo stato.
% 
% Al tempo $t{=}1$ osserviamo {\sf T} all'istante successivo (tempo $t{=}2$) osserviamo  {\sf T}.
% 
% \begin{itemize}
% \item[1.] Qual è la probabilità che il processo al tempo $t{=}1$ si trovi nello stato {\sf a}~?
% \item[2.] Qual è la probabilità che il processo si trovi nello stato {\sf a} al tempo $t{=}2$~?
% \end{itemize}
% 
% Siano $S_t\in\{{\sf a}, {\sf b}\}$ le variabili aleatorie che danno lo stato al tempo $t$. Sia $X_t\in\{{\sf T}, {\sf C}\}$ le variabili aleatorie che danno il risultato del lancio al tempo $t$. Si esprima usando queste v.a.\@ la probabilità condizionata che si intende calcolare.
% 
% Esprimere i numeri razionali come frazioni.
% 
% 
% \begin{answer}
% 
% Dalla figura inferiamo che matrice di transizione è $P=\pyl{P}$
% 
% Il testo riferisce i valori \hfill$\Pr\big(S_0={\sf a}\big)=1$\hfill  $\Pr\big(X_t={\sf T}~|~S_t={\sf a}\big)=\pyl{qa}$\hfill $\Pr\big(X_t={\sf T}~|~S_t={\sf b}\big)=\pyl{qb}$
% 
% $\begin{bmatrix}\Pr(S_1={\sf a})\\\Pr(S_1={\sf b})\end{bmatrix}=P\pyl{S0}=\pyl{P*S0}$
% \hfil
% $\begin{bmatrix}\Pr(S_2={\sf a})\\\Pr(S_2={\sf b})\end{bmatrix}=P^2\pyl{S0}= P\pyl{P*S0} =\pyl{P*P*S0}$
% 
% \bigskip
% $\Pr\big(X_1={\sf T}\big)\ =\ \Pr\big(X_1={\sf T}~|~S_1={\sf a}\big)\cdot\Pr\big(S_1={\sf a}\big) \ + \ \Pr\big(X_1={\sf T}~|~S_1={\sf b}\big)\cdot\Pr\big(S_1={\sf b}\big)$
% \\[2ex]
% $\phantom{\Pr\big(X_1={\sf T}\big)}\ =\ \pyl{qa}\cdot\pyl{(P*S0)[0]} \ + \ \pyl{qb}\cdot\pyl{(P*S0)[1]}\ =\ \pyl{px1}$
% 
% \bigskip
% {\color{blue}$\Pr(S_1={\sf a}~|~X_1={\sf T})$}$\ =\ \frac{\Pr(X_1={\sf T}~|~S_1={\sf a})\cdot\Pr(S_1={\sf a})}{\Pr(X_1={\sf T})}${\color{blue}$\ =\ \pyl{qa*(P*S0)[0]/px1}$\hfill Risposta 1} 
% 
% {\color{blue}\hfill Risposta 2} 
% \end{answer}
% \end{question}



\end{document}

