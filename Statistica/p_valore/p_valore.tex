\documentclass[11pt,twoside,a4paper]{article}
\usepackage[T1]{fontenc}
\usepackage[utf8]{inputenc}
\usepackage[top=20mm, head=6mm, headsep=6mm, foot=6mm, bottom= 15mm, left=15mm, right=15mm]{geometry}
\usepackage{amsmath}
\usepackage{calc} 
\usepackage{pythontex}
\newcommand{\mylabel}[1]{#1\hfill}
\renewenvironment{itemize}
  {\begin{list}{$\triangleright$}{%
   \setlength{\parskip}{0mm}
   \setlength{\topsep}{.4\baselineskip}
   \setlength{\rightmargin}{0mm}
   \setlength{\listparindent}{0mm}
   \setlength{\itemindent}{0mm}
   \setlength{\labelwidth}{2ex}
   \setlength{\itemsep}{.4\baselineskip}
   \setlength{\parsep}{0mm}
   \setlength{\partopsep}{0mm}
   \setlength{\labelsep}{1ex}
   \setlength{\leftmargin}{\labelwidth+\labelsep}
   \let\makelabel\mylabel}}{%
   \end{list}\vspace*{-1.3mm}}
\parindent0ex
\parskip2ex
\def\Pr{{\rm Pr\,}}
\def\Ex{{\rm E\,}}
\def\Var{{\rm Var\,}}
\newcounter{quesito}
\newenvironment{question}{\bigskip\addtocounter{quesito}{1}\par\textbf{Quesito \thequesito.\kern1ex}}{\vspace{\parskip}}
\newenvironment{xquestion}{\bigskip\addtocounter{quesito}{1}\par\textbf{Quesito \thequesito.\kern1ex}}{\vspace{\parskip}}
\newenvironment{answer}{\par\textbf{Risposta\quad}}{\vspace{\parskip}}

\pagestyle{empty} 

\begin{document}
Domande (qualcuna capziosa e artificiale) per verificare la comprensione del significato di p-valore.\\
N.B. Le domande contengono anche informazioni irrilevanti.\bigskip

\begin{pycode}
import random
\end{pycode}

\begin{question}
\begin{pycode}
n = random.choice([2,3])
pval = random.choice([0.05,0.02])
test = random.choice(['T-test a due code','T-test a una coda','Z-test ($\sigma$ nota, una coda)', 'Z-test ($\sigma$ nota, due code)'])
dim  = random.choice(['sempre della stessa dimensione', 'di dimensione crescente'])
risposta = round(1-(1-pval)**n, 3)
\end{pycode}
Ripetiamo \py{n} volte lo stesso \py{test} con campioni \py{dim}.
Assumendo vera $H_0$, qual è la probabilità che in almeno uno di questi test il p-valore risulti $\le\py{pval}$~? 

Nel caso non sia possibile determinare il valore esatto ma solo un limite superiore/inferiore. Si scelga tra le seguenti opzioni la più opportuna.
\begin{itemize}
\item[1.] La probabilità è $=\dots$
\item[2.] La probabilità è $\le\dots$
\item[3.] La probabilità è $\ge\dots$
\item[4.] Non ci sono sufficienti informazioni per stimare questa probabilità.
\end{itemize}
\begin{answer}
{\color{blue}1. La probabilità è $=1-(\py{1-pval})^{\py{n}}=\py{risposta}$.}
\end{answer}
\end{question}






\begin{question}
\begin{pycode}
n = random.choice([30,40,50])
pval = random.choice([0.05,0.04,0.02])
test = random.choice(['T-test a due code','T-test a una coda','Z-test ($\sigma$ nota, una coda)', 'Z-test ($\sigma$ nota, due code)'])
condizioni  = random.choice(['con un campione della stessa dimensione', 'con un campione doppio'])
factor =  random.choice([1,2])
\end{pycode}
Abbiamo fatto un \py{test} con un campione di dimensione \py{n} e abbiamo ottenuto come p-valore \py{pval}.
Assumendo vera $H_0$, qual è la probabilità che ripetendo il test una seconda volta \py{condizioni} il p-valore risulti $\le\py{pval/factor}$~?

Nel caso non sia possibile determinare il valore esatto ma solo un limite superiore/inferiore. Si scelga tra le seguenti opzioni la più opportuna.
\begin{itemize}
\item[1.] La probabilità è $=\dots$
\item[2.] La probabilità è $\le\dots$
\item[3.] La probabilità è $\ge\dots$
\item[4.] Non ci sono sufficienti informazioni per stimare questa probabilità.
\end{itemize}
\begin{answer}
{\color{blue}1. La probabilità è $=\py{pval/factor}$.}
\end{answer}
\end{question}


\begin{question}
\begin{pycode}
pval = random.choice([0.03,0.02,0.01])
test = random.choice(['T-test a due code','T-test a una coda','Z-test ($\sigma$ nota, una coda)', 'Z-test ($\sigma$ nota, due code)'])
\end{pycode}
Abbiamo fatto un \py{test} e abbiamo ottenuto come p-valore \py{pval}.
Assumendo vera $H_A$, qual è la probabilità che ripetendo il test una seconda volta con un campione della stessa dimensione il p-valore risulti di nuovo $\le\py{pval}$~?

Nel caso non sia possibile determinare il valore esatto ma solo un limite superiore/inferiore. Si scelga tra le seguenti opzioni la più opportuna.
\begin{itemize}
\item[1.] La probabilità è $=\dots$
\item[2.] La probabilità è $\le\dots$
\item[3.] La probabilità è $\ge\dots$
\item[4.] Non ci sono sufficienti informazioni per stimare questa probabilità.
\end{itemize}
\begin{answer}
{\color{blue}4. Non ci sono sufficienti informazioni per stimare questa probabilità.}
\end{answer}
\end{question}





\end{document}


