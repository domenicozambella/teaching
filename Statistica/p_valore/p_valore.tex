\documentclass[11pt,twoside,a4paper]{article}
\usepackage[T1]{fontenc}
\usepackage[utf8]{inputenc}
\usepackage[top=20mm, head=6mm, headsep=6mm, foot=6mm, bottom= 15mm, left=15mm, right=15mm]{geometry}
\usepackage{amsmath}
\usepackage{calc} 
\usepackage{pythontex}
\newcommand{\mylabel}[1]{#1\hfill}
\renewenvironment{itemize}
  {\begin{list}{$\triangleright$}{%
   \setlength{\parskip}{0mm}
   \setlength{\topsep}{.4\baselineskip}
   \setlength{\rightmargin}{0mm}
   \setlength{\listparindent}{0mm}
   \setlength{\itemindent}{0mm}
   \setlength{\labelwidth}{2ex}
   \setlength{\itemsep}{.4\baselineskip}
   \setlength{\parsep}{0mm}
   \setlength{\partopsep}{0mm}
   \setlength{\labelsep}{1ex}
   \setlength{\leftmargin}{\labelwidth+\labelsep}
   \let\makelabel\mylabel}}{%
   \end{list}\vspace*{-1.3mm}}
\parindent0ex
\parskip2ex
\def\Pr{{\rm Pr\,}}
\def\Ex{{\rm E\,}}
\def\Var{{\rm Var\,}}
\newcounter{quesito}
\newenvironment{question}{\bigskip\addtocounter{quesito}{1}\par\textbf{Quesito \thequesito.\kern1ex}}{\vspace{\parskip}}
\newenvironment{xquestion}{\bigskip\addtocounter{quesito}{1}\par\textbf{Quesito \thequesito.\kern1ex}}{\vspace{\parskip}}
\newenvironment{answer}{\par\textbf{Risposta\quad}}{\vspace{\parskip}}

\pagestyle{empty} 

\begin{document}
\colorbox{blue!10}{\begin{minipage}{\textwidth}
Domande (capziose e artificiali) per verificare la comprensione del significato di p-valore.\bigskip

N.B. Spesso le domande contengono informazioni irrilevanti.
\end{minipage}}

\bigskip\bigskip

\begin{pycode}
import random
random.seed(25846456445)
ESAME = False
\end{pycode}

\begin{question} %1
\begin{pycode}
dim = 'di dimensione $n = 25$'
if ESAME: dim  = random.choice(['della stessa dimensione', 'di dimensione crescente', 'di dimesione $n =16$'])
n = random.choice([2,3])

test = 'T-test a due code'
if ESAME: test = random.choice(['T-test coda inferiore','T-test coda superiore', 'Z-test ($\sigma$ nota) a due code', 'Z-test ($\sigma$ nota) coda inferiore' 'Z-if ESAME: test ($\sigma$ nota) coda superiore'])

pval = random.choice([0.05, 0.06, 0.07,0.08, 0.09, 0.1])

dis = '\\le'
if ESAME: dis = random.choice(['\\le', '\\ge'] )

if   dis=='\\le' : risposta = '=1-({:0.2g})^{:d} = {:g}'.format(1-pval, n, 1-(1-pval)**n )
elif dis=='\\ge' : risposta = '=1-({:0.2g})^{:d} = {:g}'.format(pval, n, 1-(pval)**n )
\end{pycode}
Ripetiamo \py{n} volte lo stesso \py{test} con campioni \py{dim}.
Assumendo vera $H_0$, qual è la probabilità che in almeno uno di questi test il p-valore risulti $\py{dis}\py{pval}$~? 

Si scelga tra le seguenti opzioni la più opportuna.
\begin{itemize}
\item[1.] La probabilità è $=\dots$ (specificare)
\item[2.] La probabilità è $<\dots$ (specificare)
\item[3.] La probabilità è $>\dots$ (specificare)
\item[4.] Non ci sono sufficienti informazioni per stimare questa probabilità.
\end{itemize}
\begin{answer}
{\color{blue}1. La probabilità è $\py{risposta}$.}
\end{answer}
\end{question}




\begin{question} %22222222
\begin{pycode}
n = 25
if ESAME: n = random.choice([16,36])

dim = 'di dimensione doppia'
if ESAME: dim  = random.choice(['di dimensione doppia','di dimensione $n/2$'])

n = 25
if ESAME: n = random.choice([30,40,50])

test = 'T-test a due code'
if ESAME: test = random.choice(['T-test coda inferiore','T-test coda superiore', 'Z-test ($\sigma$ nota) a due code', 'Z-test ($\sigma$ nota) coda inferiore' 'Z-if ESAME: test ($\sigma$ nota) coda superiore'])

pval = random.choice([0.05,0.04,0.02])
factor =  random.choice([0.5,2])


dis = '\\ge'
if ESAME: dis = random.choice(['\\le', '\\ge'] )

if   dis=='\\le' : risposta = '= {:g}'.format(pval/factor)
elif dis=='\\ge' : risposta = '= 1-{:g} = {:g}'.format(pval/factor, 1 - pval/factor)
\end{pycode}
Abbiamo fatto un \py{test} con un campione di dimensione $n=\py{n}$ e abbiamo ottenuto come p-valore \py{pval}.
Assumendo vera $H_0$, qual è la probabilità che, ripetendo il test una seconda volta con un campione \py{dim}, il p-valore risulti $\le\py{pval/factor}$~?

Si scelga tra le seguenti opzioni la più opportuna.
\begin{itemize}
\item[1.] La probabilità è $=\dots$ (specificare)
\item[2.] La probabilità è $<\dots$ (specificare)
\item[3.] La probabilità è $>\dots$ (specificare)
\item[4.] Non ci sono sufficienti informazioni per stimare questa probabilità.
\end{itemize}
\begin{answer}
{\color{blue}1. La probabilità è $\py{risposta}$.}
\end{answer}
\end{question}




\begin{question} %333333333
\begin{pycode}
n = 25
if ESAME: n = random.choice([30,40,50])
test = 'T-test a due code'
test = random.choice(['T-test coda inferiore','T-test coda superiore', 'Z-test ($\sigma$ nota) a due code', 'Z-test ($\sigma$ nota) coda inferiore' 'Z-if ESAME: test ($\sigma$ nota) coda superiore'])
pval = random.choice([0.03,0.02,0.01])
\end{pycode}
Abbiamo fatto un \py{test} con un campione di dimensione $n=\py{n}$ e abbiamo ottenuto come p-valore \py{pval}.
Assumendo vera $H_A$, qual è la probabilità che ripetendo il test una seconda volta con un campione della stessa dimensione il p-valore risulti di nuovo $\le\py{pval}$~?

Nel caso non sia possibile determinare il valore esatto ma solo un limite superiore/inferiore. Si scelga tra le seguenti opzioni la più opportuna.
\begin{itemize}
\item[1.] La probabilità è $=\dots$ (specificare)
\item[2.] La probabilità è $<\dots$ (specificare)
\item[3.] La probabilità è $>\dots$ (specificare)
\item[4.] Non ci sono sufficienti informazioni per stimare questa probabilità.
\end{itemize}
\begin{answer}
{\color{blue}4. Non ci sono sufficienti informazioni per stimare questa probabilità.}
\end{answer}
\end{question}


\clearpage
\begin{question} %444444444444444
\begin{pycode}
from sympy import Rational, Integer
from scipy.stats import norm
mu = random.choice( [ [1,2,4], [2,4,7], [2,6,9] ] ) 
n = random.choice([2,3,4,5])
sigma = random.choice([2,3,5])
xbar = mu[1] + random.choice( [ -1, 1, 2 ] ) 
\end{pycode}
Preleviamo un campione di rango $n=\py{n**2}$ da una popolazione con distribuzione $N(\mu,\sigma^2)$. Sappiamo che la deviazione standard è $\sigma=\py{sigma}$. La media $\mu$ invece potrebbe avere uno qualsiasi dei tre valori $\py{mu[0]}$, $\py{mu[1]}$, o $\py{mu[2]}$. 

Vogliamo testare $H_0:\mu=\py{mu[1]}$ contro $H_A:\mu\in\{\py{mu[0]},\py{mu[2]}\}$.
\begin{itemize}
\item[1.] Che test facciamo?
\item[2.] Se la media del campione di cui sopra è $\bar x=\py{xbar}$, quant'è il p-valore?
\item[3.] Dato questo risultato, la probabilità di $\mu\in\{\py{mu[0]},\py{mu[2]}\}$ è \ldots (si scelga la risposta corretta ta le seguenti)\medskip

(a) $=$ p-valore;\hfill (b) $=1-$ p-valore;\hfill (c) $2/3$;\hfill (d) Non ci sono sufficienti informazioni per dirlo.
\end{itemize}

Esprimere il risutato numerico tramite (solo) le funzioni elencate in calce.
\begin{answer}

{\color{blue}Facciamo uno z-test a due code.\hfill Risposta 1}

$\Pr\big(|\bar X|\ge\py{xbar}\big)\ =\ \Pr\bigg(\bigg|\dfrac{\bar X-\py{mu[1]}}{\sigma/\sqrt{n}}\bigg|\ge\dfrac{|\py{xbar}-\py{mu[1]}| }{\sigma/\sqrt{n} }\bigg)\ =\ \Pr\big(|Z|\ge\py{Rational(Integer(n*abs(xbar-mu[1]) ) / Integer(sigma) ) }\big)$

$\phantom{\Pr\big(|\bar X|\ge\py{xbar}\big)}\ =\ ${\color{blue}\tt 2 * norm.cdf(-\py{Rational(Integer(n*abs(xbar-mu[1]) ) / Integer(sigma) ) }) }{\tt\ =  \ \py{round( 2 * norm.cdf(-(n*abs(xbar-mu[1]) / sigma) ), 3) } }{\color{blue}\hfill Risposta 2}

\medskip
{\color{blue}(d) Non ci sono sufficienti informazioni per dirlo.\hfill Risposta 3}

\end{answer}
\end{question}


\begin{question} %5555555
\begin{pycode}
from sympy import Rational, Integer
from scipy.stats import norm
mu = random.choice( [ [1,4,5], [2,6,7], [3,8,9] ] ) 
n = random.choice([2,3,4,5])
sigma = random.choice([2,3,5])
xbar = mu[0] + random.choice( [ 1, 2 ] ) 
\end{pycode}
Preleviamo un campione di rango $n=\py{n**2}$ da una popolazione con distribuzione $N(\mu,\sigma^2)$. Sappiamo che la deviazione standard è $\sigma=\py{sigma}$. La media $\mu$ invece potrebbe avere uno qualsiasi dei tre valori $\py{mu[0]}$, $\py{mu[1]}$, o $\py{mu[2]}$. 

Vogliamo testare $H_0:\mu=\py{mu[0]}$ contro $H_A:\mu\in\{\py{mu[1]},\py{mu[2]}\}$.
\begin{itemize}
\item[1.] Che test facciamo?
\item[2.] Se la media del campione di cui sopra è $\bar x=\py{xbar}$, quant'è il p-valore?
\end{itemize}

Esprimere il risutato numerico tramite (solo) le funzioni elencate in calce.
\begin{answer}

{\color{blue}Facciamo uno z-test coda superiore.\hfill Risposta 1}

$\Pr\big(\bar X\ge\py{xbar}\big)\ =\ \Pr\bigg(\dfrac{\bar X-\py{mu[0]}}{\sigma/\sqrt{n}}\ge\dfrac{\py{xbar}-\py{mu[0]} }{\sigma/\sqrt{n} }\bigg)\ =\ \Pr\big(Z\ge\py{Rational(Integer(n*(xbar-mu[0]) ) / Integer(sigma) ) }\big)$

$\phantom{\Pr\big(\bar X\ge\py{xbar}\big)}\ =\ ${\color{blue}\tt\ 1 - norm.cdf(\py{Rational(Integer(n*(xbar-mu[0]) ) / Integer(sigma) ) }) }{\tt\ =  \ \py{round( 2 * norm.cdf(-(n*(xbar-mu[0]) / sigma) ), 3) } }{\color{blue}\hfill Risposta 2}



\end{answer}
\end{question}

\vfill
\hrulefill

Si assuma noto il valore delle seguenti funzioni della libreria {\tt scipy.stats\/}

{\tt norm.cdf(z)} = $\Pr\big(Z<{\tt z}\big)$ per $Z\sim N(0,1)$ 

{\tt norm.ppf($\alpha$)} = $z_\alpha$ dove $z_\alpha$ è tale che $\Pr\big(Z<z_\alpha\big)=\alpha$ per $Z\sim N(0,1)$ 


\end{document}


