\documentclass[11pt,twoside,a4paper]{article}
\usepackage[italian]{babel}
\usepackage[T1]{fontenc}
\usepackage[utf8]{inputenc}
\usepackage[top=20mm, head=6mm, headsep=6mm, foot=6mm, bottom= 15mm, left=15mm, right=15mm]{geometry}
\usepackage{amsmath}
\usepackage{calc} 
\usepackage{pythontex}
\usepackage{tikz}
\usetikzlibrary{automata,positioning}
\newcommand{\mylabel}[1]{#1\hfill}
\renewenvironment{itemize}
  {\begin{list}{$\triangleright$}{%
   \setlength{\parskip}{0mm}
   \setlength{\topsep}{.4\baselineskip}
   \setlength{\rightmargin}{0mm}
   \setlength{\listparindent}{0mm}
   \setlength{\itemindent}{0mm}
   \setlength{\labelwidth}{2ex}
   \setlength{\itemsep}{.4\baselineskip}
   \setlength{\parsep}{0mm}
   \setlength{\partopsep}{0mm}
   \setlength{\labelsep}{1ex}
   \setlength{\leftmargin}{\labelwidth+\labelsep}
   \let\makelabel\mylabel}}{%
   \end{list}\vspace*{-1.3mm}}
\parindent0ex
\parskip2ex
\newcounter{quesito}
\newenvironment{question}{\bigskip\addtocounter{quesito}{1}\bigskip\bigskip\par\textbf{Quesito \thequesito.\kern0ex}}{\par\vspace{\parskip}}
\newenvironment{xquestion}{\bigskip\addtocounter{quesito}{1}\bigskip\bigskip\par\textbf{Quesito \thequesito.\kern1ex}}{\par\vspace{\parskip}}
\newenvironment{answer}{\par\textbf{Risposta\quad}}{\par\vspace{\parskip}}

\pagestyle{empty} 

\begin{document}
\colorbox{blue!10}{\begin{minipage}{\textwidth}
Esperimento con catene di Markov (senza definirle)

\end{minipage}}


\begin{pycode}
import random
random.seed(258346435)
ESAME = False
\end{pycode}


%%%%%%%%%%%%%%%%%%%%%%%%%
%%%%%%%%%%%%%%%%%%%%%%%%%
%%%%%%%%%%%%%%%%%%%%%%%%%
%%%%%%%%%%%%%%%%%%%%%%%%%
\begin{question}
\def\Pr{{\rm Pr\,}}
\def\pyl#1{\py{latex(#1)}}
\everymath{\displaystyle}
\renewcommand{\arraystretch}{2}
\begin{pycode}
from sympy import *
units = Integer( random.choice([2,3,4,5,6,7] ) )
pw =  Integer(6)
pz =  Integer(6)
wp =  Integer(2)
wz =  Integer(1)
zw =  Integer(5)
M = Matrix([ [100-pz-pw, wp,        0     ],
             [pw,        100-wp-wz, zw    ],
             [pz,        wz,        100-zw] ] )

x0 = Matrix([ 0, units, 0 ] )

d = (M/100).eigenvals()
autovalori = dict( zip( [r'\lambda_1', r'\lambda_2', r'\lambda_3'], list(d.keys() ) ) )

p = Symbol('p')
w = Symbol('w')
z = Symbol('z')


M1 = M - 100*eye(3)
M1 = M1.row_insert(3, Matrix([ [1, 1, 1] ] ) )
M1 = M1.col_insert(3, Matrix([0, 0, 0, units] ) )
equilibrio = solve_linear_system(M1, p,w,z)

\end{pycode}
Vogliamo modellare il trasferimento di nutrienti nella catena alimentare di un acquario. La catena alimentare consite di tre compartimenti: phytoplankton ({\sf P}), acqua ({\sf W}), zooplankton ({\sf Z}).

Dissolviamo $\py{units}$ unità del radioisotopo $^{14}C$. Qui sotto riassumiamo le percentuali di $^{14}C$ che ogni ora passano tra i compartimenti. Possiamo assumere che la concentrazione di radioisotopo nel sistema rimanga constante quindi gli isotopi che non vengono trasferiti rimangono nello stesso compartimento (nel grafico sono sottointese)

\hfil
\begin{tikzpicture}[font=\sffamily]
 
        % Setup the style for the states
        \tikzset{node style/.style={state, 
                                    minimum width=1.5cm,
                                    line width=0.5mm,
                                    fill=gray!20!white}}
 
        % Draw the states
        \node[node style] at (0, 0)     (P)     {P};
        \node[node style] at (6, 0)     (W)     {W};
        \node[node style] at (12, 0)    (Z)     {Z};
 
        % Connect the states with arrows
        \draw[every loop,
              auto=right,
              line width=0.5mm,
              >=latex]
            (P)     edge[bend right=20, auto=right]  node {$\py{pz}\%$} (Z)
            (P)     edge[bend right=5, auto=right]   node {$\py{pw}\%$} (W)
           %(W)     edge[bend right=30]            node {0.3} (P)
            (W)     edge[bend right=5, auto=right]  node {$\py{wz}\%$} (Z)
            (Z) edge[bend right=20]                  node {$\py{zw}\%$} (W)
            (W) edge[bend right=20, auto=right]      node {$\py{wp}\%$} (P);
\end{tikzpicture}




\begin{itemize}
\item[1.]  Scrivere la matrice $M$ che descrive l'evoluzione del processo $\vec x_{n+1}=M\vec x_{n}$ dove 
$\vec x_n\ =\ 
% \begin{bmatrix*}[l]
% p_n\\
% z_n\\
% w_n\\
% \end{bmatrix*}
[p_n,w_n,z_n]^{\rm T}$ dove $p_n$, $w_n$, e $z_n$ sono le quantità radioisotopo dopo $n$ ore nei rispettivi compartimenti.
\item[2.] Descrivere $\vec x_{0}$ lo stato iniziale del sistema descritto nel testo.
\item[3.] Calcolare $\vec x_{1}$
\item[4.] La matrice $M$ ha questo insieme di autovalori 

\hfil$\py{latex(autovalori)[8:-9 ].replace(':','=')}$.\medskip

Dire se partendo da $\vec x_0$ si raggiunge uno stato di equilibrio e nel caso calcolarlo.
\end{itemize}



\begin{answer}
\vskip-4ex
La matrice di transizione è {\color{blue}$M=\displaystyle\frac{1}{100}\pyl{M}$}\quad 
Lo stato iniziale è {\color{blue}$\vec x_{0}=\pyl{x0}$\hfill Risposte 1 e 2} 

{\color{blue}$\displaystyle\vec x_{1}\ =\ $}$M\,\vec x_{0}\ =\ \frac{1}{100}\pyl{M}\pyl{x0}\  =\ ${\color{blue}$\pyl{(M*x0)/100}$\hfill Risposta 3} 

Notiamo che $|\lambda_2|, |\lambda_3|<|\lambda_1|=1$. Quindi la concentrazione tende all'equilibrio $\vec x_\infty$ che è l'autovettore corrispondente a $\lambda_1=1$. Possiamo calcolare  $\vec x_\infty$ risolvendo il sistema:

\hfil$M\,\big[p,\ w,\ z\big]^T = \,\big[p,\ w,\ z\big]^T$\qquad e\qquad $p+w+z=\py{units}$



{\color{blue}$p_\infty=\pyl{equilibrio[p]}$,\quad $w_\infty=\pyl{equilibrio[w]}$,\quad $z_\infty=\pyl{equilibrio[z]}$\hfill Risposta 4} 
\end{answer}
\end{question}






\end{document}

