\documentclass[10pt]{article}
\usepackage[utf8]{inputenc}
\usepackage[a4paper,height=24cm,width=14cm]{geometry}
\usepackage[italian]{babel}
\usepackage{amssymb}
\usepackage{dsfont}
\usepackage{calc}
\usepackage{graphicx}
\usepackage{pstricks}
\usepackage{pst-node}
\usepackage{fourier}
\usepackage{euscript}
\usepackage{amsmath,amssymb, amsthm}

\def\lh{\textrm{lh}}
\def\phi{\varphi}
\def\P{\EuScript P}
\def\M{\EuScript M}
\def\D{\EuScript D}
\def\U{\EuScript U}
\def\S{\EuScript S}
\def\sm{\smallsetminus}
\def\niff{\nleftrightarrow}
\def\proves{\vdash}
\def\ZZ{\mathds Z}
\def\NN{\mathds N}
\def\QQ{\mathds Q}
\def\RR{\mathds R}
\def\<{\langle}
\def\>{\rangle}
\def\E{\exists}
\def\A{\forall}
\def\0{\varnothing}
\def\imp{\rightarrow}
\def\iff{\leftrightarrow}
\def\IMP{\Rightarrow}
\def\IFF{\Leftrightarrow}
\def\range{\textrm{im}}
\def\Mod{\textrm{Mod}}
\def\Aut{\textrm{Aut}}
\def\Th{\textrm{Th}}
\def\acl{\textrm{acl}}
\def\eq{{\rm eq}}
\def\tp{\textrm{tp}}
\def\equivL{\stackrel{\smash{\scalebox{.5}{\rm L}}}{\equiv}}

\newcommand{\labella}[1]{{\sf\footnotesize #1}\hfill}
\renewenvironment{itemize}
  {\begin{list}{$\triangleright$}{%
   \setlength{\parskip}{0mm}
   \setlength{\topsep}{0mm}
   \setlength{\rightmargin}{0mm}
   \setlength{\listparindent}{0mm}
   \setlength{\itemindent}{0mm}
   \setlength{\labelwidth}{3ex}
   \setlength{\itemsep}{0mm}
   \setlength{\parsep}{0mm}
   \setlength{\partopsep}{0mm}
   \setlength{\labelsep}{1ex}
   \setlength{\leftmargin}{\labelwidth+\labelsep}
   \let\makelabel\labella}}{%
   \end{list}}
%\def\ssf#1{\textsf{\small #1}}
\newcounter{ex}
\newenvironment{exercise}{\bigskip\addtocounter{ex}{1}\textbf{Esercizio \theex.\quad}}{}
\pagestyle{empty}
\parindent0ex
\parskip2ex
\raggedbottom
\def\nsR{{}^*\!\RR}
\def\ssf#1{\textsf{#1}}
\renewcommand{\baselinestretch}{1.3}


\usepackage{fancyhdr}
\pagestyle{fancy}
\lhead{Logica Matematica a.a.~2018/19}
\rhead{}
\cfoot{}
\rfoot{\rput(1.5,-0.5){\small\thepage}}

\begin{document}


\clearpage%%%%%%%%%%%%%%%%%%%%%%%%%%%%%%%
\rhead{Marco Mombelli, Simone Ramello, Davide Manca}\setcounter{ex}{0}


\begin{exercise}
Si consideri $\RR$ come una struttura nel linguaggio degli anelli orinati e con un simbolo per ogni $X\subseteq\RR$. Sia ${}^*\!X$ l'interpretazione di $X$ in ${}^*\!\RR$ una estensione elementare propria di $\RR$. Si dimostri che le seguenti affermazioni sono equivalenti per ogni $X\subseteq\RR$:\nobreak
\begin{itemize}
\item[1.]  $X$ is un aperto nell'usuale topologia di $\RR$;
\item[2.]  $b\approx a\in{}^*\!X \ \ \IMP\ \ b\in{}^*\!X$ for every $a$ standard e $b$ arbitrario.
\end{itemize}
\end{exercise}

\begin{exercise}
Con la stessa notazione dell'esercizio precedente. Si dimostri che le seguenti affermazioni sono equivalenti per ogni $X\subseteq\RR$\nobreak
\begin{itemize}
\item[1.]  $X$ is un chiuso nell'usuale topologia di $\RR$;
\item[2.]  $a\in{}^*\!X \ \ \IMP\ \ {\rm st}(a)\in{}^*\!X$ for every finito $a$.
\end{itemize}
\end{exercise}

\begin{exercise}
Let $N\models T_{\rm rg}$ prove that for every $\phi(x)\in L(N)$ the set $r(b,N)$ is a random graph.
Is every random graph $M\subseteq N$ of the form $\phi(N)$ for some $\phi(x)\in L(N)$~?
\end{exercise}

\clearpage%%%%%%%%%%%%%%%%%%%%%%%%%%%%%%%
\rhead{Cristian Alaimo, Beatrice Pitton}\setcounter{ex}{0}

\begin{exercise}
Si consideri $\RR$ come una struttura nel linguaggio degli anelli orinati e con un simbolo per ogni $X\subseteq\RR$. Sia ${}^*\!X$ l'interpretazione di $X$ in ${}^*\!\RR$ una estensione elementare propria di $\RR$.

Si dimostri che $\RR$ and $\0$ sono gli unici due sottoinsiemi $X\subseteq\RR$ tali che\nobreak
\begin{itemize}
\item[]  $b\approx a\in{}^*\!X \ \ \IMP\ \ b\in{}^*\!X$ for every coppia di iperreali $a, b$.
\end{itemize}
\end{exercise}

\begin{exercise}
Let $L$ be the language of strict orders expanded with countably many constants $\big\{c_i: i\in\omega\big\}$.
Let $T$ be the theory that extends $T_{\rm dlo}$ by the axioms $c_i<c_{i+1}$ for all $i$.
Prove that $T$ is complete.
\end{exercise} 

\begin{exercise}
Find 3 non isomorphic models of the theory $T$ in the exercise above.
\end{exercise}

\end{document}


