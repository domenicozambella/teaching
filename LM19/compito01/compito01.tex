\documentclass[10pt]{article}
\usepackage[utf8]{inputenc}
\usepackage[a4paper,height=24cm,width=14cm]{geometry}
\usepackage[italian]{babel}
\usepackage{amssymb}
\usepackage{dsfont}
\usepackage{calc}
\usepackage{graphicx}
\usepackage{pstricks}
\usepackage{pst-node}
\usepackage{fourier}
\usepackage{euscript}
\usepackage{amsmath,amssymb, amsthm}

\def\lh{\textrm{lh}}
\def\phi{\varphi}
\def\P{\EuScript P}
\def\M{\EuScript M}
\def\D{\EuScript D}
\def\U{\EuScript U}
\def\S{\EuScript S}
\def\sm{\smallsetminus}
\def\niff{\nleftrightarrow}
\def\proves{\vdash}
\def\ZZ{\mathds Z}
\def\NN{\mathds N}
\def\QQ{\mathds Q}
\def\RR{\mathds R}
\def\<{\langle}
\def\>{\rangle}
\def\E{\exists}
\def\A{\forall}
\def\0{\varnothing}
\def\imp{\rightarrow}
\def\iff{\leftrightarrow}
\def\IMP{\Rightarrow}
\def\IFF{\Leftrightarrow}
\def\range{\textrm{im}}
\def\Mod{\textrm{Mod}}
\def\Aut{\textrm{Aut}}
\def\Th{\textrm{Th}}
\def\acl{\textrm{acl}}
\def\eq{{\rm eq}}
\def\tp{\textrm{tp}}
\def\equivL{\stackrel{\smash{\scalebox{.5}{\rm L}}}{\equiv}}

\newcommand{\labella}[1]{{\sf\footnotesize #1}\hfill}
\renewenvironment{itemize}
  {\begin{list}{$\triangleright$}{%
   \setlength{\parskip}{0mm}
   \setlength{\topsep}{0mm}
   \setlength{\rightmargin}{0mm}
   \setlength{\listparindent}{0mm}
   \setlength{\itemindent}{0mm}
   \setlength{\labelwidth}{3ex}
   \setlength{\itemsep}{0mm}
   \setlength{\parsep}{0mm}
   \setlength{\partopsep}{0mm}
   \setlength{\labelsep}{1ex}
   \setlength{\leftmargin}{\labelwidth+\labelsep}
   \let\makelabel\labella}}{%
   \end{list}}
%\def\ssf#1{\textsf{\small #1}}
\newcounter{ex}
\newenvironment{exercise}{\bigskip\addtocounter{ex}{1}\textbf{Esercizio \theex.\quad}}{}
\pagestyle{empty}
\parindent0ex
\parskip2ex
\raggedbottom
\def\nsR{{}^*\!\RR}
\def\ssf#1{\textsf{#1}}
\renewcommand{\baselinestretch}{1.3}


\usepackage{fancyhdr}
\pagestyle{fancy}
\lhead{Logica Matematica a.a.~2018/19}
\rhead{}
\cfoot{}
\rfoot{\rput(1.5,-0.5){\small\thepage}}

\begin{document}


\clearpage%%%%%%%%%%%%%%%%%%%%%%%%%%%%%%%
\rhead{Cristian Alaimo, Davide Manca}\setcounter{ex}{0}



\begin{exercise}
Assume $L$ is countable and both $M\preceq N$ have arbitrary (large) cardinality. Let $A\subseteq N$ be a countable set. Adapt the construction used to prove the downward L\"owenheim-Skolem Theorem to prove there is a countable model $K$ such that $A\subseteq K\preceq N$ and $K\cap M\preceq N$ (in particular, $K\cap M$ is a model).  
\end{exercise}


\begin{exercise}
Consider $\RR$ in the language of strict orders. Prove that $\RR\sm\{0\}\preceq \RR$. (You may use the downward L\"owenheim-Skolem Theorem.) \ Are these two structures isomorphic?
\end{exercise}


\begin{exercise}
Let $M\preceq N$ and let $\phi(x)\in L(M)$.
Prove that $\phi(M)$ is finite if and only if $\phi(N)$ is finite and in this case $\phi(N)=\phi(M)$.
\end{exercise}

\clearpage%%%%%%%%%%%%%%%%%%%%%%%%%%%%%%%
\rhead{Marco Mombelli, Gabriele Pavanello}\setcounter{ex}{0}


\begin{exercise}
  Assume $L$ is countable and both $M\preceq N$ have arbitrary (large) cardinality. Let $A\subseteq N$ be a countable set. Apply the downward L\"owenheim-Skolem Theorem to prove there is a countable model $K$ such that $A\subseteq K\preceq N$ and $K\cap M\preceq N$ (in particular, $K\cap M$ is a model). Hint: contruct an elementary chain and apply the elementary chain lemma.
\end{exercise}

\begin{exercise}
  Let $M\preceq N$ and let $\phi(x,z)\in L$.
  Suppose there are finitely many sets of the form $\phi(a,N)$ for some $a\in N^{|x|}$.
  Prove that all these sets are definable over $M$.
\end{exercise}

\begin{exercise}
  Consider $\ZZ^n$ as a structure in the additive language of groups with the natural interpretation.
  Prove that $\ZZ^n\not\equiv\ZZ^m$ for every positive integers $n\neq m$.
\end{exercise}

\clearpage%%%%%%%%%%%%%%%%%%%%%%%%%%%%%%%
\rhead{Beatrice Pitton, Simone Ramello}\setcounter{ex}{0}

\begin{exercise}
  Prove that if $T$ has exactly $2$ maximally consistent extension $T_1$ and $T_2$ then there is a sentence $\phi$ such that $T,\phi\proves T_1$ and $T,\neg\phi\proves T_2$.
  State and prove the generalization to finitely many maximally consistent extensions (this seems easy to prove, but difficult to write down well).
\end{exercise}

\begin{exercise}
  Prove that the following are equivalent for every consistent theory $T$
  \begin{itemize}
    \item[1.] $T$ is complete;
    \item[2.] if $M,N\models T$ then $M\equiv N$.
  \end{itemize} 
\end{exercise} 

\begin{exercise}
  Let $L$ contain the symbols $0,1,\,\cdot\,$ with the usual interpretation. Prove that $\QQ\not\preceq\RR$. N.B. this contrasts with the (non obvious) fact that $\QQ\preceq\RR$ in the language $0,1,+$.
\end{exercise}


\end{document}


