\documentclass[10pt]{article}
\usepackage[utf8]{inputenc}
\usepackage[a4paper,height=24cm,width=14cm]{geometry}
\usepackage[italian]{babel}
\usepackage{amssymb}
\usepackage{dsfont}
\usepackage{calc}
\usepackage{graphicx}
\usepackage{pstricks}
\usepackage{pst-node}
\usepackage{fourier}
\usepackage{euscript}
\usepackage{amsmath,amssymb, amsthm}

\def\lh{\textrm{lh}}
\def\phi{\varphi}
\def\P{\EuScript P}
\def\M{\EuScript M}
\def\D{\EuScript D}
\def\U{\EuScript U}
\def\S{\EuScript S}
\def\sm{\smallsetminus}
\def\niff{\nleftrightarrow}
\def\proves{\vdash}
\def\ZZ{\mathds Z}
\def\NN{\mathds N}
\def\QQ{\mathds Q}
\def\RR{\mathds R}
\def\<{\langle}
\def\>{\rangle}
\def\E{\exists}
\def\A{\forall}
\def\0{\varnothing}
\def\imp{\rightarrow}
\def\iff{\leftrightarrow}
\def\IMP{\Rightarrow}
\def\IFF{\Leftrightarrow}
\def\range{\textrm{im}}
\def\Mod{\textrm{Mod}}
\def\Aut{\textrm{Aut}}
\def\Th{\textrm{Th}}
\def\acl{\textrm{acl}}
\def\eq{{\rm eq}}
\def\tp{\textrm{tp}}
\def\equivL{\stackrel{\smash{\scalebox{.5}{\rm L}}}{\equiv}}

\newcommand{\labella}[1]{{\sf\footnotesize #1}\hfill}
\renewenvironment{itemize}
  {\begin{list}{$\triangleright$}{%
   \setlength{\parskip}{0mm}
   \setlength{\topsep}{0mm}
   \setlength{\rightmargin}{0mm}
   \setlength{\listparindent}{0mm}
   \setlength{\itemindent}{0mm}
   \setlength{\labelwidth}{3ex}
   \setlength{\itemsep}{0mm}
   \setlength{\parsep}{0mm}
   \setlength{\partopsep}{0mm}
   \setlength{\labelsep}{1ex}
   \setlength{\leftmargin}{\labelwidth+\labelsep}
   \let\makelabel\labella}}{%
   \end{list}}
%\def\ssf#1{\textsf{\small #1}}
\newcounter{ex}
\newenvironment{exercise}{\bigskip\addtocounter{ex}{1}\textbf{Esercizio \theex.\quad}}{}
\pagestyle{empty}
\parindent0ex
\parskip2ex
\raggedbottom
\def\nsR{{}^*\!\RR}
\def\ssf#1{\textsf{#1}}
\renewcommand{\baselinestretch}{1.3}


\usepackage{fancyhdr}
\pagestyle{fancy}
\lhead{Logica Matematica a.a.~2018/19}
\rhead{}
\cfoot{}
\rfoot{\rput(1.5,-0.5){\small\thepage}}

\begin{document}


\clearpage%%%%%%%%%%%%%%%%%%%%%%%%%%%%%%%
\rhead{}\setcounter{ex}{0}

\begin{exercise}
The language contains only the binary relations $<$ and $e$.
The theory $T_0$ says that $<$ is a strict linear order and that $e$ is an equivalence relation.
Let $\M$ consists of models of $T_0$ and partial isomorphisms.
Do rich models exist? Can we axiomatize their theory? If so, does it have elimination of quantifiers? Is it $\lambda$-categorical for some $\lambda$?
\end{exercise}



\begin{exercise}
Prove that every model of $T_{\rm dag}$ is $\omega$-ultraomogeneous (indipendenlty of cardinality and rank). Does the same holds for models of $T_{\rm acf}$~?
\end{exercise}


\begin{exercise}
Let $\phi({\mr x})\in L$. Prove that the following are equivalent
\begin{itemize}
 \item[1.] $\phi({\mr x})$ is equivalent to some $\psi({\mr x})\in L_{\rm qf}$;
 \item[2.] $\phi({\mr a})\iff \phi(f{\mr a})$ for every partial isomorphism $f:\U\to\U$ defined in ${\mr a}$.
\end{itemize}
Use the result to prove that if $T$ complete and partial isomorphisms between models of $T$ preserve the truth of all formulas then $T$ has elimination of quantifiers.\QED
\end{exercise}


\begin{exercise}
Let $\phi(x,y)\in L(\U)$. Prove that if the set $\big\{\phi(a,\U)\ :\ a\in\U^{|x|}\big\}$ is infinite then it has cardinality $\kappa$.
Does the claim remain true with a type $p(x,y)\subseteq L(A)$ for $\phi(x,y)$?
\end{exercise}

\end{document}


