\documentclass[10pt]{article}
\usepackage[utf8]{inputenc}
\usepackage[a4paper,height=24cm,width=13cm]{geometry}
\usepackage[italian]{babel}
\usepackage{amssymb}
\usepackage{dsfont}
\usepackage{calc}
\usepackage{graphicx}
\usepackage{pstricks}
\usepackage{pst-node}
\usepackage{fourier}
\usepackage{euscript}
\usepackage{amsmath,amssymb, amsthm}

\def\lh{\textrm{lh}}
\def\phi{\varphi}
\def\P{\EuScript P}
\def\M{\EuScript M}
\def\D{\EuScript D}
\def\U{\EuScript U}
\def\S{\EuScript S}
\def\sm{\smallsetminus}
\def\niff{\nleftrightarrow}
\def\ZZ{\mathds Z}
\def\NN{\mathds N}
\def\PP{\mathds P}
\def\QQ{\mathds Q}
\def\RR{\mathds R}
\def\<{\langle}
\def\>{\rangle}
\def\E{\exists}
\def\A{\forall}
\def\0{\varnothing}
\def\imp{\rightarrow}
\def\iff{\leftrightarrow}
\def\IMP{\Rightarrow}
\def\IFF{\Leftrightarrow}
\def\range{\textrm{rng}}
\def\Mod{\textrm{Mod}}
\def\Aut{\textrm{Aut}}
\def\Th{\textrm{Th}}
\def\acl{\textrm{acl}}
\def\eq{{\rm eq}}
\def\tp{\textrm{tp}}
\def\equivL{\stackrel{\smash{\scalebox{.5}{\rm L}}}{\equiv}}
\def\swedge{\mathbin{\raisebox{.2ex}{\tiny$\mathbin\wedge$}}}
\def\svee{\mathbin{\raisebox{.2ex}{\tiny$\mathbin\vee$}}}

\newcommand{\labella}[1]{{\sf\footnotesize #1}\hfill}
\renewenvironment{itemize}
  {\begin{list}{$\triangleright$}{%
   \setlength{\parskip}{0mm}
   \setlength{\topsep}{0mm}
   \setlength{\rightmargin}{0mm}
   \setlength{\listparindent}{0mm}
   \setlength{\itemindent}{0mm}
   \setlength{\labelwidth}{3ex}
   \setlength{\itemsep}{0mm}
   \setlength{\parsep}{0mm}
   \setlength{\partopsep}{0mm}
   \setlength{\labelsep}{1ex}
   \setlength{\leftmargin}{\labelwidth+\labelsep}
   \let\makelabel\labella}}{%
   \end{list}}
%\def\ssf#1{\textsf{\small #1}}
\newcounter{ex}
\newenvironment{exercise}{\clearpage\addtocounter{ex}{1}\textbf{Esercizio \theex.\quad}}{}
\pagestyle{empty}
\parindent0ex
\parskip2ex
\raggedbottom
\def\nsR{{}^*\!\RR}
\def\ssf#1{\textsf{#1}}
\renewcommand{\baselinestretch}{1.3}


\usepackage{fancyhdr}
\pagestyle{fancy}
\lhead{Teoria dei Modelli a.a.~2018/19}
\rhead{}
\cfoot{}
\rfoot{\rput(1.5,-0.5){\small\thepage}}

\begin{document}


Let $\D$ be a definable set.
%
Let $n$ be the minimal integer such that $\D=\phi(\U\,;a)$ for some $\phi(x\,;z)\in L$ and $a\in\U^n$.
%
We claim that, if $c$ is any other tuple such that $\D=\phi(\U\,;c)$, then $\range\, a=\range\, c$. 

Given $a$ and $c$ as above, assume for a contradiction that  $\range\, a\neq\range\, c$.

Let $c=\<c_0,\dots,c_{n-1}\>$ and $a=\<a_0,\dots,a_{n-1}\>$.

Let $I\subseteq n$ be the set of those $i<n$ such that $a_i\in\range\,c$.

Let $J\subseteq n$ be the set of those $j<n$ such that $c_j\in\range\,a$.

Clearly $k:=|J|=|I|<n$ and $b:=a_{\restriction I}=c_{\restriction J}$.

Let $u$ and $v$ be two distinct tuples of variables of length $k$. Write $(au)$ and $(cv)$ for tuples of length $n$ comprising parameters and variables such that $u=(au)_{\restriction I}$ and $v=(cv)_{\restriction J}$ and $a_{\restriction n\sm I}=(au)_{\restriction n\sm I}$ and $c_{\restriction n\sm J}=(cv)_{\restriction n\sm J}$.






$\A u,v\ \A x\ [\phi(x\,;au)\iff\phi(x\,;cv)]$




Let $y$ be a tuple of length $k$
\end{document}


