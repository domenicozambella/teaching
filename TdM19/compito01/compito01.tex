\documentclass[10pt]{article}
\usepackage[utf8]{inputenc}
\usepackage[a4paper,height=24cm,width=13cm]{geometry}
\usepackage[italian]{babel}
\usepackage{amssymb}
\usepackage{dsfont}
\usepackage{calc}
\usepackage{graphicx}
\usepackage{pstricks}
\usepackage{pst-node}
\usepackage{fourier}
\usepackage{euscript}
\usepackage{amsmath,amssymb, amsthm}

\def\lh{\textrm{lh}}
\def\phi{\varphi}
\def\P{\EuScript P}
\def\M{\EuScript M}
\def\D{\EuScript D}
\def\U{\EuScript U}
\def\S{\EuScript S}
\def\sm{\smallsetminus}
\def\niff{\nleftrightarrow}
\def\ZZ{\mathds Z}
\def\NN{\mathds N}
\def\PP{\mathds P}
\def\QQ{\mathds Q}
\def\RR{\mathds R}
\def\<{\langle}
\def\>{\rangle}
\def\E{\exists}
\def\A{\forall}
\def\0{\varnothing}
\def\imp{\rightarrow}
\def\iff{\leftrightarrow}
\def\IMP{\Rightarrow}
\def\IFF{\Leftrightarrow}
\def\range{\textrm{im}}
\def\Mod{\textrm{Mod}}
\def\Aut{\textrm{Aut}}
\def\Th{\textrm{Th}}
\def\acl{\textrm{acl}}
\def\eq{{\rm eq}}
\def\tp{\textrm{tp}}
\def\equivL{\stackrel{\smash{\scalebox{.5}{\rm L}}}{\equiv}}
\def\swedge{\mathbin{\raisebox{.2ex}{\tiny$\mathbin\wedge$}}}
\def\svee{\mathbin{\raisebox{.2ex}{\tiny$\mathbin\vee$}}}

\newcommand{\labella}[1]{{\sf\footnotesize #1}\hfill}
\renewenvironment{itemize}
  {\begin{list}{$\triangleright$}{%
   \setlength{\parskip}{0mm}
   \setlength{\topsep}{0mm}
   \setlength{\rightmargin}{0mm}
   \setlength{\listparindent}{0mm}
   \setlength{\itemindent}{0mm}
   \setlength{\labelwidth}{3ex}
   \setlength{\itemsep}{0mm}
   \setlength{\parsep}{0mm}
   \setlength{\partopsep}{0mm}
   \setlength{\labelsep}{1ex}
   \setlength{\leftmargin}{\labelwidth+\labelsep}
   \let\makelabel\labella}}{%
   \end{list}}
%\def\ssf#1{\textsf{\small #1}}
\newcounter{ex}
\newenvironment{exercise}{\clearpage\addtocounter{ex}{1}\textbf{Esercizio \theex.\quad}}{}
\pagestyle{empty}
\parindent0ex
\parskip2ex
\raggedbottom
\def\nsR{{}^*\!\RR}
\def\ssf#1{\textsf{#1}}
\renewcommand{\baselinestretch}{1.3}


\usepackage{fancyhdr}
\pagestyle{fancy}
\lhead{Logica Matematica a.a.~2016/17}
\rhead{}
\cfoot{}
\rfoot{\rput(1.5,-0.5){\small\thepage}}

\begin{document}

\clearpage%%%%%%%%%%%%%%%%%%%%%%%%%%%%%%%
\rhead{\hfill Moreno Prieobon e Michele Bailetti}\setcounter{ex}{0}

\begin{exercise}
Let $T$ be a consistent theory. Suppose that all completions of $T$ are of the form 
$T\cup S$ for some set $S$ of quantifier-free sentences. Prove (in the  most direct possible way) that, if all completion of $T$ have elimination of quantifiers, so does $T$. Show that this fails when the completions of $T$ have arbitrary complexity. 
\end{exercise}


\begin{exercise}
For every $a\in\U^{|u|}$ and $A\subseteq\U$, the following are equivalent
\begin{itemize}
\item[1.] $a$ is solution of some algebraic formula  $\phi(u)\in L(A)$;
\item[2.] $a= a_1,\dots, a_n$ for some $ a_1,\dots, a_n\in\acl(A)$. 
\end{itemize}
\end{exercise}

\begin{exercise}\label{pofu}
Let $\phi( z)\in L(A)$ be a consistent formula. Prove that, if $a\in\acl(A, b)$ for every $b\models\phi(z)$, then $a\in\acl(A)$. Prove the same claim with a type $p(z)\subseteq L(A)$ for $\phi(z)$.
\end{exercise}


\clearpage%%%%%%%%%%%%%%%%%%%%%%%%%%%%%%%
\rhead{\hfill  Tutti $\sm\{\}$}\setcounter{ex}{0}

\begin{exercise}
Dimostrare che se $N$ è (elementarmente) saturo allora è (elementarmente) omogeneo.

La dimostrazione usa la tecnica dell'andirivieni (back-and-forth). Nelle note è esposta nel caso generale. Si provi a fare una dimostrazione nel caso specifico.
\end{exercise}


\begin{exercise}
Let $\phi(x\,;z)\in L$. Prove that if the set $\big\{\phi(a\,;\U)\ :\ a\in\U^{|x|}\big\}$ is infinite then it has cardinality $\kappa$. Does the claim remains true with a type $p(x\,;z)\subseteq L$ for $\phi(x\,;z)$?

Suggerimento per la seconda domanda: potrebbe esserci un controesempio in $\U\equiv\NN$ nel linguaggio degli ordini.
\end{exercise}


\begin{exercise}
prove that for every $a\in\U^{|u|}$ and $A\subseteq\U$, the following are equivalent
\begin{itemize}
\item[1.] $a$ is solution of some algebraic formula  $\phi(u)\in L(A)$;
\item[2.] $a= a_1,\dots, a_n$ for some $ a_1,\dots, a_n\in\acl(A)$. 
\end{itemize}
\end{exercise}



\end{document}


