\documentclass[10pt]{article}
\usepackage[utf8]{inputenc}
\usepackage[a4paper,height=24cm,width=13cm]{geometry}
\usepackage{amssymb}
\usepackage{dsfont}
\usepackage{calc}
\usepackage{graphicx}
% \usepackage{pst-node}
\usepackage{fourier-otf}
\usepackage{euscript}
\usepackage{amsmath,amsthm}
\def\lh{\textrm{lh}}
\def\P{\EuScript P}
\def\M{\EuScript M}
\def\D{\EuScript D}
\def\U{\EuScript U}
\def\S{\EuScript S}
\def\sm{\smallsetminus}
\def\niff{\nleftrightarrow}
\def\ZZ{\mathds Z}
\def\NN{\mathds N}
\def\PP{\mathds P}
\def\QQ{\mathds Q}
\def\RR{\mathds R}
\def\<{\langle}
\def\>{\rangle}
\def\E{\exists}
\def\A{\forall}
\def\0{\varnothing}
\def\imp{\rightarrow}
\def\iff{\leftrightarrow}
\def\IMP{\Rightarrow}
\def\IFF{\Leftrightarrow}
\def\range{\textrm{im}}
\def\Mod{\textrm{Mod}}
\def\Aut{\textrm{Aut}}
\def\Th{\textrm{Th}}
\def\acl{\textrm{acl}}
\def\eq{{\rm eq}}
\def\tp{\textrm{tp}}
\def\equivL{\stackrel{\smash{\scalebox{.5}{\rm L}}}{\equiv}}

\def\cnonfork{\mathbin{\raise1.8ex\rlap{\kern0.6ex\rule{0.6ex}{0.1ex}}\rlap{\kern1.1ex\rule{0.1ex}{1.9ex}}\raise-0.3ex\hbox{$\smile$} } }

\newcommand{\labella}[1]{{\sf\footnotesize #1}\hfill}
\renewenvironment{itemize}
  {\begin{list}{$\triangleright$}{%
   \setlength{\parskip}{0mm}
   \setlength{\topsep}{0mm}
   \setlength{\rightmargin}{0mm}
   \setlength{\listparindent}{0mm}
   \setlength{\itemindent}{0mm}
   \setlength{\labelwidth}{3ex}
   \setlength{\itemsep}{0mm}
   \setlength{\parsep}{0mm}
   \setlength{\partopsep}{0mm}
   \setlength{\labelsep}{1ex}
   \setlength{\leftmargin}{\labelwidth+\labelsep}
   \let\makelabel\labella}}{%
  %  \vspace*{-.5\baselineskip}
  \end{list}}
%\def\ssf#1{\textsf{\small #1}}
\newcounter{ex}
\newenvironment{exercise}{\bigskip\addtocounter{ex}{1}\textbf{Esercizio \theex.\ }}{}
\pagestyle{empty}
\parindent0ex
\parskip2ex
\raggedbottom
\def\nsR{{}^*\!\RR}
\def\ssf#1{\textsf{#1}}
\renewcommand{\baselinestretch}{1.3}


\usepackage{fancyhdr}
\pagestyle{fancy}
\lfoot{Teoria dei Modelli a.a.~2023/24}
\rhead{}
\rfoot{{\small\thepage}}
\cfoot{}



%
% INDICARE NOME E COGNOME DI TUTTI GLI AUTORI
%
\lhead{Nome/i Cognome/i}
%

\begin{document}

\begin{exercise}
  Prove that the equivalence relation $a\stackrel{\smash{\scalebox{.5}{\rm L}}}{\equiv}_Ab$ is the transitive closure of the relation: there is a sequence $\<c_i:i<\omega\>$ indiscernible over $A$ such that $c_0=a$ and $c_1=b$.

  Si ragioni come nella proposizione 16.19 e nel teorema 16.8.
\end{exercise}


\begin{exercise}
  ($T$ stable) \  Prove that the following are equivalent for every $p(x)\in S(\U)$
  \begin{itemize}
    \item[1.] $p(x)$ is finitely satisfiable in $M$;
    \item[2.] $p(x)$ is invariant over $M$.
  \end{itemize}
\end{exercise}


\begin{exercise}
  ($T$ stable) \ 
  Prove that $a\cnonfork_M b$ if and only if $b\cnonfork_M a$.
\end{exercise}

\hfill\hrulefill\hfill\hfill

\begin{exercise}
  Let $\varphi(x,y)\in  L$, where $|x|=|y|=1$.
  Suppose there is an infinite set $A\subseteq\U$ such that $\varphi(a,b)\niff\varphi(b,a)$ for every two distinct $a,b\in A$.
  Prove that $\varphi(x\,;y)$ is unstable.
\end{exercise}

\begin{exercise}
  ($T$ stable) \ 
  Prove that the following are equivalent
  \begin{itemize}
    \item[1.] $\varphi(x\,;z)$ is stable;
    \item[2.] for every $M$ and $a\in\U^{|x|}$ there is a formula $\psi(z)\in L(M)$ such that $\varphi(a\,;\U)=_M\psi(\U)$.
  \end{itemize} 
\end{exercise}

\begin{exercise}
  Prove that if every formula $\varphi(x\,;z)\in L$ with $|x|=1$ is stable then $T$ is stable. 
  
  Si dimostri che tutte le formule con parametri sono stabili, poi si usino gli indiscernibili (ma altre vie sono possibili). 
\end{exercise}



\end{document}


