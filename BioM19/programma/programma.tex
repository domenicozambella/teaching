\documentclass[10pt]{article}
\usepackage[a4paper,hmargin={2cm,7cm},vmargin={3cm,3cm}]{geometry}
\usepackage[utf8]{inputenc}
\usepackage[italian]{babel}
\usepackage{amssymb}
\usepackage[T1]{fontenc}
\usepackage[colorlinks=true,bookmarksopen=false,linkcolor=blue,citecolor=red]{hyperref}
\usepackage{calc} 
\usepackage[absolute, overlay]{textpos}
\usepackage{palatino} 
\usepackage{xcolor}
\newcommand{\mylabel}[1]{#1\hfill}
\renewenvironment{itemize}
  {\begin{list}{$\triangleright$}{%
   \baselineskip3ex
   \setlength{\parskip}{4mm}
   \setlength{\topsep}{.4\baselineskip}
   \setlength{\rightmargin}{0mm}
   \setlength{\listparindent}{0mm}
   \setlength{\itemindent}{0mm}
   \setlength{\labelwidth}{2ex}
   \setlength{\itemsep}{.4\baselineskip}
   \setlength{\parsep}{0mm}
   \setlength{\partopsep}{0mm}
   \setlength{\labelsep}{1ex}
   \setlength{\leftmargin}{\labelwidth+\labelsep}
   \let\makelabel\mylabel}}{%
   \end{list}\vspace*{-1.3mm}}

\newenvironment{sitemize}
  {\begin{list}{$\triangleright$}{%
   \setlength{\parskip}{0mm}
   \setlength{\topsep}{2mm}
   \setlength{\rightmargin}{0mm}
   \setlength{\listparindent}{0mm}
   \setlength{\itemindent}{0mm}
   \setlength{\labelwidth}{2ex}
   \setlength{\itemsep}{.4\baselineskip}
   \setlength{\parsep}{0mm}
   \setlength{\partopsep}{0mm}
   \setlength{\labelsep}{1ex}
   \setlength{\leftmargin}{\labelwidth+\labelsep}
   \let\makelabel\mylabel}}{%
   \end{list}}
\parskip2mm
\parindent0mm


\begin{document}

\section{Programma a.a. 2018/19}

Bozza. Scaricare l'ultima versione da\smallskip

\href{https://github.com/domenicozambella/teaching/raw/master/BioM19/programma/programma.pdf}{\small github.com/domenicozambella/teaching/raw/master/BioM19/programma/programma.pdf}



\subsection{Calcolo per funzioni univariate (\boldmath$\lesssim$ 20 ore)}

\colorbox{blue!10}{\begin{minipage}{\textwidth}
Ridotta al minimo e mirata soprattutto a una conoscenza passiva (riconoscere formalismo). 
\end{minipage}}



\begin{itemize}
\item Le funzioni $e^x$, $\ln x$, le funzioni trigometriche

\item Trasformazioni e simmetrie di funzioni

\item Derivate; il significato geometrico; approssimazioni lineari

      Le derivate di $x^n$ , $e^x$ , $\ln x$, $\sin x$, $\cos x$\hfill\rlap{Solo enunciati}

      Le regole per il calcolo delle derivate\hfill\rlap{Solo conoscenza passiva?}
      
      Calcolo dei punti critici

      Derivate parziali\hfill\rlap{Solo definizione}
      
\item Le primitive (antiderivate)

Regola di integrazione per sostituzione.\hfill\rlap{Unica regola di calcolo}

Integrali definiti; significato geometrico

Integrali impropri

\item Equazioni differenziali \hfill\rlap{\parbox{30ex}{Solo primo ordine?} }

  Soluzione generale e soluzione particolare. Condizioni iniziali
  
  Discussione qualitativa di equilibrio e stabilità
  
  Esempi:\hfill\rlap{\parbox{30ex}{Altri esempi rilevanti?}} 
  
  \noindent\kern7ex Newton's law of cooling $T' = -r(T-T_0)$
  
  \noindent\kern7ex Equazione logistica (con raccolto) $y' = r\, y\,\big(1- y/K\big) + b$
  
  \noindent\kern7ex Equazione di Gompertz $y' = r\, y\,\ln(y/K)$
\end{itemize}

\clearpage
\subsection{Matematica discreta e algebra lineare (\boldmath$\lesssim$ 20 ore) }
\colorbox{blue!10}{\begin{minipage}{\textwidth}
Ridotta al minimo e mirata soprattutto a una conoscenza passiva.\medskip 

La speranza è di poter menzionare un po di statistica multivariata (PCA?) ma non già nel 2018/19, serve altra sperimentazione.
\end{minipage}}
\begin{itemize}

\item Successioni e serie \hfill\rlap{Solo serie geometrica}

Il limite di $(1+1/n)^n$ per $n\to\infty$\hfill\rlap{Solo enunciato}

\item L'equazione ricorsiva $x_{n+1} = a\, x_n+b$\hfill\rlap{Solo questa}

  Esempi:\hfill\rlap{Altri suggerimenti?}

  \noindent\kern7ex Serie geometrica

  \noindent\kern7ex Crescita Mathusiana in tempo discreto (con/senza raccolto)
  
  \noindent\kern7ex Concentrazione medicinale (con somministrazioni periodiche)
  
  \noindent\kern7ex Evoluzione del debito per un mutuo bancario
  
  \noindent\kern7ex \ldots\hfill\rlap{\parbox{30ex}{Altro?}} 

\item Vettori, indipendenza e basi\hfill\rlap{Solo intuizione}

  Matrici, le trasformazioni lineari, sistemi lineari
  
  Moltiplicazione matriciale\hfill\rlap{No determinante}
  
  Matrice inversa\hfill\rlap{No calcolo matrice inversa}
  
  Autovalori e autovettori, significato\hfill\rlap{No polinomio caratteristico}
   
  Esempi:
  
  \noindent\kern7ex Evoluzione lineare in tempo discreto: $\vec {x}_{n+1} = A\,\vec {x}_n +  \vec b$\hfill\rlap{Irrilevante?}
  
  \noindent\kern7ex Studio dei punti di equibrio e del comportamento asintotico
  
  \noindent\kern7ex Modellini concreti di catene Markov (senza nominarle)
  
  \noindent\kern7ex Modellini demografici (tipo matrice di Leslie)
  
  \noindent\kern7ex \ldots\hfill\rlap{\parbox{30ex}{Altro?}} 

\end{itemize}

\clearpage
\subsection{Probabilità e statistica (\boldmath$\gtrsim$ 24 ore) }
\colorbox{blue!10}{\begin{minipage}{\textwidth}
Se ho capito bene: 
\begin{sitemize}
\item Più distribuzioni discrete (e meno continue)
\item Comprensione critica del significato di test statistico
\item No approccio procedurale (che tanto si dimentica)
\item No approccio numerico/informatico (che già devono imparare {\tt Perl})
\end{sitemize}
\end{minipage}}


\begin{itemize}
\item Spazi di probabilità; probabilità condizionata e indipendenza

\item Teorema delle probabilità totali; regola di Bayes

Esempi:
    
    \noindent\kern7ex Equilibrio di Hardy-Weinberg
    
    \noindent\kern7ex Specificità, sensibilità, falsi positivi/negativi, valore predittivo
    
\item Variabili aleatorie discrete; valore atteso e varianza

Distribuzioni:
  
  \noindent\kern7ex Binomiale
  
  \noindent\kern7ex Geometrica
  
  \noindent\kern7ex Binomiale negativa
  
  \noindent\kern7ex Poisson

  \noindent\kern7ex \ldots\hfill\rlap{\parbox{30ex}{Altre?}} 
  
\item Variabili aleatorie continue

Distribuzioni:
  
  \noindent\kern7ex Normale
  
  \noindent\kern7ex Student
  
  \noindent\kern7ex Esponenziale

  \noindent\kern7ex \ldots\hfill\rlap{\parbox{30ex}{Altro?}} 
  
\item Test di ipotesi, significatività, p-valore, potenza, effect-size

Test:
  
  \noindent\kern7ex Test binomiale per proporzioni
  
  \noindent\kern7ex Z-test per la media
  
  \noindent\kern7ex T-test per la media

  \noindent\kern7ex \ldots\hfill\rlap{\parbox{30ex}{Altro?}} 
  
\item Intervallo di confidenza, discussione dell'interpretazione

Esempi:
  
  \noindent\kern7ex Clopper–Pearson (exact) interval for a proportion
  
  \noindent\kern7ex Intervallo di confidenza per la media di una normale con $\sigma$ nota
  
  \noindent\kern7ex Intervallo di confidenza per la media di una normale con $\sigma$ ignota
  
  \noindent\kern7ex Intervallo di confidenza per la media di una Poisson

  \noindent\kern7ex \ldots\hfill\rlap{\parbox{30ex}{Altro?}} 
  
  
\item Problema dei test multipli (FWER)\hfill\rlap{\parbox{30ex}{Anche FDR?}} 

Esempi:
  
  \noindent\kern7ex Sidak correction

  \noindent\kern7ex Bonferroni correction
  
  \noindent\kern7ex \ldots\hfill\rlap{\parbox{30ex}{Benjamini-Hochberg?\\ q-valore?}} 

\end{itemize}


\clearpage
\subsection{Applicazioni informatiche (\boldmath$\approx$ 0 ore)}
\baselineskip3ex

Agli studenti \textit{non\/} verrà richiesta alcuna competenza di programmazione né per seguire il corso né per superare l'esame. 

Gli studenti verranno però esposti (in modo blando e passivo) a strumenti per esplorazione di dati e la presentazione di ricerca riproducibile.

Conto di usare Jupyter Notebook/Lab come ausilio didattico. Mi avvarrò di una piattaforma pubblica (non è elegante, ma è solo un esperimento). Al momento sto valutando \href{https://notebooks.azure.com/help/jupyter-notebooks}{Microsoft Azure} e \href{https://colab.research.google.com}{Google Colaboratory}.

Userò {\tt Python\/} perché offre un ecosistema più completo di {\tt R}.

Strumenti cui gli studenti verranno esposti: {\tt bqplot} e {\tt pandas} per la esplorazione dei dati, {\tt scipy.stat} per le funzioni statistiche. Valuterò quanto fare di calcolo matriciale, nel caso, con {\tt Python} è possibile rimpiazzare {\tt Matlab}/{\tt Octave}.


\clearpage
\section{Criticità a.a.\@ 2017/18}


\subsection{Esercizi ed esempi d'esame}

Cricità: gli studenti non avevano materiale per esercitarsi. Io stesso avevo a disposizione un numero insufficiente di domande d'esame. Anche solo passare materiale all'esercitatore è stato diffcile.

Rimedio: ho iniziato a costruire una \href{https://github.com/domenicozambella/teaching/tree/master/Statistica}{repositoria pubblica\/} con esercizi risolti. Già dal prossimo anno questa offrirà una rappresentazione sufficientemente fedele delle domande d'esame. 


\subsection{Testi}

Cricità: manca un testo/manuale di riferimento.

Rimedio: l'unico rimedio efficace sarebbero delle dispense. Escludo di riuscire produrle nell'a.a.2018/19. Per la parte di statistica indirizzero gli studenti a una selezione (annotata) di paragrafi del primo e terzo capitolo di:

Ewens e Grant, \textit{Statistical methods in bioinformatics}. Springer (2005)

Per la parte di calcolo, un qualsiasi testo di liceo è sufficiente, per la parte di matematica discreta e algebra lineare il problema è ancora più difficile e sono ancora alla ricerca. 

\end{document}
