\documentclass[10pt]{article}
\usepackage[utf8]{inputenc}
\usepackage[a4paper,height=24cm,width=13cm]{geometry}
\usepackage{amssymb}
\usepackage{dsfont}
\usepackage{calc}
\usepackage{graphicx}
% \usepackage{pst-node}
\usepackage{fourier-otf}
\usepackage{euscript}
\usepackage{amsmath,amsthm}

\def\lh{\textrm{lh}}
% \def\varphi{\varphi}
\def\P{\EuScript P}
\def\M{\EuScript M}
\def\D{\EuScript D}
\def\U{\EuScript U}
\def\S{\EuScript S}
\def\sm{\smallsetminus}
\def\niff{\nleftrightarrow}
\def\ZZ{\mathds Z}
\def\NN{\mathds N}
\def\PP{\mathds P}
\def\QQ{\mathds Q}
\def\RR{\mathds R}
\def\<{\langle}
\def\>{\rangle}
\def\E{\exists}
\def\A{\forall}
\def\0{\varnothing}
\def\imp{\rightarrow}
\def\iff{\leftrightarrow}
\def\IMP{\Rightarrow}
\def\IFF{\Leftrightarrow}
\def\range{\textrm{im}}
\def\Mod{\textrm{Mod}}
\def\Aut{\textrm{Aut}}
\def\Th{\textrm{Th}}
\def\acl{\textrm{acl}}
\def\eq{{\rm eq}}
\def\tp{\textrm{tp}}
\def\equivL{\stackrel{\smash{\scalebox{.5}{\rm L}}}{\equiv}}
\def\swedge{\mathbin{\raisebox{.2ex}{\tiny$\mathbin\wedge$}}}
\def\svee{\mathbin{\raisebox{.2ex}{\tiny$\mathbin\vee$}}}

\newcommand{\labella}[1]{{\sf\footnotesize #1}\hfill}
\renewenvironment{itemize}
  {\begin{list}{$\triangleright$}{%
   \setlength{\parskip}{0mm}
   \setlength{\topsep}{0mm}
   \setlength{\rightmargin}{0mm}
   \setlength{\listparindent}{0mm}
   \setlength{\itemindent}{0mm}
   \setlength{\labelwidth}{3ex}
   \setlength{\itemsep}{0mm}
   \setlength{\parsep}{0mm}
   \setlength{\partopsep}{0mm}
   \setlength{\labelsep}{1ex}
   \setlength{\leftmargin}{\labelwidth+\labelsep}
   \let\makelabel\labella}}{%
   \end{list}}
%\def\ssf#1{\textsf{\small #1}}
\newcounter{ex}
\newenvironment{exercise}{\addtocounter{ex}{1}\textbf{Esercizio \theex.\quad}}{}
\pagestyle{empty}
\parindent0ex
\parskip2ex
\raggedbottom
\def\nsR{{}^*\!\RR}
\def\ssf#1{\textsf{#1}}
\renewcommand{\baselinestretch}{1.3}


\usepackage{fancyhdr}
\pagestyle{fancy}
\lfoot{Teoria dei Modelli a.a.~2022/23}
\rhead{}
\rfoot{{\small\thepage}}
\cfoot{}







%
% INDICARE NOME E COGNOME DI TUTTI GLI AUTORI
%
\lhead{Nome/i Cognome/i}
%
\begin{document}

\begin{exercise}
  Let $M$ be an $L$-structure and let $\psi(x), \varphi(x,y)\in L$. For each of the following conditions, write a sentence true in $M$ exactly when
  \begin{itemize}
  \item[a.] $\psi(M)\ \in\ \big\{\varphi(a,M): a\in M\big\}$;
  \item[b.] $\big\{\varphi(a,M): a\in M\big\}$ contains at least two sets;
  \item[c.] $\big\{\varphi(a,M): a\in M\big\}$ contains only sets that are pairwise disjoint.
  \end{itemize}
  Risposta secca, nessuna giustificazione.
\end{exercise}
  
\begin{exercise}
  Let $M$ be a structure in a signature that contains a symbol $r$ for a binary relation.
  Write a sentence $\varphi$ such that 
  \begin{itemize} 
  \item[a.] $M\models\varphi$ if and only if there is an $A\subseteq M$ such that $r^M\ \subseteq\ A\times\neg A$.
  \end{itemize}
  Risposta secca, nessuna giustificazione.
\end{exercise}


\begin{exercise}
  Let $M\preceq N$ and let $\varphi(x)\in L(M)$.
  Prove that $\varphi(M)$ is finite if and only if $\varphi(N)$ is finite and in this case $\varphi(N)=\varphi(M)$.
\end{exercise}
  
\begin{exercise}
  Let $M\preceq N$ and let $\varphi(x,z)\in L$.
  Suppose there are finitely many sets of the form $\varphi(a,N)$ for some $a\in N^{|x|}$.
  Prove that all these sets are definable over $M$.
\end{exercise}

\begin{exercise}
Let ${}^*\QQ$ be an elementary extension of $\QQ$ in a language that contains symbols for all relation and function of $\QQ$.
Prove that ${}^*\QQ$ contains infinite and infinitesimal language.
\end{exercise}

\end{document}


