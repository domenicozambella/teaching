\documentclass[10pt]{article}
\usepackage[utf8]{inputenc}
\usepackage[a4paper,height=24cm,width=13cm]{geometry}
\usepackage[]{hyperref}
%\usepackage{amssymb}
\usepackage{dsfont}
\usepackage{calc}
\usepackage{graphicx}
% \usepackage{pst-node}
\usepackage{fourier-otf}
\usepackage{euscript}
\usepackage{amsmath,amsthm}

\def\lh{\textrm{lh}}
% \def\varphi{\varphi}
\def\P{\EuScript P}
\def\M{\EuScript M}
\def\D{\EuScript D}
\def\U{\EuScript U}
\def\S{\EuScript S}
\def\sm{\smallsetminus}
\def\niff{\nleftrightarrow}
\def\ZZ{\mathds Z}
\def\NN{\mathds N}
\def\PP{\mathds P}
\def\QQ{\mathds Q}
\def\RR{\mathds R}
\def\<{\langle}
\def\>{\rangle}
\def\E{\exists}
\def\A{\forall}
\def\0{\varnothing}
\def\imp{\rightarrow}
\def\iff{\leftrightarrow}
\def\IMP{\Rightarrow}
\def\IFF{\Leftrightarrow}
\def\range{\textrm{im}}
\def\Mod{\textrm{Mod}}
\def\Aut{\textrm{Aut}}
\def\Th{\textrm{Th}}
\def\acl{\textrm{acl}}
\def\eq{{\rm eq}}
\def\tp{\textrm{tp}}
\def\equivL{\stackrel{\smash{\scalebox{.5}{\rm L}}}{\equiv}}
\def\swedge{\mathbin{\raisebox{.2ex}{\tiny$\mathbin\wedge$}}}
\def\svee{\mathbin{\raisebox{.2ex}{\tiny$\mathbin\vee$}}}

\newcommand{\labella}[1]{{\sf\footnotesize #1}\hfill}
\renewenvironment{itemize}
  {\begin{list}{$\triangleright$}{%
   \setlength{\parskip}{0mm}
   \setlength{\topsep}{0mm}
   \setlength{\rightmargin}{0mm}
   \setlength{\listparindent}{0mm}
   \setlength{\itemindent}{0mm}
   \setlength{\labelwidth}{3ex}
   \setlength{\itemsep}{0mm}
   \setlength{\parsep}{0mm}
   \setlength{\partopsep}{0mm}
   \setlength{\labelsep}{1ex}
   \setlength{\leftmargin}{\labelwidth+\labelsep}
   \let\makelabel\labella}}{%
   \vspace*{-.5\baselineskip}\end{list}}
%\def\ssf#1{\textsf{\small #1}}
\newcounter{ex}
\newenvironment{exercise}{\addtocounter{ex}{1}\textbf{Esercizio \theex.\quad}}{}
\pagestyle{empty}
\parindent0ex
\parskip2ex
\raggedbottom
\def\nsR{{}^*\!\RR}
\def\ssf#1{\textsf{#1}}
\renewcommand{\baselinestretch}{1.3}


\usepackage{fancyhdr}
\pagestyle{fancy}
\lfoot{Logica Matematica 2 a.a.~2022/23}
\rhead{}
\rfoot{{\small\thepage}}
\cfoot{}







%
% INDICARE NOME E COGNOME DI TUTTI GLI AUTORI
%
\lhead{Nome/i Cognome/i}
%
\begin{document}
Esercizi 1,2,3 per gli studenti della LM, esercizio 4 per gli studenti della LT.

\begin{exercise} 
  Let $A\subseteq B$ and $p(x)\subseteq L(A)$. 
  Suppose that $\tp(a/B)$ is isolated (over B) for every $a\models p(x)$.
  Prove that $p(x)$ is isolated (over A).
\end{exercise}

\begin{exercise}
  Let $|x|=1$.
  Prove that if $S_{x}(A)$ is countable for every finite set $A$, then $T$ is small.
\end{exercise}
  
\begin{exercise}
  Assumimamo $L$ numerabile.
  Sia $p(x)\subseteq L$ un tipo consistente non isolato.
  Esiste sempre un modello omogeneo che non realizza $p(x)$?
\end{exercise}

\begin{exercise}
Let $p(x)\subseteq L(B)$ and $p_n(x)\subseteq L(A)$, for $n<\omega$, be consistent types such that\smallskip
$$
p(x)\imp\bigvee_{n<\omega}p_n(x)
$$
\noindent Prove that there is an $n<\omega$ and a formula $\varphi(x)\in L(A)$ consistent with $p(x)$ such that \smallskip
$$
p(x)\wedge\varphi(x)\imp p_n(x).
$$
Give an example that proves that the claim does not hold when $\omega$ is replaced with un uncountable cardinal.
\end{exercise}


\end{document}

