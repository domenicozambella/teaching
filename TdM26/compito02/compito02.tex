\documentclass[10pt]{article}
\usepackage[utf8]{inputenc}
\usepackage[a4paper,height=20cm,width=13cm]{geometry}
\usepackage{amssymb}
\usepackage{dsfont}
\usepackage{calc}
\usepackage{graphicx}
% \usepackage{pst-node}
\usepackage{fourier}
\usepackage{euscript}
\usepackage{amsmath,amsthm}
\def\phi{\varphi}
\def\P{\EuScript P}
\def\M{\EuScript M}
\def\D{\EuScript D}
\def\U{\EuScript U}
\def\S{\EuScript S}
\def\sm{\smallsetminus}
\def\niff{\nleftrightarrow}
\def\ZZ{\mathds Z}
\def\NN{\mathds N}
\def\PP{\mathds P}
\def\QQ{\mathds Q}
\def\RR{\mathds R}
\def\<{\langle}
\def\>{\rangle}
\def\E{\exists}
\def\A{\forall}
\def\0{\varnothing}
\def\imp{\rightarrow}
\def\iff{\leftrightarrow}
\def\IMP{\Rightarrow}
\def\IFF{\Leftrightarrow}
\def\range{\textrm{im}}
\def\Mod{\textrm{Mod}}
\def\Aut{\textrm{Aut}}
\def\Th{\textrm{Th}}
\def\acl{\textrm{acl}}
\def\dom{{\rm dom}}
\def\eq{{\rm eq}}
\def\tp{\textrm{tp}}
\def\equivL{\stackrel{\smash{\scalebox{.5}{\rm L}}}{\equiv}}

\def\cnonfork{\mathbin{\raise1.8ex\rlap{\kern0.6ex\rule{0.6ex}{0.1ex}}\rlap{\kern1.1ex\rule{0.1ex}{1.9ex}}\raise-0.3ex\hbox{$\smile$} } }

\newcommand{\labella}[1]{{\sf\footnotesize #1}\hfill}
\renewenvironment{itemize}
  {\begin{list}{$\triangleright$}{%
   \setlength{\parskip}{0mm}
   \setlength{\topsep}{0mm}
   \setlength{\rightmargin}{0mm}
   \setlength{\listparindent}{0mm}
   \setlength{\itemindent}{0mm}
   \setlength{\labelwidth}{3ex}
   \setlength{\itemsep}{0mm}
   \setlength{\parsep}{0mm}
   \setlength{\partopsep}{0mm}
   \setlength{\labelsep}{1ex}
   \setlength{\leftmargin}{\labelwidth+\labelsep}
   \let\makelabel\labella}}{%
   \vspace*{-.3\baselineskip}
  \end{list}}
%\def\ssf#1{\textsf{\small #1}}
\newcounter{ex}
\newenvironment{exercise}{\bigskip\addtocounter{ex}{1}\textbf{Esercizio \theex.\ }}{}
\pagestyle{empty}
\parindent0ex
\parskip1ex
\raggedbottom
\def\nsR{{}^*\!\RR}
\def\ssf#1{\textsf{#1}}
\renewcommand{\baselinestretch}{1.3}


\usepackage{fancyhdr}
\pagestyle{fancy}
\lfoot{Teoria dei Modelli a.a.~2025/26}
\rhead{}
\rfoot{{\small\thepage}}
\cfoot{}



%
% INDICARE NOME E COGNOME DI TUTTI GLI AUTORI
%
\lhead{Nome/i Cognome/i}
%

\begin{document}

\begin{exercise}
  (Angela Dosio)
  Let $L(A)$ be countable.
  Let $P\subseteq S_x(A)$.
  Sketch (stress sketch) a proof of the following (if true).
  The following are equivalent
  \begin{itemize}
    \item [1.] there is a model $M\supseteq A$ that omits all types in $P$
    \item [2.] $P$ is meager in the $A$-topology
    \item [3.] there is a model $M$ such that $\big\{p\in S_x(M)\ :\ p{\restriction} A\in P\big\}$ is meager in the $M$-topology.
  \end{itemize} 
\end{exercise}

\begin{exercise}
  (Costanza Furone)
  Let $p(x)\subseteq L(B)$ and $p_n(x)\subseteq L(A)$, for $n<\omega$, be such that

  \hfil $p(x)\ \ \imp\ \ \displaystyle\bigvee_{i<\omega}p_i(x)$

  Prove that $p(x)\wedge\phi(x)\ \imp\ p_n(x)$
  for some $n<\omega$ and some formula $\phi(x)\in L(A)$ consistent with $p(x)$.
\end{exercise}

\begin{exercise} 
  (Leonardo Centazzo)
  Let $A\subseteq B$ and $p(x)\subseteq L(A)$. 
  Suppose that $\tp(a/B)$ is isolated over $B$ for every $a\models p(x)$.
  Prove that $p(x)$ is isolated over $A$.
\end{exercise}

% \begin{exercise}
%    Prove that the following are equivalent
%    \begin{itemize}   
%    \item[1.] $T$ is $\omega$-categorical
%    \item[2.] for every $M$ countable and $x$ finite, $\Aut(M)$ induces finitely many orbits in $M^x$.
%    \end{itemize}
% \end{exercise}

\begin{exercise}
  (Pietro Giura)
  Prove that the following are equivalent for every finite set $A$
  \begin{itemize}   
  \item[1.] $T$ is $\omega$-categorical
  \item[2.] $T$ is $\omega$-categorical over $A$.
  \end{itemize}
\end{exercise}

% \begin{exercise}
%   Prove that the following are equivalent
%   \begin{itemize}   
%   \item[1.] $T$ is $\omega$-categorical
%   \item[2.] there is a countable model that is both saturated and atomic. 
%   \end{itemize}
% \end{exercise}

\begin{exercise} 
  (Francesco Sulpizi)
  Assume $L$ is countable and that $T$ is complete.
  Suppose that for every finite tuple $x$ there is a model $M$ that realizes only finitely many types in $S_x(T)$.
  Prove that $T$ is $\omega$-categorical.
\end{exercise}

\begin{exercise}
  (Roberto Carnevale)
  The language $L$ contains only a finite number of relational symbols.
  Let $M$ be a countable structure that is \textit{set-ultrahomogeneous}, that is,  for every finite partial embedding $k:M\to M$ there is an $h\in\Aut(M)$ such that $h[\dom k]=\range k$.
  Prove that $\Th(M)$ is $\omega$-categorical and model-complete.
\end{exercise}

\begin{exercise}
  (Alessandro Martina)
  Assume $L$ is countable and let $T$ be strongly minimal.
  Prove that the following are equivalent
  \begin{itemize}
  \item[1.] $T$ is $\omega$-categorical
  \item[2.] the algebraic closure of a finite set is finite.
  \end{itemize}
\end{exercise}

\begin{exercise}
  (Davide Peccioli)
Let $|x|=1$.
Prove that if $S_{x}(A)$ is countable for every finite set $A$, then $T$ is small.
\end{exercise}

\begin{exercise}
  (Matteo Bisi)
  Let $T$ be $\omega$-categorical.
  Prove that if all algebraically closed sets are homogeneous, then $T$ is not finitely axiomatizable.
\end{exercise}

\end{document}


