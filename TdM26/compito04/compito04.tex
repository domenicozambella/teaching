\documentclass[10pt]{article}
\usepackage[utf8]{inputenc}
\usepackage[a4paper,height=20cm,width=13cm]{geometry}
\usepackage{amssymb}
\usepackage{dsfont}
\usepackage{calc}
\usepackage{graphicx}
% \usepackage{pst-node}
\usepackage{fourier}
\usepackage{euscript}
\usepackage{amsmath,amsthm}
\def\phi{\varphi}
\def\P{\EuScript P}
\def\M{\EuScript M}
\def\D{\EuScript D}
\def\U{\EuScript U}
\def\V{\EuScript V}
\def\S{\EuScript S}
\def\sm{\smallsetminus}
\def\niff{\nleftrightarrow}
\def\ZZ{\mathds Z}
\def\NN{\mathds N}
\def\PP{\mathds P}
\def\QQ{\mathds Q}
\def\RR{\mathds R}
\def\<{\langle}
\def\>{\rangle}
\def\E{\exists}
\def\A{\forall}
\def\0{\varnothing}
\def\imp{\rightarrow}
\def\iff{\leftrightarrow}
\def\IMP{\Rightarrow}
\def\IFF{\Leftrightarrow}
\def\range{\textrm{im}}
\def\Mod{\textrm{Mod}}
\def\Aut{\textrm{Aut}}
\def\Th{\textrm{Th}}
\def\acl{\textrm{acl}}
\def\dcl{\textrm{dcl}}
\def\dom{{\rm dom}}
\def\eq{{\rm eq}}
\def\tp{\textrm{tp}}
\def\equivSh{\stackrel{\smash{\scalebox{.5}{\rm Sh}}}{\equiv}}
\def\equivL{\stackrel{\smash{\scalebox{.5}{\rm L}}}{\equiv}}

\def\cnonfork{\mathbin{\raise1.8ex\rlap{\kern0.6ex\rule{0.6ex}{0.1ex}}\rlap{\kern1.1ex\rule{0.1ex}{1.9ex}}\raise-0.3ex\hbox{$\smile$} } }

\newcommand{\labella}[1]{{\sf\footnotesize #1}\hfill}
\renewenvironment{itemize}
  {\begin{list}{$\triangleright$}{%
   \setlength{\parskip}{0mm}
   \setlength{\topsep}{0mm}
   \setlength{\rightmargin}{0mm}
   \setlength{\listparindent}{0mm}
   \setlength{\itemindent}{0mm}
   \setlength{\labelwidth}{3ex}
   \setlength{\itemsep}{0mm}
   \setlength{\parsep}{0mm}
   \setlength{\partopsep}{0mm}
   \setlength{\labelsep}{1ex}
   \setlength{\leftmargin}{\labelwidth+\labelsep}
   \let\makelabel\labella}}{%
   \vspace*{-.3\baselineskip}
  \end{list}}
%\def\ssf#1{\textsf{\small #1}}
\newcounter{ex}
\newenvironment{exercise}{\bigskip\addtocounter{ex}{1}\textbf{Esercizio \theex.\ }}{}
\pagestyle{empty}
\parindent0ex
\parskip1ex
\raggedbottom
\def\nsR{{}^*\!\RR}
\def\ssf#1{\textsf{#1}}
\renewcommand{\baselinestretch}{1.3}


\usepackage{fancyhdr}
\pagestyle{fancy}
\lfoot{Teoria dei Modelli a.a.~2025/26}
\rhead{}
\rfoot{{\small\thepage}}
\cfoot{}

%
% INDICARE NOME E COGNOME DI TUTTI GLI AUTORI
%
\lhead{Nome/i Cognome/i}
%

\begin{document}


\begin{exercise}
  (Pietro Giura)
  Let $a\cnonfork_M b$ then there is $\V\preceq\U$ that is isomorphic to $\U$ over $M,a$ and such that $\V\cnonfork_Mb$.
\end{exercise}

\begin{exercise} 
  (Costanza Furone)
  Let $\bar c=\langle c_i:i\in I\rangle$ be an $M$-indiscernible sequence.
  Let $a\cnonfork_M\bar c$.
  Prove that $\bar c$ is indiscernible over $M,b$.
\end{exercise}

\begin{exercise} 
  (Francesco Sulpizi)
  Let $a\cnonfork_M b$. 
  Prove that for every $c$ there is $b'\equiv_{M,a} b$ such that $a,c\cnonfork_M b'$. 
\end{exercise}

\begin{exercise}
  (Leonardo Centazzo)
  Show that for every $b\in\U^z$ there is a type $p(x\,;z)\subseteq L(M)$ such that for every  $a\in\U^x$ and $b'\in\U^z$ 
  
  \hfil$a,b'\models p(x\,;z)\ \ \IFF\ \ a\cnonfork_M b'\equiv_M b.$
\end{exercise}

\begin{exercise}
  (Roberto Carnevale)
  Let $T$ be strongly minimal.
  Let $a\in\U^x$ and $b\in\U^z$. 
  Characterize $a\cnonfork_M b$ using the notion of algebraic closure (or possibly, dimension).
  Show that it is equivalent to $b\cnonfork_M a$.
\end{exercise}

\begin{exercise}
  (Alessandro Martina)
  Prove Remark 14.7
\end{exercise}

\begin{exercise}
  (Davide Peccioli)
  Dei sequenti risultati dire quali (pochi) non valgono se sostituiamo $M$ con un generico insieme $A$: Lemma 14.8, Proposizione 14.9, Poposizione 14.11.
\end{exercise}

\begin{exercise}
  (Matteo Bisi) 
  Let $N\succeq M$ be saturated and of cardinality $>|M|$.
  Let $a_i\cnonfork_M N,b$, for $i=1,2$.
  Prove that if $a_1\equiv_N a_2$ then $a_1\equiv_{N,b} a_2$.
\end{exercise}

\begin{exercise}
  (Angela Dosio)
  Show that in general there is no type $p(x\,;z)\subseteq L(M)$ such that for every  $a\in\U^x$ and $b\in\U^z$\vspace*{-2ex}

  \hfil$a\,;b\models p(x\,;z)\ \ \IFF\ \ a\cnonfork_M b.$

  Hint: consider the theory of dense linear orders without endpoints.
\end{exercise}
  
\end{document}


