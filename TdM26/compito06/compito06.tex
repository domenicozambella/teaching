\documentclass[10pt]{article}
\usepackage[utf8]{inputenc}
\usepackage[a4paper,height=20cm,width=13cm]{geometry}
\usepackage{amssymb}
\usepackage{dsfont}
\usepackage{calc}
\usepackage{graphicx}
% \usepackage{pst-node}
\usepackage{fourier}
\usepackage{euscript}
\usepackage{amsmath,amsthm}
\def\phi{\varphi}
\def\P{\EuScript P}
\def\M{\EuScript M}
\def\N{\EuScript N}
\def\D{\EuScript D}
\def\C{\EuScript C}
\def\U{\EuScript U}
\def\V{\EuScript V}
\def\S{\EuScript S}
\def\sm{\smallsetminus}
\def\niff{\nleftrightarrow}
\def\ZZ{\mathds Z}
\def\NN{\mathds N}
\def\PP{\mathds P}
\def\QQ{\mathds Q}
\def\RR{\mathds R}
\def\<{\langle}
\def\>{\rangle}
\def\E{\exists}
\def\A{\forall}
\def\0{\varnothing}
\def\imp{\rightarrow}
\def\iff{\leftrightarrow}
\def\IMP{\Rightarrow}
\def\IFF{\Leftrightarrow}
\def\range{\textrm{im}}
\def\Mod{\textrm{Mod}}
\def\Aut{\textrm{Aut}}
\def\Th{\textrm{Th}}
\def\acl{\textrm{acl}}
\def\dcl{\textrm{dcl}}
\def\dom{{\rm dom}}
\def\eq{{\rm eq}}
\def\tp{\textrm{tp}}
\def\equivSh{\stackrel{\smash{\scalebox{.5}{\rm Sh}}}{\equiv}}
\def\equivL{\stackrel{\smash{\scalebox{.5}{\rm L}}}{\equiv}}

\def\cnonfork{\mathbin{\raise1.8ex\rlap{\kern0.6ex\rule{0.6ex}{0.1ex}}\rlap{\kern1.1ex\rule{0.1ex}{1.9ex}}\raise-0.3ex\hbox{$\smile$} } }

\newcommand{\labella}[1]{{\sf\footnotesize #1}\hfill}
\renewenvironment{itemize}
  {\begin{list}{$\triangleright$}{%
   \setlength{\parskip}{0mm}
   \setlength{\topsep}{0mm}
   \setlength{\rightmargin}{0mm}
   \setlength{\listparindent}{0mm}
   \setlength{\itemindent}{0mm}
   \setlength{\labelwidth}{3ex}
   \setlength{\itemsep}{0mm}
   \setlength{\parsep}{0mm}
   \setlength{\partopsep}{0mm}
   \setlength{\labelsep}{1ex}
   \setlength{\leftmargin}{\labelwidth+\labelsep}
   \let\makelabel\labella}}{%
   \vspace*{-.3\baselineskip}
  \end{list}}
%\def\ssf#1{\textsf{\small #1}}
\newcounter{ex}
\newenvironment{exercise}{\bigskip\addtocounter{ex}{1}\textbf{Esercizio \theex.\ }}{}
\pagestyle{empty}
\parindent0ex
\parskip1ex
\raggedbottom
\def\nsR{{}^*\!\RR}
\def\ssf#1{\textsf{#1}}
\renewcommand{\baselinestretch}{1.3}


\usepackage{fancyhdr}
\pagestyle{fancy}
\lfoot{Teoria dei Modelli a.a.~2025/26}
\rhead{}
\rfoot{{\small\thepage}}
\cfoot{}

%
% INDICARE NOME E COGNOME DI TUTTI GLI AUTORI
%
\lhead{Nome/i Cognome/i}
%

\begin{document}

\begin{exercise}
  (Alessandro Martina)
  Let $\phi(x,y)\in  L$, where $|x|=|y|=1$.
  Suppose there is an infinite set $A\subseteq\U$ such that $\phi(a,b)\niff\phi(b,a)$ for every two distinct $a,b\in A$.
  Prove that $\phi(x\,;y)$ is unstable.
\end{exercise}

\begin{exercise} 
  (Francesco Sulpizi)
  Find some unstable $p(x\,;z)\subseteq L(A)$ that admits no ladder of infinite length.
  Hint: work in a dense linear order, and consider the set of formulas of the form $x<y\ \vee\  y\notin\{a_0,\dots,a_n\}$.
\end{exercise}

\begin{exercise}
  (Leonardo Centazzo)
  Prove that

 \hfil$\displaystyle\bigcup\big\{\M\ :\ \M\text{ a minimal left ideal}\big\}\quad  =\quad\bigcup\big\{\M\ :\ \M\text{ a minimal right ideal}\big\}$.

\end{exercise}

\begin{exercise}
  (Roberto Carnevale)
  For every finite coloring of $\NN$ and every $k<\omega$, 
  there is an infinite sequence $\bar a=\langle a_i(x):i<\omega\rangle$ of nonconstant univariate polynomials with coefficients in $\NN$ such that $\big\{a(h): a(x)\in{\rm fp}(\bar a),\ h<k\big\}$ is monochromatic.
  Can we require that the $a_i(x)$ have increasing degrees?
\end{exercise}

\begin{exercise}
  (Matteo Bisi)
  Let $\Sigma$ be a finite set of homomorphisms $\sigma:S\to C$ 
  between infinite semigroups such that the following intersection is nonempty for all $c\in C$
  
  \hfil$\displaystyle
  \Sigma^{-1}[c]\quad =\quad \bigcap_{\sigma\in\Sigma}\sigma^{-1}[c]$

  where $\sigma^{-1}[c]=\{s\in S:\ \sigma(s)=c\}$.
  Prove that, for every finite coloring of $C$, there is some $g\in S\sm C$ such that the set $\{\sigma\,g\ :\ \sigma\in\Sigma\}$ is monochromatic.
\end{exercise}

\begin{exercise}
  (Davide Peccioli)
  The following claim is too strong to be true.
  For every finite coloring of $\NN^{(2)}$ there is an infinite sequence $\bar a=\langle a_i:i<\omega\rangle$ in $\NN$ such that ${\rm fp}(\bar a)^{(2)}$ is monochromatic.
  Replace ${\rm fp}(\bar a)^{(2)}$ with a (slightly) smaller set that makes the claim true.
\end{exercise}

\begin{exercise}
  (Pietro Giura)
  Let $\M$ and $\N$ be type-definable minimal left ideals of $\C$.
  Let $u\in\M$ and $v\in\N$ be idempotents.
  Prove that $u*\M$ and $v*\N$ are isomorphic groups.
\end{exercise}

\begin{exercise} 
  (Costanza Furone)
  Let $\M$ be a type-definable minimal left ideal of $\C$.
  Let $u\in\M$ be an idempotent.
  Prove that $u*\M$ is a minimal right ideal of $\C$.
\end{exercise}

\begin{exercise}
  (Angela Dosio)
  For every finite coloring of $\NN$ and every $k<\omega$, 
  there is an infinite sequence $\bar a=\langle a_i(x):i<\omega\rangle$ of nonconstant univariate polynomials with coefficients in $\NN$ such that $\big\{a(h): a(x)\in{\rm fp}(\bar a),\ h<k\big\}$ is monochromatic.
  Can we require that the $a_i(x)$ have increasing degrees?
\end{exercise}
  
\end{document}


