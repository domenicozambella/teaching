\documentclass[10pt]{article}
\usepackage[utf8]{inputenc}
\usepackage[a4paper,height=20cm,width=13cm]{geometry}
\usepackage{amssymb}
\usepackage{dsfont}
\usepackage{calc}
\usepackage{graphicx}
% \usepackage{pst-node}
\usepackage{fourier}
\usepackage{euscript}
\usepackage{amsmath,amsthm}
\def\phi{\varphi}
\def\P{\EuScript P}
\def\M{\EuScript M}
\def\D{\EuScript D}
\def\U{\EuScript U}
\def\S{\EuScript S}
\def\sm{\smallsetminus}
\def\niff{\nleftrightarrow}
\def\ZZ{\mathds Z}
\def\NN{\mathds N}
\def\PP{\mathds P}
\def\QQ{\mathds Q}
\def\RR{\mathds R}
\def\<{\langle}
\def\>{\rangle}
\def\E{\exists}
\def\A{\forall}
\def\0{\varnothing}
\def\imp{\rightarrow}
\def\iff{\leftrightarrow}
\def\IMP{\Rightarrow}
\def\IFF{\Leftrightarrow}
\def\range{\textrm{im}}
\def\Mod{\textrm{Mod}}
\def\Aut{\textrm{Aut}}
\def\Th{\textrm{Th}}
\def\acl{\textrm{acl}}
\def\dom{{\rm dom}}
\def\eq{{\rm eq}}
\def\tp{\textrm{tp}}
\def\equivL{\stackrel{\smash{\scalebox{.5}{\rm L}}}{\equiv}}

\def\cnonfork{\mathbin{\raise1.8ex\rlap{\kern0.6ex\rule{0.6ex}{0.1ex}}\rlap{\kern1.1ex\rule{0.1ex}{1.9ex}}\raise-0.3ex\hbox{$\smile$} } }

\newcommand{\labella}[1]{{\sf\footnotesize #1}\hfill}
\renewenvironment{itemize}
  {\begin{list}{$\triangleright$}{%
   \setlength{\parskip}{0mm}
   \setlength{\topsep}{0mm}
   \setlength{\rightmargin}{0mm}
   \setlength{\listparindent}{0mm}
   \setlength{\itemindent}{0mm}
   \setlength{\labelwidth}{3ex}
   \setlength{\itemsep}{0mm}
   \setlength{\parsep}{0mm}
   \setlength{\partopsep}{0mm}
   \setlength{\labelsep}{1ex}
   \setlength{\leftmargin}{\labelwidth+\labelsep}
   \let\makelabel\labella}}{%
   \vspace*{-.3\baselineskip}
  \end{list}}
%\def\ssf#1{\textsf{\small #1}}
\newcounter{ex}
\newenvironment{exercise}{\bigskip\addtocounter{ex}{1}\textbf{Esercizio \theex.\ }}{}
\pagestyle{empty}
\parindent0ex
\parskip1ex
\raggedbottom
\def\nsR{{}^*\!\RR}
\def\ssf#1{\textsf{#1}}
\renewcommand{\baselinestretch}{1.3}


\usepackage{fancyhdr}
\pagestyle{fancy}
\lfoot{Teoria dei Modelli a.a.~2025/26}
\rhead{}
\rfoot{{\small\thepage}}
\cfoot{}



%
% INDICARE NOME E COGNOME DI TUTTI GLI AUTORI
%
\lhead{Nome/i Cognome/i}
%

\begin{document}

\begin{exercise}
  (Angela Dosio)
  Let $M$ and $N$ be elementarily homogeneous structures of the same cardinality $\lambda$. 
  Suppose that $M\models\E x\, p(x)\,\IFF\,N\models\E x\, p(x)$ for every $p(x)\subseteq L$ such that $|x|<\lambda$
  Prove that the two structures are isomorphic. 
\end{exercise}

\begin{exercise}
  (Davide Peccioli)
  Let $|L|\le\lambda$. 
  Prove that the following are equivalent\nobreak
  \begin{itemize}
  \item[1.] $N$ is $\lambda$-saturated
  \item[2.] $N$ is weakly $\lambda$-saturated and weakly $\lambda$-homogeneous.
  \end{itemize}\smallskip
\end{exercise}

\begin{exercise}
  (Costanza Furone)
  Let $\phi(x)\in L$.
  Prove that the following are equivalent
  \begin{itemize}
   \item[1.] $\phi(x)$ is equivalent (in $\U$) to some $\psi(x)\in L_{\rm qf}$
   \item[2.] $\phi(a)\iff\phi(fa)$ for every partial embedding $f:\U\to\U$ and $a\in(\dom f)^x$.
  \end{itemize}
  Can we use this result to infer that if all partial embeddings between models of $T$ (any $T$, non necessarily complete) are elementary then $T$ has elimination of quantifiers?
\end{exercise}

\begin{exercise}
  (Alessandro Martina)
  Let $M$ be $\omega$-saturated and let $k:M\imp N$ be a finite $\Delta$-morphism. 
  Prove that the following are equivalent
  \begin{itemize}
  \item[1.] $k:M\imp N$ is a $\{\A\vee\}\Delta$-morphism
  \item[2.] for every $c\in N^{<\omega}$ there is $b\in M^{<\omega}$ such that $k\cup\{\<b,c\>\}:M\imp N$ is a $\Delta$-morphism.
  \end{itemize}
\end{exercise}

\begin{exercise}
  (Matteo Bisi)
  Let $T$ be a complete theory without finite models in a language that consists only of unary predicates. 
  Prove that $T$ has elimination of quantifiers.
  Is the claim true if $T$ is not complete?
\end{exercise}

\begin{exercise}
  (Roberto Carnevale)
  Let $T$ be the theory of \emph{discrete linear orders}, that is, $T$ extends the theory of linear orders $T_{\rm lo}$ with the following two of axioms
  \begin{itemize}
  \item[dis$\uparrow$.] $\E z\ \big[x<z\ \wedge\ \neg\E y\ x<y<z\big]$
  \item[dis$\downarrow$.] $\E z\ \big[ z<x\ \wedge\ \neg\E y\ z<y<x\big]$.
  \end{itemize}
  Let $\Delta$ be the set of formulas that contains (all alphabetic variants of) the formulas 

  \hfil $x<_ny\ \ :=\ \ \E^{\ge n}z\ (x\mathord<z\mathord< y)$ 

  and their negations, for every $n>0$.

  Prove that the theory of discrete linear orders has $\Delta$-elimination of quantifiers.
  Prove that the structure $\QQ\times\ZZ$ ordered with the lexicographic order is a saturated model of $T$.
\end{exercise}

\begin{exercise}
  (Leonardo Centazzo)
  Work in a monster model $\U$.
  Suppose that every formula $\phi(x)\in L$ is equivalent to a some quantifier-free formula $\psi(x)\in L(\U)$.
  Prove that $T$ has positive $\Delta$-elimination of quantifiers for $\Delta$ the set of formulas of the form $\E y\,\A z\,\theta(x,y,z)$ with $\theta(x,y,z)\in L_{\rm at^\pm}$.
\end{exercise}

\begin{exercise}
  (Francesco Sulpizi)
  Let $T$ be a consistent theory without finite models.
  Suppose that all completions of $T$ are of the form $T\cup S$ for some set $S$ of quantifier-free sentences.
  Apply Corollary 10.12 to prove that if all completions of $T$ have elimination of quantifiers, so does $T$.
  Show that this fails when the completions of $T$ have arbitrary complexity.
\end{exercise}

\begin{exercise}
  (Pietro Giura)
  Let $T$ be a consistent theory without finite models.
  Suppose that all completions of $T$ are of the form $T\cup S$ for some set $S$ of quantifier-free sentences.
  Apply the compactness theorem prove that if all completions of $T$ have elimination of quantifiers, so does $T$.
\end{exercise}

\end{document}


