\documentclass[10pt]{article}
\usepackage[utf8]{inputenc}
\usepackage[a4paper,height=20cm,width=13cm]{geometry}
\usepackage{amssymb}
\usepackage{dsfont}
\usepackage{calc}
\usepackage{graphicx}
% \usepackage{pst-node}
\usepackage{fourier}
\usepackage{euscript}
\usepackage{amsmath,amsthm}
\def\phi{\varphi}
\def\P{\EuScript P}
\def\M{\EuScript M}
\def\D{\EuScript D}
\def\U{\EuScript U}
\def\S{\EuScript S}
\def\sm{\smallsetminus}
\def\niff{\nleftrightarrow}
\def\ZZ{\mathds Z}
\def\NN{\mathds N}
\def\PP{\mathds P}
\def\QQ{\mathds Q}
\def\RR{\mathds R}
\def\<{\langle}
\def\>{\rangle}
\def\E{\exists}
\def\A{\forall}
\def\0{\varnothing}
\def\imp{\rightarrow}
\def\iff{\leftrightarrow}
\def\IMP{\Rightarrow}
\def\IFF{\Leftrightarrow}
\def\range{\textrm{im}}
\def\Mod{\textrm{Mod}}
\def\Aut{\textrm{Aut}}
\def\Th{\textrm{Th}}
\def\acl{\textrm{acl}}
\def\dcl{\textrm{dcl}}
\def\dom{{\rm dom}}
\def\eq{{\rm eq}}
\def\tp{\textrm{tp}}
\def\equivSh{\stackrel{\smash{\scalebox{.5}{\rm Sh}}}{\equiv}}
\def\equivL{\stackrel{\smash{\scalebox{.5}{\rm L}}}{\equiv}}

\def\cnonfork{\mathbin{\raise1.8ex\rlap{\kern0.6ex\rule{0.6ex}{0.1ex}}\rlap{\kern1.1ex\rule{0.1ex}{1.9ex}}\raise-0.3ex\hbox{$\smile$} } }

\newcommand{\labella}[1]{{\sf\footnotesize #1}\hfill}
\renewenvironment{itemize}
  {\begin{list}{$\triangleright$}{%
   \setlength{\parskip}{0mm}
   \setlength{\topsep}{0mm}
   \setlength{\rightmargin}{0mm}
   \setlength{\listparindent}{0mm}
   \setlength{\itemindent}{0mm}
   \setlength{\labelwidth}{3ex}
   \setlength{\itemsep}{0mm}
   \setlength{\parsep}{0mm}
   \setlength{\partopsep}{0mm}
   \setlength{\labelsep}{1ex}
   \setlength{\leftmargin}{\labelwidth+\labelsep}
   \let\makelabel\labella}}{%
   \vspace*{-.3\baselineskip}
  \end{list}}
%\def\ssf#1{\textsf{\small #1}}
\newcounter{ex}
\newenvironment{exercise}{\bigskip\addtocounter{ex}{1}\textbf{Esercizio \theex.\ }}{}
\pagestyle{empty}
\parindent0ex
\parskip1ex
\raggedbottom
\def\nsR{{}^*\!\RR}
\def\ssf#1{\textsf{#1}}
\renewcommand{\baselinestretch}{1.3}


\usepackage{fancyhdr}
\pagestyle{fancy}
\lfoot{Teoria dei Modelli a.a.~2025/26}
\rhead{}
\rfoot{{\small\thepage}}
\cfoot{}



%
% INDICARE NOME E COGNOME DI TUTTI GLI AUTORI
%
\lhead{Nome/i Cognome/i}
%

\begin{document}

\begin{exercise}
  (Leonardo Centazzo)
  Let $A\subseteq\U$ and let $\D$ be a definable set with finite orbit over $A$.
  Without using the $\eq$-expansion, prove that $\D$ is union of classes of a finite equivalence relation definable over $A$.
\end{exercise}

\begin{exercise}
  (Pietro Giura)
    Prove that the following are equivalent
    \begin{itemize}
    \item[1.]  $T$ has weak elimination of imaginaries
    \item[2.]  $T$ has geometric elimination of imaginaries and $\acl^\eq A=\dcl^\eq\big(\acl A\big)$ for every $A\subseteq\U$
    \end{itemize} 
\end{exercise}

\begin{exercise} 
  (Costanza Furone)
  Let $A\subseteq N\models T_{\rm rg}$.
  Let $\phi(x)\in L(A)$.
  Prove that if $\phi(N)$ nonempty and disjoint from $A$ then $\phi(N)$ is a random graph.
\end{exercise}

\begin{exercise} 
  (Francesco Sulpizi)
  Prove that if $\phi(N)$ nonempty and disjoint from $A$ then $\phi(N)$ is a random graph.
  Let $p(x)\subseteq L(A)$ and let $\phi(x\,;y)\in L(A)$ be a formula that defines, when restricted to $p(\U^x)$, an equivalence relation with finitely many classes.
  Prove that there is a finite equivalence relation definable over $A$ that coincides with $\phi(x\,;y)$ on $p(\U^x)$.
\end{exercise}

\begin{exercise}
  (Roberto Carnevale)
  The theory $T$ has \textit{uniform\/} elimination of imaginaries if for every formula $\phi(x\,;u)$ there is a formula $\sigma(x,;z)$ such that  

  \hfil  $\A u\ \E^{=1} z\ \A x\ \big[\phi(x\,;u)\ \iff\ \sigma(x\,;z)\big]$

  Claim. If $\U$ contains two definable elements (for simplicity, call them $0$ and $1$) then elimination of imaginaries implies uniform elimination of imaginaries.
  
  Unfortunately, as stated the claim is false~--~if only for fatuous reasons. So, prove it assuming (in the definition of uniform elimination) that $\A u\ \E x\ \phi(x\,;u)$.
\end{exercise}

\begin{exercise}
  (Alessandro Martina)
  Prove that the following are equivalent
  \begin{itemize}
  \item[1.] $T$ has weak elimination of imaginaries
  \item[2.] for every $\D\in \U^\eq$ there exists the least algebraically closed set $A\subseteq\U$ such that $\D$ is definable over $A$.
  \end{itemize}
\end{exercise}

\begin{exercise}
  (Davide Peccioli)
  Prove that following are equivalent for every $A\subseteq\U$
  \begin{itemize}
  \item[1.]  $\acl^\eq A=\dcl^\eq\big(\acl A\big)$ for every $A\subseteq\U$
  \item[2.]  $\Aut(\U/\acl^\eq A)= \Aut(\U/\acl A)$
  \item[3.] $c\,\equiv_{\acl A}b\ \IFF\ c\ \smash{\equivSh_A}\, b$ \ \ for every $A\subseteq\U$ and  $c,b\in\U^{<\omega}$.
  \end{itemize} 
\end{exercise}

\begin{exercise}
  (Matteo Bisi)
  Let $T$ have elimination of imaginaries.
  Let $\phi(x\,;z)\in L(A)$ and  $c\in\U^z$ be given.
  Prove that if the orbit of $\phi(\U^x; c)$ over $A$ is finite, then $\phi(\U^x; c)$ is definable over $\acl A$.
  Is the same true assuming only weak elemination?
\end{exercise}

\begin{exercise}
  (Angela Dosio)
  For which of the following theories every completion has elimination of imaginaries?
  (Some answers are trivial, others I do not know.)
  \begin{itemize}
    \item [1.] PA
    \item [2.] ZF+V=L
    \item [3.] ZFC
    \item [4.] ZF.
  \end{itemize}
\end{exercise}
\end{document}


