\documentclass[10pt]{article}
\usepackage[utf8]{inputenc}
\usepackage[a4paper,height=20cm,width=13cm]{geometry}
\usepackage{amssymb}
\usepackage{dsfont}
\usepackage{calc}
\usepackage{graphicx}
% \usepackage{pst-node}
\usepackage{fourier}
\usepackage{euscript}
\usepackage{amsmath,amsthm}
\def\phi{\varphi}
\def\P{\EuScript P}
\def\M{\EuScript M}
\def\D{\EuScript D}
\def\U{\EuScript U}
\def\V{\EuScript V}
\def\S{\EuScript S}
\def\sm{\smallsetminus}
\def\niff{\nleftrightarrow}
\def\ZZ{\mathds Z}
\def\NN{\mathds N}
\def\PP{\mathds P}
\def\QQ{\mathds Q}
\def\RR{\mathds R}
\def\<{\langle}
\def\>{\rangle}
\def\E{\exists}
\def\A{\forall}
\def\0{\varnothing}
\def\imp{\rightarrow}
\def\iff{\leftrightarrow}
\def\IMP{\Rightarrow}
\def\IFF{\Leftrightarrow}
\def\range{\textrm{im}}
\def\Mod{\textrm{Mod}}
\def\Aut{\textrm{Aut}}
\def\Th{\textrm{Th}}
\def\acl{\textrm{acl}}
\def\dcl{\textrm{dcl}}
\def\dom{{\rm dom}}
\def\eq{{\rm eq}}
\def\tp{\textrm{tp}}
\def\equivSh{\stackrel{\smash{\scalebox{.5}{\rm Sh}}}{\equiv}}
\def\equivL{\stackrel{\smash{\scalebox{.5}{\rm L}}}{\equiv}}

\def\cnonfork{\mathbin{\raise1.8ex\rlap{\kern0.6ex\rule{0.6ex}{0.1ex}}\rlap{\kern1.1ex\rule{0.1ex}{1.9ex}}\raise-0.3ex\hbox{$\smile$} } }

\newcommand{\labella}[1]{{\sf\footnotesize #1}\hfill}
\renewenvironment{itemize}
  {\begin{list}{$\triangleright$}{%
   \setlength{\parskip}{0mm}
   \setlength{\topsep}{0mm}
   \setlength{\rightmargin}{0mm}
   \setlength{\listparindent}{0mm}
   \setlength{\itemindent}{0mm}
   \setlength{\labelwidth}{3ex}
   \setlength{\itemsep}{0mm}
   \setlength{\parsep}{0mm}
   \setlength{\partopsep}{0mm}
   \setlength{\labelsep}{1ex}
   \setlength{\leftmargin}{\labelwidth+\labelsep}
   \let\makelabel\labella}}{%
   \vspace*{-.3\baselineskip}
  \end{list}}
%\def\ssf#1{\textsf{\small #1}}
\newcounter{ex}
\newenvironment{exercise}{\bigskip\addtocounter{ex}{1}\textbf{Esercizio \theex.\ }}{}
\pagestyle{empty}
\parindent0ex
\parskip1ex
\raggedbottom
\def\nsR{{}^*\!\RR}
\def\ssf#1{\textsf{#1}}
\renewcommand{\baselinestretch}{1.3}


\usepackage{fancyhdr}
\pagestyle{fancy}
\lfoot{Teoria dei Modelli a.a.~2025/26}
\rhead{}
\rfoot{{\small\thepage}}
\cfoot{}

%
% INDICARE NOME E COGNOME DI TUTTI GLI AUTORI
%
\lhead{Nome/i Cognome/i}
%

\begin{document}



\begin{exercise}
  (Alessandro Martina)
  Let $\bar a=\<a_i:i\in I\>$ be an $A$-indiscernible sequence and let $J\supseteq I$ with $|J|\le \kappa$.
  Then there is an $A$-indiscernible sequence $\bar c=\<c_i:i\in J\>$ such that $c_{\restriction I}=\bar a$.
\end{exercise}

\begin{exercise} 
  (Francesco Sulpizi)
  Let  $p(x)\in S(\U)$ be a global type invariant over $A$.
  Let $a,b\models p_{\restriction A}(x)$.
  Prove that there is a sequence $\bar c=\<c_i:i<\omega\>$ such that $a,\bar c$ and $b,\bar c$ are both sequences of $A$-indiscernibles.
\end{exercise}

\begin{exercise}
  (Matteo Bisi)
  Let $M$ be an arbitrary model.
  Let $\phi(x,y)\in L(M)$, where $|x|=|y|$, be such that $M\subseteq\phi(\U,a)$ for some $a\in\U$.
  Prove that there is a sequence $\langle a_i:i<\omega\rangle$ in $M^{x}$ such that $\phi(a_i,a_j)$ holds for every $i<j<\omega$.
\end{exercise}

\begin{exercise}
  (Leonardo Centazzo)
  Assume that there is an elementary extension $\U\preceq\NN$ that contains an element $a$ such that $a+a\equiv a$ (the signature is arbitrary but contains a symbol for $+$).
  Prove that for every coloring of $\NN$ in finitely many colors and for every $n<\omega$, there is a monochromatic arithmetical progression of length $n$.
  An \textit{arithmetical progression\/} of length $n$ is a sequence of natural numbers $\langle a_i : i\le n\rangle$ such that $a_i - a_{i+1}$ is constant for $i<n$.
\end{exercise}

\begin{exercise}
  (Roberto Carnevale)
  Let $\<c_i:i<\omega\>$ be an indiscernible sequence.
  Prove that there is an indiscernible sequence $\<d_i:i<\omega\>$ such that $d_0,d_1 = c_1,c_0$.
\end{exercise}

\begin{exercise}
  (Davide Peccioli)
  Let $\<\D_i:i<\omega\>$ be an $A$-indiscernible sequence in $\U^\eq$.
  Prove that there is an $A$-indiscernible sequence $\<b_i:i<\omega\>$ in $\U^z$ and a formula $\phi(x\,;z)\in L$ such that $\D_i=\phi(\U\,;b_i)$.
\end{exercise}

\begin{exercise}
  (Pietro Giura)
  Let $M$ be an arbitrary model.
  Let $\phi(x,y)\in L(M)$, where $|x|=|y|$.
  Prove that there is a sequence $\langle a_i:i<\omega\rangle$ in $M^{x}$ such that

  \hfil$\phi(a_i,a_j)\iff\phi(a_h,a_k)$

  for every $i<j<\omega$ and $h<k<\omega$.
\end{exercise}

\begin{exercise} 
  (Costanza Furone)
  Let $M$ be a graph with the property that for every finite $A\subseteq M$ there is a $c\in M$ such that $A\subseteq r(c,\U)$. 
  A star in $M$ is a subgraph whose edges all share a common vertex. We say that a coloring of the edges of $M$ is locally finite if there is a $k$ such that every star has at most $k$ colors.
  Prove that for every locally finite coloring of the edges of $M$, there is an infinite monochromatic complete subgraph.
\end{exercise}

\begin{exercise}
  (Angela Dosio)
\end{exercise}
  
\end{document}


