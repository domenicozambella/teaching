\documentclass[11pt,twoside,a4paper]{article}
\usepackage[T1]{fontenc}
\usepackage[utf8]{inputenc}
\usepackage[top=20mm, head=6mm, headsep=6mm, foot=6mm, bottom= 15mm, left=15mm, right=15mm]{geometry}
\usepackage{amsmath}
\usepackage{pythontex}
\parindent0ex
\parskip2ex
\def\Pr{{\rm Pr\,}}
\def\Ex{{\rm E\,}}
\def\Var{{\rm Var\,}}
\newenvironment{question}{\bigskip\par\textbf{Quesito.\kern1ex}}{\vspace{\parskip}}
\newenvironment{answer}{\par\textbf{Risposta.\quad}}{}

\newenvironment{xquestion}{\par\textbf{XQuesito.\kern1ex}}{\vspace{\parskip}}

\pagestyle{empty} 

\begin{document}
Domande (qualcuna capziosa e artificiale) per verificare la comprensione del significato di p-valore. 

N.B. Le domande contengono anche informazioni irrilevanti.

\begin{pycode}
import random
\end{pycode}

\begin{question}
\begin{pycode}
n = random.choice([2,3])
pval = random.choice([0.05,0.02])
test = random.choice(['T-test a due code','T-test a una coda','Z-test ($\sigma$ nota, una coda)', 'Z-test ($\sigma$ nota, due code)'])
dim  = random.choice(['sempre della stessa dimensione', 'di dimensione crescente'])
risposta = round(1-(1-pval)**n, 3)
\end{pycode}
Ripetiamo \py{n} volte lo stesso \py{test} con campioni \py{dim}.
Assumendo vera $H_0$, qual'è la probabilità che in almeno uno di questi test il p-valore risulti $\le\py{pval}$? 

Nel caso non sia possibile determinare il valore esatto ma solo un limite superiore/inferiore. Si scelga tra le seguenti opzioni la più opportuna.
\begin{itemize}
\item[1.] La probabilità è $=\dots$
\item[2.] La probabilità è $\le\dots$
\item[3.] La probabilità è $\ge\dots$
\item[4.] Non ci sono sufficienti informazioni per stimare questa probabilità.
\end{itemize}
\begin{answer}
1. La probabilità è $=1-(\py{1-pval})^{\py{n}}=\py{risposta}$.
\end{answer}
\end{question}






\begin{question}
\begin{pycode}
n = random.choice([30,40,50])
pval = random.choice([0.05,0.04,0.02])
test = random.choice(['T-test a due code','T-test a una coda','Z-test ($\sigma$ nota, una coda)', 'Z-test ($\sigma$ nota, due code)'])
condizioni  = random.choice(['con un campione della stessa dimensione', 'con un campione doppio'])
factor =  random.choice([1,2])
\end{pycode}
Abbiamo fatto un \py{test} con un campione di dimensione \py{n} e abbiamo ottenuto il p-valore \py{pval}.
Assumendo vera $H_0$, qual'\`e la  qual'è la probabilità che ripetendo il test una seconda volta \py{condizioni} il p-valore risulti $\le\py{pval/factor}$.

Nel caso non sia possibile determinare il valore esatto ma solo un limite superiore/inferiore. Si scelga tra le seguenti opzioni la più opportuna.
\begin{itemize}
\item[1.] La probabilità è $=\dots$
\item[2.] La probabilità è $\le\dots$
\item[3.] La probabilità è $\ge\dots$
\item[4.] Non ci sono sufficienti informazioni per stimare questa probabilità.
\end{itemize}
\begin{answer}
1. La probabilità è $=\py{pval/factor}$
\end{answer}
\end{question}









\begin{question}
\begin{pycode}
pval = random.choice([0.03,0.02,0.01])
test = random.choice(['T-test a due code','T-test a una coda','Z-test ($\sigma$ nota, una coda)', 'Z-test ($\sigma$ nota, due code)'])
\end{pycode}
Abbiamo fatto un \py{test} e abbiamo ottenuto un p-valore \py{pval}.
Assumendo vera $H_A$, qual è la probabilità che ripetendo il test una seconda volta con un campione della stessa dimensione il p-valore risulti di nuovo $\le\py{pval}$ ?

Nel caso non sia possibile determinare il valore esatto ma solo un limite superiore/inferiore. Si scelga tra le seguenti opzioni la più opportuna.
\begin{itemize}
\item[1.] La probabilità è $=\dots$
\item[2.] La probabilità è $\le\dots$
\item[3.] La probabilità è $\ge\dots$
\item[4.] Non ci sono sufficienti informazioni per stimare questa probabilità.
\end{itemize}
\begin{answer}
4. Non ci sono sufficienti informazioni per stimare questa probabilità.
\end{answer}
\end{question}




\end{document}


\begin{question}
\begin{pycode}
test = random.choice(['T-test a una coda','Z-test ($\sigma$ nota, una coda)'])
test = random.choice(['identica'])
% x1 = -x[0]
% x2 = x[1]
% x3 = x[0]
% p1 = Rational(1,random.choice([2,3,4]))
% p2 = Rational(1,random.choice([3,4]))
% p3 = 1- p1 - p2
% \end{pycode}
Vogliamo fare un \py{test} con un campione di dimensione $\py{n}$.

Fissata una significatività $\alpha=\py{alpha}$ otteniamo un valore di soglia $x_\alpha$. 

Per un dato effect size $\delta$, il test risulta avere potenza $\py{1-beta}$. 

Assumendo (almeno di $\delta$), qual'è la probabilità $q$ che in almeno uno di questi test il p-valore risulti $\ge\py{pval}$. (Nel caso non sia possibile determinare il valore esatto di $q$, ma solo un limite superiore/inferiore, si risponda di consequenza.)

\begin{itemize};
\item[1.] $q=$
\item[1.] $q\le$
\item[1.] $q\ge$
\item[1.] Non ci sono sufficienti informazioni per stimare $q$.

\begin{answer}
$q\le\py{beta}$

\end{answer}
\end{question}


\end{document}

