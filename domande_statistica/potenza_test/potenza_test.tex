\documentclass[11pt,twoside,a4paper]{article}
\usepackage[T1]{fontenc}
\usepackage[utf8]{inputenc}
\usepackage[top=20mm, head=6mm, headsep=6mm, foot=6mm, bottom= 15mm, left=15mm, right=15mm]{geometry}
\usepackage{amsmath}
\usepackage{pythontex}
\parindent0ex
\parskip2ex
\def\Pr{{\rm Pr\,}}
\def\Ex{{\rm E\,}}
\def\Var{{\rm Var\,}}
\newenvironment{question}{\bigskip\par\textbf{Quesito.\kern1ex}}{\vspace{\parskip}}
\newenvironment{answer}{\par\textbf{Risposta.\quad}}{}

\newenvironment{xquestion}{\par\textbf{XQuesito.\kern1ex}}{\vspace{\parskip}}

\pagestyle{empty} 

\begin{document}
Domande (qualcuna capziosa e artificiale) per verificare la comprensione del significato di potenza di un test. 

N.B. Le domande contengono anche informazioni irrilevanti.

\begin{pycode}
import random
\end{pycode}

\begin{question}
\begin{pycode}

\end{pycode}

\begin{answer}

\end{answer}
\end{question}





\end{document}

