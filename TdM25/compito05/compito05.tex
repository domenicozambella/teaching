\documentclass[10pt]{article}
\usepackage[utf8]{inputenc}
\usepackage[a4paper,height=21cm,width=13cm]{geometry}
\usepackage{amssymb}
\usepackage{dsfont}
\usepackage{euscript}
\usepackage{calc}
\usepackage{graphicx}
% \usepackage{pst-node}
\usepackage{fourier}
% \usepackage{euscript}
\usepackage{amsmath,amsthm}
\def\phi{\varphi}
\def\P{\EuScript P}
\def\M{\EuScript M}
\def\D{\EuScript D}
\def\U{\EuScript U}
\def\S{\EuScript S}
\def\sm{\smallsetminus}
\def\niff{\nleftrightarrow}
\def\ZZ{\mathds Z}
\def\NN{\mathds N}
\def\PP{\mathds P}
\def\QQ{\mathds Q}
\def\RR{\mathds R}
\def\<{\langle}
\def\>{\rangle}
\def\E{\exists}
\def\A{\forall}
\def\0{\varnothing}
\def\imp{\rightarrow}
\def\iff{\leftrightarrow}
\def\IMP{\Rightarrow}
\def\IFF{\Leftrightarrow}
\def\range{\textrm{im}}
\def\Mod{\textrm{Mod}}
\def\Aut{\textrm{Aut}}
\def\Th{\textrm{Th}}
\def\acl{\textrm{acl}}
\def\dom{\textrm{dom}}
\def\eq{{\rm eq}}
\def\tp{\textrm{tp}}
\def\equivL{\stackrel{\smash{\scalebox{.5}{\rm L}}}{\equiv}}

\def\cnonfork{\mathbin{\raise1.8ex\rlap{\kern0.6ex\rule{0.6ex}{0.1ex}}\rlap{\kern1.1ex\rule{0.1ex}{1.9ex}}\raise-0.3ex\hbox{$\smile$} } }

\newcommand{\labella}[1]{{\sf\footnotesize #1}\hfill}
\renewenvironment{itemize}
  {\begin{list}{$\triangleright$}{%
   \setlength{\parskip}{0mm}
   \setlength{\topsep}{0mm}
   \setlength{\rightmargin}{0mm}
   \setlength{\listparindent}{0mm}
   \setlength{\itemindent}{0mm}
   \setlength{\labelwidth}{3ex}
   \setlength{\itemsep}{0mm}
   \setlength{\parsep}{0mm}
   \setlength{\partopsep}{0mm}
   \setlength{\labelsep}{1ex}
   \setlength{\leftmargin}{\labelwidth+\labelsep}
   \let\makelabel\labella}}{%
  %  \vspace*{-.5\baselineskip}
  \end{list}}
%\def\ssf#1{\textsf{\small #1}}
\newcounter{ex}
\newenvironment{exercise}{\medskip\addtocounter{ex}{1}\textbf{Esercizio \theex.\ }}{}
\pagestyle{empty}
\parindent0ex
\parskip2ex
\raggedbottom
\def\nsR{{}^*\!\RR}
\def\ssf#1{\textsf{#1}}
\renewcommand{\baselinestretch}{1.3}


\usepackage{fancyhdr}
\pagestyle{fancy}
\lfoot{Teoria dei Modelli a.a.~2024/25}
\rhead{}
\rfoot{{\small\thepage}}
\cfoot{}



%
% INDICARE NOME E COGNOME DI TUTTI GLI AUTORI
%
\lhead{Nome/i Cognome/i}
%
\begin{document}

%%%%%%%%%%%%%%%%
\begin{exercise}
  Suppose that every formula $\phi(x)\in L$ is equivalent to a some quantifier-free formula $\psi(x)\in L(\U)$.
  Prove that $T$ has positive $\Delta$-elimination of quantifiers for $\Delta$ the set of formulas of the form $\E y\,\A z\,\theta(x,y,z)$ with $\theta(x,y,z)\in L_{\rm qf}$.
\end{exercise}

%%%%%%%%%%%%%%%%
\begin{exercise} 
  Let $A\subseteq B$ and $p(x)\subseteq L(A)$. 
  Suppose that $\tp(a/B)$ is isolated over $B$ for every $a\models p(x)$.
  Prove that $p(x)$ is isolated over $A$.
\end{exercise}

%%%%%%%%%%%%%%%%
\begin{exercise}
  A countable structure $M$ is \textit{set-ultrahomogeneous\/} if for every finite partial embedding $k:M\to M$ there is an $h\in\Aut(M)$ such that $\dom k=\dom h$ and $\range k=\range h$.
  Prove that if $L$ contains only a finite number of relational symbols, then set-homogeneous models have an $\omega$-categorical theory.
\end{exercise}

%%%%%%%%%%%%%%%%
\begin{exercise}
  Let $M$ be set-homogeneous set-homogeneous in a language $L$ that contains only a finite number of relational symbols.
  Prove that $\Th(M)$ is model-complete\footnote{Potrebbe essere meno semplice.}.
\end{exercise}

%%%%%%%%%%%%%%%%
\begin{exercise} 
  Assume $L$ is countable and that $T$ is complete.
  Suppose that for every finite tuple $x$ there is a model $M$ that realizes only finitely many types in $S_x(T)$.
  Prove that $T$ is $\omega$-categorical.
\end{exercise}

%%%%%%%%%%%%%%%%
\begin{exercise}
  Let $|x|=1$.
  Prove that if $S_{x}(A)$ is countable for every finite set $A$, then $T$ is small.
\end{exercise}

%%%%%%%%%%%%%%%%
\begin{exercise}
  Let $T$ be a complete theory without finite models in a language that consists only of unary predicates. Prove that $T$ has elimination of quantifiers.
\end{exercise}

%%%%%%%%%%%%%%%%
\begin{exercise}
  Let $T$ be a complete theory without finite models.
  Prove that the following are equivalent
  \begin{itemize}
  \item[1.] $M$ is minimal
  \item[2.] $a\equiv_M b$ for every $a, b\in \U\sm M$.
  \end{itemize}
\end{exercise}
  
%%%%%%%%%%%%%%%%
\begin{exercise}
  Let $\phi(x)\in L_{\rm qf}(\U)$.
  Find a simple theory that disproves the equivalence of the following
  \begin{itemize}
   \item[1.] $\phi(x)$ is equivalent to some formula $\psi(x)\in L_{\rm qf}(A)$
   \item[2.] $\phi(x)$ is invariant over $A$.
  \end{itemize}
\end{exercise}

\vfill
\end{document}


