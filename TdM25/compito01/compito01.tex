\documentclass[10pt]{article}
\usepackage[utf8]{inputenc}
\usepackage[a4paper,height=20cm,width=13cm]{geometry}
\usepackage{amssymb}
\usepackage{dsfont}
\usepackage{calc}
\usepackage{graphicx}
% \usepackage{pst-node}
\usepackage{fourier}
% \usepackage{euscript}
\usepackage{amsmath,amsthm}
\def\phi{\varphi}
\def\P{\EuScript P}
\def\M{\EuScript M}
\def\D{\EuScript D}
\def\U{\EuScript U}
\def\S{\EuScript S}
\def\sm{\smallsetminus}
\def\niff{\nleftrightarrow}
\def\ZZ{\mathds Z}
\def\NN{\mathds N}
\def\PP{\mathds P}
\def\QQ{\mathds Q}
\def\RR{\mathds R}
\def\<{\langle}
\def\>{\rangle}
\def\E{\exists}
\def\A{\forall}
\def\0{\varnothing}
\def\imp{\rightarrow}
\def\iff{\leftrightarrow}
\def\IMP{\Rightarrow}
\def\IFF{\Leftrightarrow}
\def\range{\textrm{im}}
\def\Mod{\textrm{Mod}}
\def\Aut{\textrm{Aut}}
\def\Th{\textrm{Th}}
\def\acl{\textrm{acl}}
\def\eq{{\rm eq}}
\def\tp{\textrm{tp}}
\def\equivL{\stackrel{\smash{\scalebox{.5}{\rm L}}}{\equiv}}

\def\cnonfork{\mathbin{\raise1.8ex\rlap{\kern0.6ex\rule{0.6ex}{0.1ex}}\rlap{\kern1.1ex\rule{0.1ex}{1.9ex}}\raise-0.3ex\hbox{$\smile$} } }

\newcommand{\labella}[1]{{\sf\footnotesize #1}\hfill}
\renewenvironment{itemize}
  {\begin{list}{$\triangleright$}{%
   \setlength{\parskip}{0mm}
   \setlength{\topsep}{0mm}
   \setlength{\rightmargin}{0mm}
   \setlength{\listparindent}{0mm}
   \setlength{\itemindent}{0mm}
   \setlength{\labelwidth}{3ex}
   \setlength{\itemsep}{0mm}
   \setlength{\parsep}{0mm}
   \setlength{\partopsep}{0mm}
   \setlength{\labelsep}{1ex}
   \setlength{\leftmargin}{\labelwidth+\labelsep}
   \let\makelabel\labella}}{%
  %  \vspace*{-.5\baselineskip}
  \end{list}}
%\def\ssf#1{\textsf{\small #1}}
\newcounter{ex}
\newenvironment{exercise}{\bigskip\addtocounter{ex}{1}\textbf{Esercizio \theex.\ }}{}
\pagestyle{empty}
\parindent0ex
\parskip2ex
\raggedbottom
\def\nsR{{}^*\!\RR}
\def\ssf#1{\textsf{#1}}
\renewcommand{\baselinestretch}{1.3}


\usepackage{fancyhdr}
\pagestyle{fancy}
\lfoot{Teoria dei Modelli a.a.~2024/25}
\rhead{}
\rfoot{{\small\thepage}}
\cfoot{}



%
% INDICARE NOME E COGNOME DI TUTTI GLI AUTORI
%
\lhead{Nome/i Cognome/i}
%

\begin{document}

\begin{exercise}
  ($\le 2$ righe.) \ 
  Let $T$ be a consistent theory in a countable language $L$.
  Find a (super easy) counterexample to the following. 
  If $T$ is $\omega$-categorical then $T$ is complete.
\end{exercise}


\begin{exercise}
  ($\le 4$ righe.) \ 
  Let $L$ be the language of strict orders expanded with the constants $\big\{c_i: i\in\omega\big\}$.
  Let $T$ be the theory that extends $T_{\rm dlo}$ with the axioms $c_i<c_{i+1}$ for all $i$.
  Prove that $T$ is complete.
\end{exercise}


\begin{exercise}
  ($\le 2$ righe.) \ 
  Let $L$ be the language of strict orders expanded with the constants $\big\{c_i: i\in\omega\big\}$.                                 
  Let $T$ be the theory that extends $T_{\rm dlo}$ with the axioms $c_i<c_{i+1}$ for all $i$.
  Find a countable model that is not ultrahomogeneous (i.e.\@ the analogue of Theorem~6.15 does not hold for $T$).
\end{exercise}

\begin{exercise}
  ($\le 4$ righe.) \ 
  Let $N$ be free union of two random graphs $N_1$ and $N_2$.
  That is, $N=N_1\sqcup N_2$ and $r^N= r^{N_1}\sqcup r^{N_2}$, where  $\sqcup$ denotes the disjoint union.
  Prove that $N$ is not a random graph.
  Show that $N_1$ is not definable without parameters.
  Write a first order formula $\psi(x,y)$ true if $x$ and $y$ belong to the same connected component of $N$.
\end{exercise}

\begin{exercise}
  ($\le 4$ righe.) \ 
  Let $N$ be a random graph prove that if $M\subseteq N$ then $M$ or $N\sm M$ is also a random graph.
\end{exercise}

\begin{exercise}
  ($\le 4$ righe.) \ 
  Let $A\subseteq N\models T_{\rm rg}$.
  Prove that $\phi(N)$ is a random graph whenever $\phi(x)\in L(A)$ is consistent and such that $\phi(N)$ is disjoint of $A$.
\end{exercise}

\end{document}


