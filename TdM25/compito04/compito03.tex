\documentclass[10pt]{article}
\usepackage[utf8]{inputenc}
\usepackage[a4paper,height=20cm,width=13cm]{geometry}
\usepackage{amssymb}
\usepackage{dsfont}
\usepackage{euscript}
\usepackage{calc}
\usepackage{graphicx}
% \usepackage{pst-node}
\usepackage{fourier}
% \usepackage{euscript}
\usepackage{amsmath,amsthm}
\def\phi{\varphi}
\def\P{\EuScript P}
\def\M{\EuScript M}
\def\D{\EuScript D}
\def\U{\EuScript U}
\def\S{\EuScript S}
\def\sm{\smallsetminus}
\def\niff{\nleftrightarrow}
\def\ZZ{\mathds Z}
\def\NN{\mathds N}
\def\PP{\mathds P}
\def\QQ{\mathds Q}
\def\RR{\mathds R}
\def\<{\langle}
\def\>{\rangle}
\def\E{\exists}
\def\A{\forall}
\def\0{\varnothing}
\def\imp{\rightarrow}
\def\iff{\leftrightarrow}
\def\IMP{\Rightarrow}
\def\IFF{\Leftrightarrow}
\def\range{\textrm{im}}
\def\Mod{\textrm{Mod}}
\def\Aut{\textrm{Aut}}
\def\Th{\textrm{Th}}
\def\acl{\textrm{acl}}
\def\dom{\textrm{dom}}
\def\eq{{\rm eq}}
\def\tp{\textrm{tp}}
\def\equivL{\stackrel{\smash{\scalebox{.5}{\rm L}}}{\equiv}}

\def\cnonfork{\mathbin{\raise1.8ex\rlap{\kern0.6ex\rule{0.6ex}{0.1ex}}\rlap{\kern1.1ex\rule{0.1ex}{1.9ex}}\raise-0.3ex\hbox{$\smile$} } }

\newcommand{\labella}[1]{{\sf\footnotesize #1}\hfill}
\renewenvironment{itemize}
  {\begin{list}{$\triangleright$}{%
   \setlength{\parskip}{0mm}
   \setlength{\topsep}{0mm}
   \setlength{\rightmargin}{0mm}
   \setlength{\listparindent}{0mm}
   \setlength{\itemindent}{0mm}
   \setlength{\labelwidth}{3ex}
   \setlength{\itemsep}{0mm}
   \setlength{\parsep}{0mm}
   \setlength{\partopsep}{0mm}
   \setlength{\labelsep}{1ex}
   \setlength{\leftmargin}{\labelwidth+\labelsep}
   \let\makelabel\labella}}{%
  %  \vspace*{-.5\baselineskip}
  \end{list}}
%\def\ssf#1{\textsf{\small #1}}
\newcounter{ex}
\newenvironment{exercise}{\bigskip\addtocounter{ex}{1}\textbf{Esercizio \theex.\ }}{}
\pagestyle{empty}
\parindent0ex
\parskip2ex
\raggedbottom
\def\nsR{{}^*\!\RR}
\def\ssf#1{\textsf{#1}}
\renewcommand{\baselinestretch}{1.3}


\usepackage{fancyhdr}
\pagestyle{fancy}
\lfoot{Teoria dei Modelli a.a.~2024/25}
\rhead{}
\rfoot{{\small\thepage}}
\cfoot{}



%
% INDICARE NOME E COGNOME DI TUTTI GLI AUTORI
%
\lhead{Nome/i Cognome/i}
%
\begin{document}

%%%%%%%%%%%%%%%%
\begin{exercise}
  Let $L$ be a countable language containing $L_{\rm gr}$ and assume $\U$ is a group.
  Let $p(x)\subseteq L$ define a subgroup of $\U$.
  Prove that there are some formulas $\phi_n(x)\in L$ such that 
  \begin{itemize}
  \item[1.] $p(x)\ \iff\ \{\phi_n(x):n<\omega\}$
  \item[2.] $\phi_{n+1}(x)\wedge\phi_{n+1}(y)\ \imp\ \phi_n(x\cdot y)$.
  \end{itemize}
\end{exercise}

%%%%%%%%%%%%%%%%
\begin{exercise}
  Let $\phi(x\,;z)\in L$ and $a\in\U^z$.
  Prove that the following are equivalent
  \begin{itemize}
  \item[1.] the type $\{\phi(x\,;fa)\,:\,f\in\Aut(\U)\}$ is realized (in $\U$)
  \item[2.] there is a formula $\psi(z)\in L$ such that $\psi(a)$ and $\E x\A z\,[\psi(z)\imp\phi(x\,;z)]$.
  \end{itemize}  
\end{exercise}

%%%%%%%%%%%%%%%%
\begin{exercise}
  Let $\phi(x,y)\in L(\U)$. Prove that if the set $\big\{\phi(a,\U)\, :\, a\in\U^x\big\}$ is infinite then it has cardinality $\kappa$.
  Does the claim remain true with a type $p(x,y)\subseteq L(A)$ for $\phi(x,y)$?
\end{exercise}

%%%%%%%%%%%%%%%%
\begin{exercise}
  Let $\phi(z)\in L(A)$ be a consistent formula.
  Prove that, if $a\in\acl(A,b)$ for every $b\models\phi(z)$, then $a\in\acl A$.
  Prove the same claim with a type $p(z)\subseteq L(A)$ for $\phi(z)$.
\end{exercise}

%%%%%%%%%%%%%%%%
\begin{exercise}
Let $C$ be a finite set.
Prove that if $C\cap M\neq\varnothing$ for every model $M$ containing $A$, then $C\cap\acl A\neq\varnothing$.
\end{exercise}

%%%%%%%%%%%%%%%%
\begin{exercise}
Prove that for every $A\subseteq N$ there is an $M$ such that $\acl A\,=\,M\cap N$.
\end{exercise} 
\end{document}


