\documentclass[10pt]{article}
\usepackage[utf8]{inputenc}
\usepackage[a4paper,height=24cm,width=13cm]{geometry}
\usepackage[italian]{babel}
\usepackage{amssymb}
\usepackage{dsfont}
\usepackage{calc}
\usepackage{graphicx}
\usepackage{pstricks}
\usepackage{pst-node}
\usepackage{fourier}
\usepackage{euscript}
\usepackage{amsmath,amssymb, amsthm}

\def\lh{\textrm{lh}}
\def\phi{\varphi}
\def\P{\EuScript P}
\def\M{\EuScript M}
\def\D{\EuScript D}
\def\U{\EuScript U}
\def\S{\EuScript S}
\def\sm{\smallsetminus}
\def\niff{\nleftrightarrow}
\def\ZZ{\mathds Z}
\def\NN{\mathds N}
\def\PP{\mathds P}
\def\QQ{\mathds Q}
\def\RR{\mathds R}
\def\<{\langle}
\def\>{\rangle}
\def\E{\exists}
\def\A{\forall}
\def\0{\varnothing}
\def\imp{\rightarrow}
\def\iff{\leftrightarrow}
\def\IMP{\Rightarrow}
\def\IFF{\Leftrightarrow}
\def\dom{\textrm{dom}}
\def\range{\textrm{rang}}
\def\Mod{\textrm{Mod}}
\def\Aut{\textrm{Aut}}
\def\Th{\textrm{Th}}
\def\acl{\textrm{acl}}
\def\eq{{\rm eq}}
\def\tp{\textrm{tp}}
\def\equivL{\stackrel{\smash{\scalebox{.5}{\rm L}}}{\equiv}}
\def\swedge{\mathbin{\raisebox{.2ex}{\tiny$\mathbin\wedge$}}}
\def\svee{\mathbin{\raisebox{.2ex}{\tiny$\mathbin\vee$}}}

\newcommand{\labella}[1]{{\sf\footnotesize #1}\hfill}
\renewenvironment{itemize}
  {\begin{list}{$\triangleright$}{%
   \setlength{\parskip}{0mm}
   \setlength{\topsep}{0mm}
   \setlength{\rightmargin}{0mm}
   \setlength{\listparindent}{0mm}
   \setlength{\itemindent}{0mm}
   \setlength{\labelwidth}{3ex}
   \setlength{\itemsep}{0mm}
   \setlength{\parsep}{0mm}
   \setlength{\partopsep}{0mm}
   \setlength{\labelsep}{1ex}
   \setlength{\leftmargin}{\labelwidth+\labelsep}
   \let\makelabel\labella}}{%
   \end{list}}
%\def\ssf#1{\textsf{\small #1}}
\newcounter{ex}
\newenvironment{exercise}{\addtocounter{ex}{1}\textbf{Esercizio \theex.\quad}}{}
\pagestyle{empty}
\parindent0ex
\parskip2ex
\raggedbottom
\def\nsR{{}^*\!\RR}
\def\ssf#1{\textsf{#1}}
\renewcommand{\baselinestretch}{1.3}


\usepackage{fancyhdr}
\pagestyle{fancy}
\lfoot{Teoria dei Modelli a.a.~2021/22}
\rhead{}
\rfoot{\rput(1.5,-0.5){\small\thepage}}
\cfoot{}







%
% INDICARE NOME E COGNOME DI TUTTI GLI AUTORI
%
\lhead{Nome/i Cognome/i}
%
\begin{document}

  
\begin{exercise}
  Let $L = \{<\}$ and let $N$ be a $\omega_1$-saturated extension of $\QQ$. Prove that there is an embedding $f:\RR \to  N$. Is it elementary? Can it be an isomorphism?
\end{exercise}

\begin{exercise}
  Prove that every model of $T_{\rm acf}$ is $\omega$-ultrahomogeneous (independently of cardinality and transcendence degree).
\end{exercise}
  
\begin{exercise}
  Let $M$ and $N$ be elementarily homogeneous structures of the same cardinality $\lambda$. Suppose that $M\models\E x\, p(x)\,\IFF\,N\models\E x\, p(x)$ for every $p(x)\subseteq L$ such that $|x|<\lambda$. Prove that the two structures are isomorphic.
\end{exercise}

\begin{exercise}
  Let $\phi(x)\in L$. Prove that the following are equivalent
  \begin{itemize}
   \item[1.] $\phi(x)$ is equivalent to some $\psi(x)\in L_{\rm qf}$;
   \item[2.] $\phi(a)\iff \phi(fa)$ for every $a$ and every partial isomorphism $f:\U\to\U$ defined in $a$.
  \end{itemize}
  \vskip-1ex
  Use the result to prove Theorem~7.14 for $T$ complete.
  \end{exercise}
  


\end{document}


