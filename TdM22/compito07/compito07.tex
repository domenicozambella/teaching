\documentclass[10pt]{article}
% \usepackage[utf8]{inputenc}
\usepackage[a4paper,height=24cm,width=13cm]{geometry}
\usepackage{babel}
\usepackage{amssymb}
\usepackage{dsfont}
\usepackage{calc}
\usepackage{graphicx}
\usepackage{pst-node}
\usepackage{fourier-otf}
\usepackage{euscript}
\usepackage{amsmath,amsthm}

\def\lh{\textrm{lh}}
\def\phi{\varphi}
\def\P{\EuScript P}
\def\M{\EuScript M}
\def\D{\EuScript D}
\def\U{\EuScript U}
\def\V{\EuScript V}
\def\S{\EuScript S}
\def\sm{\smallsetminus}
\def\niff{\nleftrightarrow}
\def\ZZ{\mathds Z}
\def\NN{\mathds N}
\def\PP{\mathds P}
\def\QQ{\mathds Q}
\def\RR{\mathds R}
\def\<{\langle}
\def\>{\rangle}
\def\E{\exists}
\def\A{\forall}
\def\0{\varnothing}
\def\imp{\rightarrow}
\def\iff{\leftrightarrow}
\def\IMP{\Rightarrow}
\def\IFF{\Leftrightarrow}
\def\dom{\textrm{dom}}
\def\range{\textrm{rang}}
\def\Mod{\textrm{Mod}}
\def\Aut{\textrm{Aut}}
\def\Th{\textrm{Th}}
\def\acl{\textrm{acl}}
\def\eq{{\rm eq}}
\def\tp{\textrm{tp}}
\def\equivL{\stackrel{\smash{\scalebox{.5}{\rm L}}}{\equiv}}
\def\swedge{\mathbin{\raisebox{.2ex}{\tiny$\mathbin\wedge$}}}
\def\svee{\mathbin{\raisebox{.2ex}{\tiny$\mathbin\vee$}}}

\newcommand{\labella}[1]{{\sf\footnotesize #1}\hfill}
\renewenvironment{itemize}
  {\begin{list}{$\triangleright$}{%
   \setlength{\parskip}{0mm}
   \setlength{\topsep}{0mm}
   \setlength{\rightmargin}{0mm}
   \setlength{\listparindent}{0mm}
   \setlength{\itemindent}{0mm}
   \setlength{\labelwidth}{3ex}
   \setlength{\itemsep}{0mm}
   \setlength{\parsep}{0mm}
   \setlength{\partopsep}{0mm}
   \setlength{\labelsep}{1ex}
   \setlength{\leftmargin}{\labelwidth+\labelsep}
   \let\makelabel\labella}}{%
   \end{list}}
%\def\ssf#1{\textsf{\small #1}}
\newcounter{ex}
\newenvironment{exercise}{\addtocounter{ex}{1}\textbf{Esercizio \theex.\quad}}{}
\pagestyle{empty}
\parindent0ex
\parskip2ex
\raggedbottom
\def\nsR{{}^*\!\RR}
\def\ssf#1{\textsf{#1}}
\renewcommand{\baselinestretch}{1.3}


\usepackage{fancyhdr}
\pagestyle{fancy}
\lfoot{Teoria dei Modelli a.a.~2021/22}
\rhead{}
\rfoot{\rput(1.5,-0.5){\small\thepage}}
\cfoot{}







%
% INDICARE NOME E COGNOME DI TUTTI GLI AUTORI
%
\lhead{Nome/i Cognome/i}
%
\begin{document}

 

\begin{exercise}
  Prove that every type $p(x)\subseteq L(\U)$ that is strongly quasi-invariant over $M$ extends to a global type that is quasi-invariant over $M$.
\end{exercise}


\begin{exercise}
  Let  $p(x)\in S(\U)$ be a global type invariant over $A$.
  Let $a, b\models p_{\restriction A}(x)$.
  Prove that there is a sequence $\bar c=\<c_i:i<\omega\>$ such that $a,\bar c$ and $b,\bar c$ are both sequences of $A$-indiscernibles.
  \end{exercise}

  
\begin{exercise}
   For every $\D\subseteq\U^{|z|}$ the following are equivalent
   \begin{itemize}
   \item[1.] $\D$ is approximated by $\varphi(x\,;z)$;
   \item[2.] $\D$ is externally definable by $\varphi(x\,;z)$.
   \end{itemize}
\end{exercise}
  


\begin{exercise}
  Prove that the equivalence relation $a\stackrel{\smash{\scalebox{.5}{\rm L}}}{\equiv}_Ab$ is the transitive closure of the relation: there is a sequence $\<c_i:i<\omega\>$ indiscernible over $A$ such that $c_0=a$ and $c_1=b$.
\end{exercise}
\end{document}