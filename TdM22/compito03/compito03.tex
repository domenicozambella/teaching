\documentclass[10pt]{article}
\usepackage[utf8]{inputenc}
\usepackage[a4paper,height=24cm,width=13cm]{geometry}
\usepackage[italian]{babel}
\usepackage{amssymb}
\usepackage{dsfont}
\usepackage{calc}
\usepackage{graphicx}
\usepackage{pstricks}
\usepackage{pst-node}
\usepackage{fourier}
\usepackage{euscript}
\usepackage{amsmath,amssymb, amsthm}

\def\lh{\textrm{lh}}
\def\phi{\varphi}
\def\P{\EuScript P}
\def\M{\EuScript M}
\def\D{\EuScript D}
\def\U{\EuScript U}
\def\S{\EuScript S}
\def\sm{\smallsetminus}
\def\niff{\nleftrightarrow}
\def\ZZ{\mathds Z}
\def\NN{\mathds N}
\def\PP{\mathds P}
\def\QQ{\mathds Q}
\def\RR{\mathds R}
\def\<{\langle}
\def\>{\rangle}
\def\E{\exists}
\def\A{\forall}
\def\0{\varnothing}
\def\imp{\rightarrow}
\def\iff{\leftrightarrow}
\def\IMP{\Rightarrow}
\def\IFF{\Leftrightarrow}
\def\dom{\textrm{dom}}
\def\range{\textrm{rang}}
\def\Mod{\textrm{Mod}}
\def\Aut{\textrm{Aut}}
\def\Th{\textrm{Th}}
\def\acl{\textrm{acl}}
\def\eq{{\rm eq}}
\def\tp{\textrm{tp}}
\def\equivL{\stackrel{\smash{\scalebox{.5}{\rm L}}}{\equiv}}
\def\swedge{\mathbin{\raisebox{.2ex}{\tiny$\mathbin\wedge$}}}
\def\svee{\mathbin{\raisebox{.2ex}{\tiny$\mathbin\vee$}}}

\newcommand{\labella}[1]{{\sf\footnotesize #1}\hfill}
\renewenvironment{itemize}
  {\begin{list}{$\triangleright$}{%
   \setlength{\parskip}{0mm}
   \setlength{\topsep}{0mm}
   \setlength{\rightmargin}{0mm}
   \setlength{\listparindent}{0mm}
   \setlength{\itemindent}{0mm}
   \setlength{\labelwidth}{3ex}
   \setlength{\itemsep}{0mm}
   \setlength{\parsep}{0mm}
   \setlength{\partopsep}{0mm}
   \setlength{\labelsep}{1ex}
   \setlength{\leftmargin}{\labelwidth+\labelsep}
   \let\makelabel\labella}}{%
   \end{list}}
%\def\ssf#1{\textsf{\small #1}}
\newcounter{ex}
\newenvironment{exercise}{\addtocounter{ex}{1}\textbf{Esercizio \theex.\quad}}{}
\pagestyle{empty}
\parindent0ex
\parskip2ex
\raggedbottom
\def\nsR{{}^*\!\RR}
\def\ssf#1{\textsf{#1}}
\renewcommand{\baselinestretch}{1.3}


\usepackage{fancyhdr}
\pagestyle{fancy}
\lfoot{Teoria dei Modelli a.a.~2021/22}
\rhead{}
\rfoot{\rput(1.5,-0.5){\small\thepage}}
\cfoot{}







%
% INDICARE NOME E COGNOME DI TUTTI GLI AUTORI
%
\lhead{Nome/i Cognome/i}
%
\begin{document}


\begin{exercise}
  Let $M$ be a random graph. Prove that if $M_1,M_2\subseteq M$ are such that $M_1\sqcup M_2=M$, then $M_1$ or $M_2$ is a random graph.
\end{exercise}
  

\begin{exercise}
  Let $N$ be free union of two random graphs $N_1$ and $N_2$.
  That is, $N=N_1\sqcup N_2$ and $r^N= r^{N_1}\sqcup r^{N_2}$,
  where by $\sqcup$ we denote the disjoint union.
  \begin{itemize}
    \item[1.] Prove that $N_1$ is not definable without parameters.
    \item[2.] Prove that $N_1$ is definable (with parameters).
    \item[3.] Axiomatize (in words) $\Th(N)$.
  \end{itemize}
  \end{exercise}

\begin{exercise}
  A (countable) back-and-forth system between to models $M$ and $N$ is a set $\P$ of functions $k$ such that
  \begin{itemize}
      \item[0.] $k:M\to N$ is a finite partial embedding;
      \item[1a.] for every $b\in M$ there is $h\in\P$ such that $k\subseteq h$ and $b\in\dom h$;
      \item[1b.] for every $c\in N$ there is $h\in\P$ such that $k\subseteq h$ and $c\in\range h$.
  \end{itemize}
  Prove (briefly) that if $L$ countable and there is a back-and-forth system between $M$ and $N$ then $M\equiv N$.
\end{exercise}
  

\end{document}


