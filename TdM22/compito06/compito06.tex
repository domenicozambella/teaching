\documentclass[10pt]{article}
\usepackage[utf8]{inputenc}
\usepackage[a4paper,height=24cm,width=13cm]{geometry}
\usepackage[italian]{babel}
\usepackage{amssymb}
\usepackage{dsfont}
\usepackage{calc}
\usepackage{graphicx}
\usepackage{pstricks}
\usepackage{pst-node}
\usepackage{fourier}
\usepackage{euscript}
\usepackage{amsmath,amssymb, amsthm}

\def\lh{\textrm{lh}}
\def\phi{\varphi}
\def\P{\EuScript P}
\def\M{\EuScript M}
\def\D{\EuScript D}
\def\U{\EuScript U}
\def\V{\EuScript V}
\def\S{\EuScript S}
\def\sm{\smallsetminus}
\def\niff{\nleftrightarrow}
\def\ZZ{\mathds Z}
\def\NN{\mathds N}
\def\PP{\mathds P}
\def\QQ{\mathds Q}
\def\RR{\mathds R}
\def\<{\langle}
\def\>{\rangle}
\def\E{\exists}
\def\A{\forall}
\def\0{\varnothing}
\def\imp{\rightarrow}
\def\iff{\leftrightarrow}
\def\IMP{\Rightarrow}
\def\IFF{\Leftrightarrow}
\def\dom{\textrm{dom}}
\def\range{\textrm{rang}}
\def\Mod{\textrm{Mod}}
\def\Aut{\textrm{Aut}}
\def\Th{\textrm{Th}}
\def\acl{\textrm{acl}}
\def\eq{{\rm eq}}
\def\tp{\textrm{tp}}
\def\equivL{\stackrel{\smash{\scalebox{.5}{\rm L}}}{\equiv}}
\def\swedge{\mathbin{\raisebox{.2ex}{\tiny$\mathbin\wedge$}}}
\def\svee{\mathbin{\raisebox{.2ex}{\tiny$\mathbin\vee$}}}

\newcommand{\labella}[1]{{\sf\footnotesize #1}\hfill}
\renewenvironment{itemize}
  {\begin{list}{$\triangleright$}{%
   \setlength{\parskip}{0mm}
   \setlength{\topsep}{0mm}
   \setlength{\rightmargin}{0mm}
   \setlength{\listparindent}{0mm}
   \setlength{\itemindent}{0mm}
   \setlength{\labelwidth}{3ex}
   \setlength{\itemsep}{0mm}
   \setlength{\parsep}{0mm}
   \setlength{\partopsep}{0mm}
   \setlength{\labelsep}{1ex}
   \setlength{\leftmargin}{\labelwidth+\labelsep}
   \let\makelabel\labella}}{%
   \end{list}}
%\def\ssf#1{\textsf{\small #1}}
\newcounter{ex}
\newenvironment{exercise}{\addtocounter{ex}{1}\textbf{Esercizio \theex.\quad}}{}
\pagestyle{empty}
\parindent0ex
\parskip2ex
\raggedbottom
\def\nsR{{}^*\!\RR}
\def\ssf#1{\textsf{#1}}
\renewcommand{\baselinestretch}{1.3}


\usepackage{fancyhdr}
\pagestyle{fancy}
\lfoot{Teoria dei Modelli a.a.~2021/22}
\rhead{}
\rfoot{\rput(1.5,-0.5){\small\thepage}}
\cfoot{}







%
% INDICARE NOME E COGNOME DI TUTTI GLI AUTORI
%
\lhead{Nome/i Cognome/i}
%
\begin{document}

\begin{exercise}\label{ex_infinite_acl}
  ($T$ strongly minimal.) Prove that every infinite algebraically closed set is a model.
\end{exercise}


\begin{exercise}
  Let $p(x)\subseteq L(A)$ and let $\phi(x\,;y)\in L(A)$ be a formula that defines, when restricted to $p(\U)$, an equivalence relation with finitely many classes.
  Prove that there is a finite equivalence relation definable over $A$ that coincides with $\phi(x\,;y)$ on $p(\U)$.
\end{exercise}

\begin{exercise}
  Let $M$ be a graph with the property that for every finite $A\subseteq M$ there is a $c\in M$ such that $A\subseteq r(c,\U)$. 
  (This holds in particular when $M$ is a random graph.)
  %
  A star in $M$ is a subgraph whose edges all share a common vertex. We say that a coloring of the edges of $M$ is locally finite if there is a $k$ such that every star has at most $k$ colors.
  Prove that for every locally finite coloring of the edges of $M$, there is an infinite monochromatic complete subgraph.
\end{exercise}


\begin{exercise}
  Let $T$ have elimination of imaginaries and $\phi(x\,;z)\in L(A)$.
  For arbitrary $c\in\U^{|z|}$, prove that if the orbit of $\phi(\U\,; c)$ over $A$ is finite, then $\phi(\U\,; c)$ is definable over $\acl A$.
  \end{exercise}
\end{document}


